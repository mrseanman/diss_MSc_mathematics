\documentclass[class=article, crop=false]{standalone}
% For whatever reason these must be in here as well as in my_preamble
\usepackage[subpreambles=true]{standalone}
\usepackage{import}

\usepackage{sty/my_preamble}


\begin{document}

Given an edge labelled poset \poset P, we can construct a group generated by the labels of $P$ with relations equating words forming loops in the \hassediag{} of $P$. For example


% filled small circle
\tikzstyle{FSM}=[circle,draw=black!50,fill=black!20,thick, inner sep=0pt,minimum size=1.5mm]
\tikzstyle{a}=[red]
\tikzstyle{b}=[blue]

\begin{center}
\begin{tikzpicture}
    \begin{scope}[on grid]
    \node[FSM] (base)             at (0,0)    [label=below:$\emptyset$]                     {};
    \node[FSM] (bottom left)            [above left = 1cm and 1cm of base,
                                            label=left:{$\{1\}$}]                             {};
    \node[FSM] (bottom right)           [right = 2cm of bottom left,
                                            label=right:{$\{2$\}}]                            {};
    \node[FSM] (top left)               [above = 1.5cm of bottom left,
                                            label=left:{$\{1,2\}$}]                           {};
    \node[FSM] (top right)              [above = 1.5cm of bottom right,
                                            label=right:{$\{2,3\}$}]                          {};
    \node[FSM] (top)                    [above right = 1cm and 1cm of top left,
                                            label=above:{$\{1,2,3\}$}]                             {};
    \end{scope}
s
    \draw[a] (base) to node[auto] {a} (bottom left);
    \draw[b] (base) to node[auto, swap] {b} (bottom right);
    \draw[b] (bottom left) to node[auto] {b} (top left);
    \draw[a] (bottom right) to node[auto, swap] {a} (top right);
    \draw[a] (top left) to node[auto] {a} (top);
    \draw[b] (top right) to node[auto, swap] {b} (top);
\end{tikzpicture}
\end{center}

Where we have taken $\leq$ to be $\subseteq$. We have that this group is
\begin{equation}
	G = \left\langle\,
	a,b \,|\, aba=bab
	\,\right\rangle .
\end{equation}

We can construct a cell complex $K$ from $P$ such that $\pi(1, K)$ is $G$. We do this by initially creating a simplicial complex $\Delta(P)$, where each $n$ simplex is an $n$--chain of $P$. We define an $n$--chain to have $n-1$ elements. E.g.~$\{\{1,2\}\}$ is a 0--chain here (the $n$ corr.~to the number of comparisons). In this example $\Delta(P)$ would be two solid tetrahedrons sharing an edge corr.~to the chain $\{\emptyset \subseteq \{1,2,3\}\}$.

For an example that requires slightly less 3-D drawing consider the following poset $P$ and corr.~$\Delta(P)$. Here we forget about edge labelling in $P$ for a moment.

\begin{center}
	\scalebox{1.15}{
	\begin{tikzpicture}
		\tikzstyle{every label}=[font=\footnotesize]
		\begin{scope}[on grid]
			\node[FSM] (base left)    	at (0,0)   	[label=below:{$\{2\}$}]   	{};
			\node[FSM] (base right)      		    [right=1cm of base left,
														label=below:{$\{1\}$}]   	{};
														
			\node[FSM] (middle left)    	    	[above left=0.866cm and 0.5cm of base left,
														label=left:{$\{2,4\}$}]   	{};
			\node[FSM] (middle middle)				[right=1cm of middle left,
														label=above right:{$\{1,2\}$}]	{};
			\node[FSM] (middle right)				[right=1cm of middle middle,
														label=right:{$\{1,4\}$}]	{};
														
			\node[FSM] (top)						[above right=0.866cm and 0.5cm of middle left,
														label=above:{$\{1,2,4\}$}]	{};
														
			\draw	(base left) to (middle left);
			\draw	(base left) to (middle middle);
			\draw	(base right) to (middle middle);
			\draw 	(base right) to (middle right);
			\draw	(middle left)	to (top);
			\draw 	(middle middle)	to (top);
		\end{scope}
\end{tikzpicture} \hspace{1cm}
\tikzstyle{green polyfill}=[fill=green!20, draw=green!50!black, thick]
\begin{tikzpicture}
	\tikzstyle{every label}=[font=\footnotesize]
	\begin{scope}[on grid]
		\node[FSM] (base left)    	at (0,0)   	[label=below:{$\{1,2\}$}]   	{};
		\node[FSM] (base middle)      		    [right=1cm of base left,
		label=below:{$\{1\}$}]   	{};
		\node[FSM] (base right)      		    [right=1cm of base middle,
		label=below:{$\{1,4\}$}]   	{};	
		
		\node[FSM] (middle left)    	    	[above left=0.866cm and 0.5cm of base left,
		label=left:{$\{2\}$}]   	{};
		\node[FSM] (middle middle)    	    	[right=1cm of middle left,
		label=right:{$\{1,2,4\}$}]   {};
		
		\node[FSM] (top)						[above right=0.866cm and 0.5cm of middle left,
		label=above:{$\{2,4\}$}]		{};
		
		\begin{pgfonlayer}{background}
			\filldraw[green polyfill] (base left.center) -- (middle left.center) -- (middle middle.center) -- cycle;
			\filldraw[green polyfill] (middle left.center) -- (middle middle.center) -- (top.center) -- cycle;
			\filldraw[green polyfill] (base left.center) -- (middle middle.center) -- (base middle.center) -- cycle;
			\draw[green polyfill] (base middle) to (base right);		
		\end{pgfonlayer}
		
		
	\end{scope}
\end{tikzpicture}
}
\end{center}

This simplicial complex $\Delta(P)$ will be retractable to the base minimum element in $P$ (existence of a minimum element is a requirement of being a Garside structure) in our case. To get a space with a more interesting topology, we use the edge labelling of $P$ to generate a quotient of $\Delta(P)$.


\end{document}
