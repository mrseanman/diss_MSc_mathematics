\documentclass[class=article, crop=false]{standalone}
% For whatever reason these must be in here as well as in my_preamble
\usepackage[subpreambles=true]{standalone}
\usepackage{import}

\usepackage{sty/my_preamble}


\begin{document}

We define an edge labelled poset to be a triple $(P,\leq,l)$ where $P$ is our set and $\leq$ some truth valued function on $P\times P$ as normal. The function $l: \text{Cov}(P) \to A$ is the data of our labels, where $\text{Cov}(P) \coloneqq \{(p,q) \in P \times P \mid p \lessdot q\}$ is the set of \emph{covered pairs} and $A$ is some alphabet of labels. We will use $P$ as a shorthand for $(P,\leq,l)$ where possible. Given an edge labelled poset $P$, we can construct a group generated by the labels of $P$ with relations equating words forming loops in the \hassediag{} of $P$. For example


% filled small circle
\tikzstyle{FSC}=[circle,draw=black!50,fill=black!20,thick, inner sep=0pt,minimum size=1.5mm]
\tikzstyle{a}=[red]
\tikzstyle{b}=[blue]

\begin{center}
\begin{tikzpicture}
    \node[FSC] (base)          	at (0,0)    					[label=below:$\emptyset$]         	{};
    \node[FSC] (bottom left)    at ($(base) + (-0.8,1) $)   	[label=left:{$\{1\}$}]         		{};
    \node[FSC] (bottom right)   at ($(bottom left) + (1.6,0)$)  [label=right:{$\{2\}$}]             {};
    \node[FSC] (top left)       at ($(bottom left) + (0,1)$)    [label=left:{$\{1,2\}$}]          	{};
    \node[FSC] (top right)    	at ($(bottom right) + (0,1)$)  	[label=right:{$\{2,3\}$}]          	{};
    \node[FSC] (top)          	at ($(top left) + (0.8,1)$)    	[label=above:{$\{1,2,3\}$}]   		{};

    \draw[a] (base) 		to 	node[auto] 			{a} 	(bottom left);
    \draw[b] (base) 		to 	node[auto, swap] 	{b} 	(bottom right);
    \draw[b] (bottom left) 	to 	node[auto] 			{b} 	(top left);
    \draw[a] (bottom right) to 	node[auto, swap] 	{a} 	(top right);
    \draw[a] (top left) 	to 	node[auto] 			{a} 	(top);
    \draw[b] (top right) 	to 	node[auto, swap] 	{b} 	(top);
\end{tikzpicture}
\end{center}

Where we have taken $\leq$ to be $\subseteq$. Here the group is

\begin{equation}
	G = \left\langle\,
	a,b \mid aba=bab
	\,\right\rangle .
\end{equation}

We can construct a cell complex $K$ from $P$ such that $\pi(1, K)$ is $G$. We do this by initially creating a simplicial complex $\Delta(P)$, where each $n$ simplex is an $n$--chain of $P$. We define an $n$--chain to have $n-1$ elements. E.g.~$(\{1,2\})$ is a 0--chain here (the $n$ corr.~to the number of comparisons). In this example $\Delta(P)$ would be two solid tetrahedrons sharing an edge (a 1--simplex) corr.~to the 1--chain $(\emptyset \subseteq \{1,2,3\})$.

For a more two dimensional example consider the following poset $P$ and corr.~$\Delta(P)$. Here we forget about edge labelling in $P$ for a moment.

\begin{center}
\begin{tikzpicture}[scale=1.5]
	\tikzstyle{every label}=[font=\footnotesize]
	\node[FSC] (base left)    	at (0,0)   							[label=below:{$\{2\}$}]   	{};
	\node[FSC] (base right)		at ($(base left) + (1,0)$)  		[label=below:{$\{1\}$}]   	{};							
	\node[FSC] (middle left)	at ($(base left) + (-0.5, 0.866)$)  [label=left:{$\{2,3\}$}]   	{};
	\node[FSC] (middle middle)	at ($(middle left) + (1,0)$)		[label={[label distance=-4]40:{$\{1,2\}$}}]	{};
	\node[FSC] (middle right)	at ($(middle middle) + (1,0)$)		[label=right:{$\{1,4\}$}]	{};
	\node[FSC] (top)			at ($(middle left) + (0.5, 0.866)$)	[label=above:{$\{1,2,3\}$}]	{};
												
	\draw	(base left) to (middle left);
	\draw	(base left) to (middle middle);
	\draw	(base right) to (middle middle);
	\draw 	(base right) to (middle right);
	\draw	(middle left)	to (top);
	\draw 	(middle middle)	to (top);
\end{tikzpicture}
\hspace{1cm}
\tikzstyle{green polyfill}=[fill=green!20, draw=green!50!black, thick]
\begin{tikzpicture}[scale = 1.5]
	\tikzstyle{every label}=[font=\footnotesize]
	\node[FSC] (base left)    	at (0,0)   							[label=below:{$\{1,2\}$}]   {};
	\node[FSC] (base middle)   	at ($(base left) + (1,0)$)   		[label=below:{$\{1\}$}]   	{};
	\node[FSC] (base right)    	at ($(base middle) + (1,0)$)  		[label=below:{$\{1,4\}$}]   {};
	\node[FSC] (middle left)	at ($(base left) + (-0.5, 0.866)$) 	[label=left:{$\{2\}$}]   	{};
	\node[FSC] (middle middle)  at ($(middle left) + (1,0)$)  	    [label=right:{$\{1,2,3\}$}]	{};
	\node[FSC] (top)			at ($(middle left) + (0.5, 0.866)$)	[label=above:{$\{2,3\}$}]	{};
	
	\begin{pgfonlayer}{background}
		\filldraw[green polyfill] (base left.center) -- (middle left.center) -- (middle middle.center) -- cycle;
		\filldraw[green polyfill] (middle left.center) -- (middle middle.center) -- (top.center) -- cycle;
		\filldraw[green polyfill] (base left.center) -- (middle middle.center) -- (base middle.center) -- cycle;
		\draw[green polyfill] (base middle) to (base right);		
	\end{pgfonlayer}	
\end{tikzpicture}
\end{center}

In this particular example, the simplicial complex is contractible. In fact, since Garside structures require a minimal element $\hat{0} \in P$, the simplicial complex of any Garside structure will contract to this minimum element vertex in $\Delta(P)$. To get a space with a more interesting topology, we use the edge labelling of $P$ to generate a quotient of $\Delta(P)$.

Now let us put some arbitrary edge labelling on $P$ to progress with this. The edge labelling and corresponding quotient space are shown.

\begin{center}
\begin{tikzpicture}[scale=1.5]
	\tikzstyle{every label}=[font=\footnotesize]
	\node[FSC] (base left)    	at (0,0)   							[label=below:{$\{2\}$}]   	{};
	\node[FSC] (base right)		at ($(base left) + (1,0)$)  		[label=below:{$\{1\}$}]   	{};							
	\node[FSC] (middle left)	at ($(base left) + (-0.5, 0.866)$)  [label=left:{$\{2,3\}$}]   	{};
	\node[FSC] (middle middle)	at ($(middle left) + (1,0)$)		[label={[label distance=-4]40:{$\{1,2\}$}}]	{};
	\node[FSC] (middle right)	at ($(middle middle) + (1,0)$)		[label=right:{$\{1,4\}$}]	{};
	\node[FSC] (top)			at ($(middle left) + (0.5, 0.866)$)	[label=above:{$\{1,2,3\}$}]	{};
	
	\tikzstyle{every node}=[font=\footnotesize]
	\draw[a]	(base left) 	to 	node[left] 			{a} 	(middle left);
	\draw[b]	(base left) 	to 	node[right] 		{b}		(middle middle);
	\draw[b]	(base right) 	to 	node[right] 		{b}		(middle middle);
	\draw[a] 	(base right) 	to 	node[right] 		{a}		(middle right);
	\draw[b]	(middle left)	to 	node[left] 			{b} 	(top);
	\draw[a] 	(middle middle)	to 	node[right] 	{a}		(top);
\end{tikzpicture}
\hspace{1cm}
\tikzstyle{green polyfill}=[fill=green!20, draw=green!50!black, thick]
\begin{tikzpicture}[scale = 1.5]
	\tikzstyle{every label}=[font=\footnotesize]
	\node[FSC] (base left)    	at (0,0)   							[label=below:{$\{1,2\}$}]   {};
	\node[FSC] (base middle)   	at ($(base left) + (1,0)$)   		[label=below:{$\{1\}$}]   	{};
	\node[FSC] (base right)    	at ($(base middle) + (1,0)$)  		[label=below:{$\{1,4\}$}]   {};
	\node[FSC] (middle left)	at ($(base left) + (-0.5, 0.866)$) 	[label=left:{$\{2\}$}]   	{};
	\node[FSC] (middle middle)  at ($(middle left) + (1,0)$)  	    [label=right:{$\{1,2,3\}$}]	{};
	\node[FSC] (top)			at ($(middle left) + (0.5, 0.866)$)	[label=above:{$\{2,3\}$}]	{};
	
	\begin{scope}[thick]
	\draw[a, thick, arrow_me=stealth] (base left) 	to (middle middle);
	\draw[a, arrow_me=stealth] (middle left) 	to (top);
	\draw[a, arrow_me=stealth] (base middle) 	to (base right);

	\draw[b, arrow_me=>>s] (middle left) to (base left);
	\draw[b, arrow_me=>>s] (top) to (middle middle);
	\draw[b, arrow_me=>>s] (base middle) to (base left);
	
	\draw (middle left) to (middle middle);
	\draw (base middle) to (middle middle); 
	\end{scope}

	
	\node at ($1/3*(base left) + 1/3*(base middle) + 1/3*(middle middle)$) {$\circlearrowright$};
	\node at ($1/3*(middle left) + 1/3*(base left) + 1/3*(middle middle)$) {$\circlearrowleft$};
		
	\begin{pgfonlayer}{background}
		\fill[fill=orange!20] (base left.center) -- (middle left.center) -- (middle middle.center) -- cycle;
		\fill[fill=orange!20] (base left.center) -- (middle middle.center) -- (base middle.center) -- cycle;
		\fill[fill=black!15] (middle left.center) -- (middle middle.center) -- (top.center) -- cycle;
		
	\end{pgfonlayer}	
\end{tikzpicture}
\end{center}

To generate this quotient space, first we define and \emph{extended labelling} $\mathcal{L}(\rho)$ on chains $\rho$, based on our edge labelling $l$. We define $\mathcal{L}(\rho) \subset A^*$ to be the language of all words corr.~to all saturated chains that contain every element of $\rho$. E.g.~considering the chain $\{\{2\} \subseteq \{1,2,3\}\}$, there are two corresponding saturated chains, $(\{2\} \subseteq \{1,2\} \subseteq \{1,2,3\})$ and $(\{2\} \subseteq \{2,3\} \subseteq \{1,2,3\})$ which respectively corr.~to the words $ba$ and $ab$. So $\mathcal{L}(\{2\} \subseteq \{1,2,3\}) = \{ba, ab\}$. Here are some examples:

\begin{itemize}
	\item $\mathcal{L}(\{1\} \subseteq \{1,2\}) = \mathcal{L}(\{2\} \subseteq \{1,2\}) = \{b\}$.
	\item $\mathcal{L}(\{1\} \subseteq \{1,2,3\}) = \{ba\} \neq \mathcal{L}(\{2\} \subseteq \{1,2,3\})$.
	\item $\mathcal{L}(\{1\} \subseteq \{1,2\} \subseteq \{1,2,3\}) = \mathcal{L}(\{2\} \subseteq \{1,2\} \subseteq \{1,2,3\}) = \{ba\}$.
	\item $\mathcal{L}(\{1\}) = \mathcal{L}(\{2\}) = \mathcal{L}(\{1,2\}) = \cdots = \emptyset$.
\end{itemize}

Now, using this labelling and the orientation induced on a chain by $\leq$, we identify simplicies corr.~to chains which have the same extended label $\mathcal{L}$. In the above example, three red edges are identiefied, three blue edges are identified and two orange triangles are identified. Note that the two black edges are in this case identified, but only because they belong to the two identified triangles. In the second example above, we see they have different labels.

We see that this space is homeomorphic to a torus, which has $\pi(1) \cong \mathbb{Z}^2 \cong \left\langle a,b \mid ab = ba \right\rangle$, which is also the group defined by this poset. 

\end{document}
