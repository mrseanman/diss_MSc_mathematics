\documentclass[class=article, crop=false]{standalone}
% For whatever reason these must be in here as well as in my_preamble
\usepackage[subpreambles=true]{standalone}
\usepackage{import}

\usepackage{sty/my_preamble}


\begin{document}
	
Here we compile many theorems from \cite{paolini_salvetti_kpi1_2021} in to one theorem.


\begin{theorem}[{\cite{paolini_salvetti_kpi1_2021}}]
	Given an affine Coxeter group $W$, the configuration space $Y_W$ is homotopy equivalent to the order complex $K_W$.
	\label{thm:proof_overview}
\end{theorem}
\begin{proof}
	This is done through a composition of homotopy equivalences
	\begin{equation}
		Y_W \labelrel\simeq{eqmid:configspace_salvetti}
		X_w \labelrel\simeq{eqmid:salvetti_salvettiprime}
		X^\prime_w \labelrel\simeq{eqmid:salvettiprime_intervalcomplexprime}
		K^\prime_W \labelrel\simeq{eqmid:intervalcomplexprime_intervalcomplex}
		K_W
	\end{equation}

	Where the results are gathered from the following sources:
	
	\eqref{eqmid:configspace_salvetti}: \cite[Theorem 1]{salvetti_topology_1987} \quad
	\eqref{eqmid:salvetti_salvettiprime}: \cite[Theorem 5.5]{paolini_salvetti_kpi1_2021} \quad
	\eqref{eqmid:salvettiprime_intervalcomplexprime}: \cite[Theorem 8.14]{paolini_salvetti_kpi1_2021} \quad
	\eqref{eqmid:intervalcomplexprime_intervalcomplex}: \cite[Theorem 7.9]{paolini_salvetti_kpi1_2021} \quad
\end{proof}

Furthermore, in the same paper another main result is shown.

\begin{theorem}[{\cite[Theorem 6.6]{paolini_salvetti_kpi1_2021}}]
	Given an affine Coxeter group $W$, corresponding affine Artin group $G_W$ and Coxeter element $w\in W$, the complex $K_W$ is a classifying space for the dual artin group $W_w$. I.e.
	\[
		K_W \simeq K(W_w, 1)
	\]
	\label{thm:KW_classifyingspace}
\end{theorem}




It was already known \cite{brieskorn_fundamentalgruppe_1971} that $\pi_1(Y_W) = G_W$. Thus considering $\pi_1(Y_W)$ and combining \cref{thm:proof_overview,thm:KW_classifyingspace} gives
\begin{align}
	Y_W &\simeq K(G_W,1)\\
	G_W &\cong W_w
	\label{eq:artin_iso_dual}
\end{align}
for affine $G_W$.

This proves the $K(\pi, 1)$ conjecture for affine Artin groups and provides a new proof than an affine Artin group is naturally isomorphic to its dual, which was already known for finite \cite{bessis_dual_2003} and affine \cite{mccammond_sulway_artin_2017} cases.

The proof of $\pi_1(Y_W) \cong G_W$ for all $W$ in \cite{brieskorn_fundamentalgruppe_1971} is in German and only German or Russian translations are available. This result is fundamental and non-trivial. Alternative proofs for Coxeter groups of type $A_n$ \cite{fox_braid_1962} or affine type \cite{viet_dung_fundamental_1983} are available in English. Here we will repeat roughly the proof in \cite{fox_braid_1962}, revealing the reason for the choice of name \emph{configuration space} for $Y_W$.

\end{document}