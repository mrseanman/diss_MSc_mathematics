\documentclass[class=article, crop=false]{standalone}
% For whatever reason these must be in here as well as in my_preamble
\usepackage[subpreambles=true]{standalone}
\usepackage{import}

\usepackage{sty/my_preamble}


\begin{document}

Here we will give the definition of the configuration space $Y_W$ for a given Coxeter group $W$ and explain the name \emph{configuration space} by going through an example where we will show that $\pi_1(Y_W) \cong G_W$ for $W$ of type $A_n$ following from \cite{fox_braid_1962}.

For some Coxeter group $W$ acting on $\R^n$, the set of reflections $R \in W$ acts on $\R^n$ by reflection through hyperplanes. For some $r \in R$ denote its hyperplane by $H(r) \subseteq \R^n$. Dentote the union of all hyperplanes by $\mathcal{H} \coloneqq \bigcup_{r \in R} H(r)$. We associate $\R^n \otimes \C$ with $\C^n$ under the natural isomorphism. This also extends the action $W\acts \R^n$ to $W\acts \C^n$ via $w\cdot(x \otimes \lambda) = (w\cdot x)\otimes \lambda$. We can then make our definition.

\begin{definition}[Configuration space] 
For some Coxeter group $W$ and associated hyperplane system $\mathcal{H}$ as above, we define
\begin{equation*}
    Y \coloneqq \C^n \, \setminus \, \left( \mathcal{H} \otimes \C \right)
\end{equation*}
and define the configuration space $Y_W$ to be the quotient $Y/W$.
\label{eq:config_space_def}
\end{definition}

\subsubsection{Proof that $\pi_1(Y_W) \cong G_W$ for $A_n$ type. }

The Coxeter group $A_n$ is naturally isomorphic to $S_{n+1}$ and so here we will talk about $S_n$ and use its associated cycle notation. Given $\R^n$ with basis $\{e_i\}$, the relevant action $S_n \acts \R^n$ is by permutation of components with respect to that basis.
The set of reflections $R$ of $S_n$ is all conjugations of the generating adjacent transpositions $(l, l+1)$, which is to say $R$ is the set of all transpositions $(l, n)$. Some $(l, n) \in R$ acts on $\R^n$ as reflection through the plane $P_{(l,n)} = \{ (x_1, \ldots, x_n ) \in \R^n \mid x_l = x_n\}$. Thus here we have $Y = \{(\mu_1, \ldots \mu_n) \in \C^n \mid \forall i,j \,\,\, \mu_i \neq \mu_j\}$. We can think of this as the space of $n$ distinct labelled points in $\C$. The action $W \acts \C^n$ is also permutation components, so we can think of the configuration space $Y_W$ as the set of $n$ distinct \emph{unlabelled} points in $\C$.

In this way, the natural picture for a point in $Y_W$ is $n$ unlabelled dots in $\C$, which we think of as the plane. We can draw a path in $Y_W$ by drawing an arrow from a starting dot to a terminal dot for each of these dots in the plane. These paths can cross on the completed picture, just as long as at a given time $t$, all dots are distinct. In this way, we see that any continuous injection $f \colon \C \to \C$ corresponds to a map $\bar{f} \colon Y_W \to Y_W$ and any continous inection $h \colon \C \times I \to \C$ corresponds to a homotopy $\bar{\text{Id}_\C} \sim h(1,\_) $



\end{document}