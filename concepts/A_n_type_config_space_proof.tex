% !TeX root = ../main.tex
\documentclass[class=article, crop=false]{standalone}
\begin{document}

Here we will give the definition of the configuration space $Y_W$ for a given Coxeter group $W$ and explain the name \emph{configuration space} by going through an example where we will show that $\pi_1(Y_W) \cong G_W$ for $W$ of type $A_n$ following from \cite{fox_neuwirth_braid_1962}.

For some Coxeter group $W$ acting on $\R^n$, the set of reflections $R \in W$ acts on $\R^n$ by reflection through hyperplanes. For some $r \in R$ denote its hyperplane by $H(r) \subseteq \R^n$. Denote the union of all hyperplanes by $\mathcal{H} \coloneqq \bigcup_{r \in R} H(r)$. We associate $\R^n \otimes \C$ with $\C^n$ under the natural isomorphism. This also extends the action $W\acts \R^n$ to $W\acts \C^n$ via $w\cdot(x \otimes \lambda) = (w\cdot x)\otimes \lambda$. We can then make our definition.

\begin{definition}[Configuration space] 
For some Coxeter group $W$ and associated hyperplane system $\mathcal{H}$ as above, we define
\begin{equation}
    Y \coloneqq \C^n \, \setminus \, \left( \mathcal{H} \otimes \C \right)
\end{equation}
and define the configuration space $Y_W$ to be the quotient $Y/W$.
\label{eq:config_space_def}
\end{definition}

For an understanding of the name \emph{configuration space} and for a concrete example, we will show what the space $Y_W$ is for Coxeter groups of type $A_n$.

The Coxeter group $A_{n}$ is naturally isomorphic to $S_{n+1}$. Here we will use the associated cycle notation in $S_n$ to talk about elements of $A_n$. Given $\R^n$ with basis $\{e_i\}$, the relevant action $A_n \acts \R^n$ is by permutation of components with respect to that basis.
The set of reflections $R$ of $A_n$ is all conjugations of the $n$ adjacent generating transpositions $(l, l+1)$. Which is to say, $R$ is the set of all transpositions $(l, n)$. Some $(l, n) \in R$ acts on $\R^n$ as reflection through the plane $P_{(l,n)} = \{ (x_1, \ldots, x_n ) \in \R^n \mid x_l = x_n\}$. Thus here we have $Y = \{(\mu_1, \ldots \mu_n) \in \C^n \mid \forall i,j \,\,\, \mu_i \neq \mu_j\}$. We can think of this as the space of $n$ distinct labelled points in $\C$. The action $A_n \acts \C^n$ is also permutation components, so we can think of the configuration space $Y_{W}$ as the set of $n$ distinct \emph{unlabelled} points in $\C$, which we will denote $X^n$.

Historically, Emile Artin \cite{artin_braids_1947} originally defined the braid group $B_n$ to be $\pi_1(X^n)$. He then showed the validity of the well known presentation of the braid group. In this context, showing the validity of that presentation in turn proves $B_n = G_W$. This proof by Artin is often considered dubious and other proofs are available. One good example is \cite{fox_neuwirth_braid_1962}.

In this way, the natural picture for a point in $X^n$ is $n$ unlabelled dots in $\C$, which we will draw in the plane. We can draw a path in $X^n$ by drawing an arrow from a starting dot to a terminal dot for each of these dots in the plane. These paths can cross on the completed picture, just as long as at a given time $t$, all dots are distinct. If the paths do not intersect in the final picture then a valid path in $X^n$ is guaranteed. With this picture in mind, we see that any continuous injection $f \colon \C \to \C$ corresponds to a continuous map $\bar{f} \colon X^n \to X^n$ and any continuous injection $h \colon \C \times I \to \C$ corresponds to a homotopy $h|_{\C\times \{0\}} \sim h|_{\C\times \{1\}}$.

When considering paths in $X^n$ up to homotopy it is beneficial to use the point $(1,2,3,\ldots,n) \in X^n$, which we call the \emph{ordered point}, as our base point. As in \cref{fig:example_path_in_X4}, we can form a curvefrom any point to the ordered. Hopefully it is clear that this is always possible using straight lines given a sensible choice when pairing up the dots. So $X^n$ is path connected.

\begin{figure}[h]
\centering
\begin{tikzpicture}

  \draw[help lines, color=gray!30, dashed] (-1.9,-1.9) grid (4.9, 4.9);
  \draw[->,ultra thick] (-2,0)--(5,0) ;
  \draw[->,ultra thick] (0,-2)--(0,5) ;
    
	\node[FSC] (start 1)     at (-1.2, 1.5)	   {};
	\node[FSC] (start 2)     at (-1.2, 3.1)    {};
	\node[FSC] (start 3)     at (1.9, -1.4)    {};
	\node[FSC] (start 4)     at (4.6, 3.9)     {};

  \node[FSC] (end 1)       at (1, 0)         {};
	\node[FSC] (end 2)       at (2, 0)         {};
	\node[FSC] (end 3)       at (3, 0)         {};
	\node[FSC] (end 4)       at (4, 0)         {};

	\draw[arrow_me=stealth] (start 1) .. controls (-1,0) and (0.5,2) .. (end 1);
  \draw[arrow_me=stealth] (start 2) .. controls (0,4) and (1,1) ..	(end 2);
  \draw[arrow_me=stealth] (start 3) .. controls (2,-1) and (3,-1) ..	(end 3);
  \draw[arrow_me=stealth] (start 4) to (end 4);

\end{tikzpicture}
	\caption{A path in $X^4$ where the terminal point is the ordered point}
    \label{fig:example_path_in_X4}
\end{figure}


\end{document}
