\documentclass[class=article, crop=false]{standalone}
\usepackage{sty/my_preamble}

\begin{document}
Starting from a Coxeter group $W$ generated by $S$, we wish to give $W$ a labelled--poset structure and use the constructions from the previous section. The edge labelled Hasse diagram for $W$ will embed in to the Cayley graph $\Cay(W,S)$, and it is useful to be able to swap between these two objects, as we will do. First we must define an order on our group $W$.
\begin{definition}[Word length in a group]
	For a group $G$ generated by $S$, the word length with respect to $S$ is the function $l_S:G\to \mathbb{Z}$ where $l_S(g) = \min\{k \mid s_1s_2\ldots s_k=g \,,\, s_i \in S \}$.
\end{definition}

We will often omit the $S$ in $l_S$.

\begin{definition}[Order on a group]
	On a group $G$ we define the order $x \leq y \iff l(x) + l(x^{-1}y) = l(y)$.
\end{definition}

It can be readily checked that this does indeed define an order on $G$. This order encodes closeness to $e \in G$ along geodesics in $\Cay(G,S)$. We have $x \leq y$ precisely when there exists a geodesic in $\Cay(G, S)$ from $e$ to $y$ with $x$ as an intermediate vertex.

For some $w \in W$, the poset $[1,w]^W$ (now no longer a group) is simply the interval $[1,w]$ with respect to this order. We now define the precise $w \in W$ for which we want to make this construction.

\begin{definition}[Coxeter element]
	For some Coxeter group $W$ for which $R \subseteq W$ is all reflections in $W$, we define a coxeter element $w\in W$ to be any product of all the elements of $R$.
\end{definition}

These Coxeter elements are what we will use as the upper bound of our interval. In principal there are many choices of Coxeter element depending on what order we multiply the elements of $R$. However, we will see that in many cases these choices necessarily result in isomorphic $[1,w]^W$. We see that $S \subseteq R$, in particular $R$ generates $W$, since in the standard presentation of Coxeter groups all of $S$ are reflections.

The interval $[1,w]^W$ is given the obvious edge labelling that makes it a subgraph of $\Cay(R,S)$, so two connected vertices $g$ and $gs$ will be labelled by $s \in S$. With this edge labelling we now define a new group $W_w$ constructed from the geometry of $[1,w]^W$.

\begin{definition}[Interval group]
	Given a Coxeter group $W$ and Coxeter element $w\in W$, construct the edge--labelled poset $[1,w]^W$ as above. The interval group $W_w$ is the poset group (as in \cref{def:poset_group}) corresponding to $[1,w]^W$.
\end{definition}

\end{document}
