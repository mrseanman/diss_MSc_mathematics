% !TeX root = ../../main.tex
\documentclass[class=article, crop=false]{standalone}
\begin{document}
In this section we will prove the next homotopy equivalence along the chain in \eqref{eqn:proof_overview}. This will again involve the use of posets and their combinatorics.
Morse theory for smooth manifolds gives a way to infer topological properties of manifolds from analytical properties of certain smooth functions on that manifold. Discrete Morse theory is a CW (non--smooth) analogue. Certain functions on the (discrete) set of cells of a CW complex can tell us topological facts about the CW complex. Here we will only give a brief introduction to the main results of this theory that are relevant to us.

\end{document}