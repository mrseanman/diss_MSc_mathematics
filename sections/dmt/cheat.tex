% !TeX root = ../../main.tex
\documentclass[class=article, crop=false]{standalone}
\begin{document}

\section{The final step}
The remaining homotopy equivalence in \eqref{eqn:proof_overview} to be shown is $K^\prime_W \simeq X^\prime_W$. This is completed using discrete Morse theory in Section 8 of \cite{paolini_salvetti_kpi1_2021}. We will not review this part of the proof in detail but instead provide a brief overview. In place of this detail we will suggest future areas of study that could provide an alternative, more general proof of $K^\prime_W \simeq X^\prime_W$.

At this stage, it is useful to remark that the only part of this work so far that has been specific to \emph{affine} Coxeter groups has been \cref{lem:intersection_fibre_components_X_prime}. Everything else generalises to all Coxeter groups. However, the proof of $K_W^\prime \simeq X^\prime_W$ in \cite{paolini_salvetti_kpi1_2021} relies heavily on the geometry of affine Coxeter groups.

The \emph{rank} of a Coxeter group $W$ is the maximum $n$ such that $\mathcal{S}^n_W \neq \emptyset$. Given a realisation of $W$ as a reflection group on $\R^n$, where $n$ is the rank of $W$, the \emph{Coxeter axis} $\ell$ corresponding to a Coxeter element $w$ is an affine subset of $\R^n$ associated to $w$. As the name suggests, the Coxeter axis is always a 1--dimensional affine subspace of $\R^n$.   

\end{document}