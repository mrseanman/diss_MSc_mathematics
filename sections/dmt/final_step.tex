% !TeX root = ../../main.tex
\documentclass[class=article, crop=false]{standalone}
\begin{document}

\section{The final step}
The remaining homotopy equivalence to be shown in \eqref{eqn:proof_overview} is $K^\prime_W \simeq X^\prime_W$. This is completed using discrete Morse theory in \cite[Section 8]{paolini_salvetti_kpi1_2021}. We will not review this part of the proof in detail but instead provide a very brief overview.
% In place of this detail we will suggest future areas of study that could provide an alternative, more general proof of $K^\prime_W \simeq X^\prime_W$.

At this stage, it is useful to remark that the only part of this work so far that has been specific to \emph{affine} Coxeter groups has been \cref{lem:intersection_fibre_components_X_prime}. Everything else generalises to all Coxeter groups. However, the proof of $K_W^\prime \simeq X^\prime_W$ in \cite{paolini_salvetti_kpi1_2021} relies heavily on the geometry of affine Coxeter groups, as we will see.

The \emph{rank} of a Coxeter group $W$ is the maximum $n$ such that $\mathcal{S}^n_W \neq \emptyset$. Given a realisation of an affine $W$ as a reflection group on $\R^n$, where $n$ is the rank of $W$, the \emph{Coxeter axis} $\ell$ corresponding to a Coxeter element $w$ is an affine subset of $\R^n$ associated to $w$. As the name suggests, the Coxeter axis is always a 1--dimensional affine subspace of $\R^n$. There is also an orientation on $\ell$. See \cite[Section 4]{paolini_salvetti_kpi1_2021} and \cite[Section 7]{mccammond_dual_2015}.

Given a Coxeter element $w$ and corresponding interval $[1,w]^W$, we recall $R_0 \coloneq R \cap [1,w]^W$. There exists a total ordering on $R_0$ using $\ell$ and its orientation \cite[Definition 4.10]{paolini_salvetti_kpi1_2021}. Using properties of this ordering and the linear structure of fibre components, the authors define a function 

\begin{equation*}
    \mu \colon \mathcal{F}(K_W^\prime)\setminus\mathcal{F}(X_W^\prime) \to \mathcal{F}(K_W^\prime)\setminus\mathcal{F}(X_W^\prime) .
\end{equation*}

It is shown that $\mu$ is an involution, i.e.~$\mu(\mu(\sigma)) = \mu(\sigma)$ and thus the authors define the matching

\begin{equation*}
    \mathcal{M} \coloneq \Set{(\mu(\sigma), \sigma) \given \sigma \in \mathcal{F}(K_W^\prime)\setminus\mathcal{F}(X_W^\prime) \text{ and } \mu(\sigma) \lessdot \sigma}.
\end{equation*}

This matching is shown to be acyclic, proper and with critical cells exactly corresponding to $X_W^\prime$. Thus, the proof concludes using \cref{thm:dmt}.

\begin{theorem}[{\hspace{1sp}\cite[Theorem 8.14]{paolini_salvetti_kpi1_2021}}]
    Given an affine Coxeter group $W$, the space $K^\prime_W$ deformation retracts on to $X^\prime_W$.
    \label{thm:K_prime_hom_equiv_X_prime}
\end{theorem}
\end{document}