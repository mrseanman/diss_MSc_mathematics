\documentclass[class=article, crop=false]{standalone}
\begin{document}

\subsection{Poset Complex}

For some edge labelled poset $P$, we can construct a cell complex $K(P)$ from $P$ such that $\pi_1(K(P))$ is $G(P)$. We do this by initially defining a simplicial complex $\Delta(P)$. An \emph{abstract simplicial complex} is a family of sets that is closed under taking arbitrary subsets. From this, we can define

\begin{definition}[Geometric Simplicial Complex]
	Given an abstract simplicial complex $X$, the \emph{geometric realisation} of that simplicial complex is defined as follows: For each single element set in $X$ assign a point, for each three element set assign an edge attaching the two vertices it contains, for each two element set assign a triangle, comprising the three edges of its three subsets containing two elements. In this way continue constructing simplices of dimension $n$ for each $n+1$ size set in $X$.
	\label{def:geometric_simplicial_cx}
\end{definition}

The set of all chains in a poset $P$ is an abstract simplicial complex. We define $\Delta(P)$ to be the geometric simplicial complex corresponding to the set of all chains in $P$ where each $n$--simplex is an $n$--chain of $P$. Note that as in \cite[Definition 1.7]{mccammond_sulway_artin_2017}, we define an $n$--chain to have $n-1$ elements. E.g.~$(\{1\} \subseteq \{1,2\})$ is a 1--chain.

For example, in \cref{fig:example_edge_labelled_poset}, $\Delta(P)$ would be three solid tetrahedrons all sharing an edge (a 1--simplex) corresponding to the 1--chain $(\emptyset \subseteq \{1,2,3,4\})$ with and two of them sharing a face corresponding to the 2--chain  $(\emptyset \subseteq \Set{1} \subseteq \Set{1,2,3,4})$. For a more two dimensional example consider the following poset $P$ and corresponding $\Delta(P)$. Here we forget about edge labelling in $P$ for a moment.

\begin{figure}[h]
\centering
\begin{tikzpicture}[scale=1.5]
	\tikzstyle{every label}=[font=\footnotesize]
	\node[FSC] (base left)    	at (0,0)   				[label=below:{$\{2\}$}]				{};
	\node[FSC] (base right)		at ($(base left) + (1,0)$)  		[label=below:{$\{1\}$}]   			{};							
	\node[FSC] (middle left)	at ($(base left) + (-0.5, 0.866)$)	[label=left:{$\{2,3\}$}]   			{};
	\node[FSC] (middle middle)	at ($(middle left) + (1,0)$)		[label={[label distance=-4]40:{$\{1,2\}$}}]	{};
	\node[FSC] (middle right)	at ($(middle middle) + (1,0)$)		[label=right:{$\{1,4\}$}]			{};
	\node[FSC] (top)		at ($(middle middle) + (0, 0.866)$)	[label=above:{$\{1,2,3,4\}$}]			{};
												
	\draw	(base left)		to		(middle left);
	\draw	(base left)		to 		(middle middle);
	\draw	(base right)		to 		(middle middle);
	\draw 	(base right)		to 		(middle right);
	\draw	(middle left)		to 		(top);
	\draw 	(middle middle)		to 		(top);
	\draw	(middle right)		to 		(top);
\end{tikzpicture}
\hspace{1cm}
\tikzstyle{green polyfill}=[fill=green!20, draw=green!50!black, thick]
\begin{tikzpicture}[scale = 1.5]
	\tikzstyle{every label}=[font=\footnotesize]
	\node[FSC] (base left)    	at (0,0)						[label=below:{$\{2\}$}]				{};
	\node[FSC] (base right)		at ($(base left) + (2,0)$)				[label=below:{$\{1\}$}]   			{};							
	\node[FSC] (middle left)	at ($(base left) + (-0.5, 0.866)$)			[label=left:{$\{2,3\}$}]   			{};
	\node[FSC] (middle middle)	at ($0.5*(base left) + 0.5*(base right) + (0,0.5)$)	[label={[label distance=4pt]270:{$\{1,2\}$}}]	{};
	\node[FSC] (middle right)	at ($(base right) + (0.5,0.866)$)			[label=right:{$\{1,4\}$}]			{};
	\node[FSC] (top)		at ($(middle middle) + (0, 0.866)$)		[label=above:{$\{1,2,3,4\}$}]			{};
	
	\begin{pgfonlayer}{background}
		\filldraw[green polyfill] (base left.center) -- (middle left.center) -- (top.center) -- cycle;
		\filldraw[green polyfill] (base left.center) -- (middle middle.center) -- (top.center) -- cycle;
		\filldraw[green polyfill] (base right.center) -- (middle middle.center) -- (top.center) -- cycle;
		\filldraw[green polyfill] (base right.center) -- (middle right.center) -- (top.center) -- cycle;		
	\end{pgfonlayer}	
\end{tikzpicture}
\caption{An example poset $P$ (left) with corresponding $\Delta(P)$ (right).}
\label{fig:example_poset_with_simplicial_cx}
\end{figure}

We continue, now using an edge labelling on $P$, to generate a quotient space $K(P)$ of $\Delta(P)$. Let us put some arbitrary edge labelling on $P$ to progress with this, shown in \cref{fig:example_edge_labelled_poset_with_KP} (left).

\begin{figure}[h]
\centering
\begin{tikzpicture}[scale=1.5]
	\tikzstyle{every label}=[font=\footnotesize]
	\node[FSC] (base left)    	at (0,0)   							[label=below:{$\{2\}$}]   	{};
	\node[FSC] (base right)		at ($(base left) + (1,0)$)  		[label=below:{$\{1\}$}]   	{};							
	\node[FSC] (middle left)	at ($(base left) + (-0.5, 0.866)$)  [label=left:{$\{2,3\}$}]   	{};
	\node[FSC] (middle middle)	at ($(middle left) + (1,0)$)		[label={[label distance=-4]40:{$\{1,2\}$}}]	{};
	\node[FSC] (middle right)	at ($(middle middle) + (1,0)$)		[label=right:{$\{1,4\}$}]	{};
	\node[FSC] (top)			at ($(middle middle) + (0, 0.866)$)	[label=above:{$\{1,2,3,4\}$}]	{};
	
	\tikzstyle{every node}=[font=\footnotesize]
	\draw[a]	(base left) 	to 	node[left] 			{a} 	(middle left);
	\draw[b]	(base left) 	to 	node[right] 		{b}		(middle middle);
	\draw[b]	(base right) 	to 	node[right] 		{b}		(middle middle);
	\draw[a] 	(base right) 	to 	node[right] 		{a}		(middle right);
	\draw[b]	(middle left)	to 	node[left] 			{b} 	(top);
	\draw[a] 	(middle middle)	to 	node[right] 		{a}		(top);
	\draw[b]	(middle right)	to 	node[right]			{b}		(top);
\end{tikzpicture}
\hspace{1.3cm}
\tikzstyle{green polyfill}=[fill=green!20, draw=green!50!black, thick]
\begin{tikzpicture}[scale = 1.5, baseline=-20pt]
	\tikzstyle{every label}=[font=\footnotesize]
	\node[FSC] (base left)    	at (0,0)   							   	{};
	\node[FSC] (base right)		at ($(base left) + (2,0)$)  		   	{};							
	\node[FSC] (middle left)	at ($(base left) + (-0.5, 0.866)$)     	{};
	\node[FSC] (middle middle)	at ($0.5*(base left) + 0.5*(base right) + (0,0.5)$)		{};
	\node[FSC] (middle right)	at ($(base right) + (0.5,0.866)$)			{};
	\node[FSC] (top)			at ($(middle middle) + (0, 0.866)$)		{};

	
	\begin{scope}[thick]
		\draw[a, arrow_me=stealth]	(base left) 	to	(middle left);
		\draw[b, arrow_me=>>s]		(base left) 	to	(middle middle);
		\draw[b, arrow_me=>>s]		(base right) 	to	(middle middle);
		\draw[a, arrow_me=stealth] 	(base right) 	to	(middle right);
		\draw[b, arrow_me=>>s]		(middle left)	to	(top);
		\draw[a, arrow_me=stealth] 	(middle middle)	to	(top);
		\draw[b, arrow_me=>>s]		(middle right)	to	(top);
		
		\draw (base left) to (top);
		\draw (base right) to (top); 
	\end{scope}	
	
	\node at ($1/3*(base left) + 1/3*(middle left) + 1/3*(top)$) {$\circlearrowright$};
	\node at ($1/3*(base left) + 1/3*(middle middle) + 1/3*(top)$) {$\circlearrowright$};
	\node at ($1/3*(base right) + 1/3*(middle middle) + 1/3*(top)$) {$\circlearrowleft$};
	\node at ($1/3*(base right) + 1/3*(middle right) + 1/3*(top)$) {$\circlearrowleft$};
	
	\begin{pgfonlayer}{background}
		\fill[fill=purple!20] (base left.center) -- (middle left.center) -- (top.center) -- cycle;
		\fill[fill=orange!20] (base left.center) -- (middle middle.center) -- (top.center) -- cycle;
		\fill[fill=orange!20] (base right.center) -- (middle middle.center) -- (top.center) -- cycle;
		\fill[fill=purple!20] (base right.center) -- (middle right.center) -- (top.center) -- cycle;
	\end{pgfonlayer}	
\end{tikzpicture}
\caption{The poset in \cref{fig:example_poset_with_simplicial_cx} with edge labelling (left) and the corresponding space $K(P)$ (right).}
\label{fig:example_edge_labelled_poset_with_KP}
\end{figure}

To construct $K(P)$, first we define a labelling on chains in $P$ which extends from the edge labelling in $P$.
\begin{definition}[Extended Labelling]
	 Given some edge--labelled poset $(P,\leq,l \colon \mathcal{E}(P) \to A)$ and some chain $C \subseteq P$, the \emph{extended label} $\mathcal{L}(\rho) \subseteq A^*$ is the language of all words corresponding to all saturated chains that contain every element of $C$.
\end{definition}

Here a \emph{saturated chain} is a chain such that every relation is a covering relation. For an example on extended labels, consider the chain $(\{2\} \subseteq \{1,2,3\})$ in the context of \cref{fig:example_edge_labelled_poset_with_KP}. There are two corresponding saturated chains, $(\{2\} \subseteq \{1,2\} \subseteq \{1,2,3\})$ and $(\{2\} \subseteq \{2,3\} \subseteq \{1,2,3\})$, which respectively correspond to the words $ba$ and $ab$. So $\mathcal{L}(\{2\} \subseteq \{1,2,3\}) = \{ba, ab\}$. Here are some illustrative examples:

\begin{itemize}
	\item $\mathcal{L}(\{1\} \subseteq \{1,2\}) = \mathcal{L}(\{2\} \subseteq \{1,2\}) = \{b\}$.
	\item $\mathcal{L}(\{1\} \subseteq \{1,2,3,4\}) = \mathcal{L}(\{2\} \subseteq \{1,2,3,4\}) = \{ba, ab\}$.
	\item $\mathcal{L}(\{1\} \subseteq \{1,2\} \subseteq \{1,2,3,4\}) = \mathcal{L}(\{2\} \subseteq \{1,2\} \subseteq \{1,2,3,4\}) = \{ba\}$.
	\item $\mathcal{L}(\{1\}) = \mathcal{L}(\{2\}) = \mathcal{L}(\{1,2\}) = \cdots = \emptyset$.
\end{itemize}

This extended labelling on chains naturally extends to a labelling on simplices in $\Delta(P)$. Using this labelling and the orientation induced on a chain by $\leq$, we can define $K(P)$.

\begin{definition}[{Poset Complex \cite[Definition 1.6]{mccammond_introduction_to_garside}}]
	For an edge--labelled poset $P$ the poset complex $K(P)$ is the quotient space $\Delta(P)/\sim$ where $\sim$ identifies simplices that share the same extended label pointwise, using the orientation on simplices induced by $\leq$.
\end{definition}

In the example in \cref{fig:example_edge_labelled_poset_with_KP}, three red edges are identiefied, four blue edges are identified, two orange triangles are identified and two purple triangles are identified. Note that the two black edges are in this case identified, but only because they belong to the two pairs of identified triangles. In the second example for $\mathcal{L}$ above, we see they have different labels.

We see that this space is homeomorphic to a torus, which has $\pi(1) \cong \mathbb{Z}^2 \cong \left\langle a,b \mid ab = ba \right\rangle$, which is also the $G(P)$ for this edge--labelled poset.

This is true in general. We can determine $\pi_1(K(P))$ from its 2--skeleton \cite[Corollary 4.12]{hatcher_algebraic_nodate}. The 1--skeleton will be a wedge of circles, one for each unlabelled 1--chain in $P$ and one for each $a \in \image(l)$. Only labelled edges will contribute generators to $\pi_1(K(P))$ since a labelled path can always be deformed to an unlabelled 1--chain through the simplex in $K(P)$. If two $n$--chains start and end at the same points, they will share an edge in an $n$--simplex corresponding to an unlabelled 1--chain. So one of the paths can be deformed to the path corresponding to the edge of the unlabelled 1--chain, and then through that shared edge to the other path, making the paths homotopic. E.g.~in \cref{fig:example_edge_labelled_poset_with_KP} we can deform $(\Set{2} \subseteq \Set{1,2} \subseteq \Set{1,2,3})$ through $(\Set{2} \subseteq \Set{1,2,3})$ to $(\Set{2} \subseteq \Set{2,3} \subseteq \Set{1,2,3})$. Identification of $n$--simplices for $n>1$ does not affect the fundamental group, but does ensure that that higher homotopy groups are trivial. We can see if we did not identify the 2--simplices in \cref{fig:example_edge_labelled_poset_with_KP}, $\pi_2(K(P))$ would be non-trivial.

\end{document}
