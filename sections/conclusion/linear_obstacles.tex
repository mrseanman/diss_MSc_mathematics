% !TeX root = ../../main.tex
\documentclass[class=article, crop=false]{standalone}
\begin{document}

\section{Linear obstacles to matchings}

Consider an element of $\sigma \in \mathcal{F}(K_W^\prime) \setminus \mathcal{F}(X^\prime_W)$ in the context of a fibre component $f$. We have that $\sigma$ is necessarily in a section between two elements, $\beta$ and $\psi$ both in $\mathcal{F}(X^\prime_W)$. As in the proof of \cref{lem:K_prime_valid_subcomplex} we can assume $\beta$ and $\psi$ to be in the bottom row of the Hasse diagram for $f$. We see in \cref{fig:no_matching} that there is no matching on the fibre component $f$ that corresponds to the contraction of $K_W^\prime$ on to $X_W^\prime$.

\tikzmath{
	\hsep = 0.5;
	\vsep = 0.9;
}
\tikzstyle{m}=[green!70!black, arrow_me=stealth]
\tikzstyle{not_m}=[arrow_me=stealth]
\tikzstyle{FSC_highlight}=[circle, draw=black!50,fill=red!40,thick, inner sep=0pt,minimum size=1.5mm]
\begin{figure}[ht]
    \centering
    \vspace{0.6cm}
    \begin{tikzpicture}[scale=1.2, baseline=0.5cm]
        \tikzstyle{every label}=[font=\footnotesize]
        \tikzstyle{every node}=[font=\footnotesize]

        \node       (a1) at (0,0)                           {};
        \node[FSC_highlight]  (a2) at ($(a1) + (\hsep, \vsep)$)       {};
        \node[FSC_highlight]  (a3) at ($(a2) + (\hsep, -\vsep)$)      [label=below:{$\smallvertin{\beta}{X^\prime_W}$}]{};
        \node[FSC]  (a4) at ($(a3) + (\hsep, \vsep)$)       {};
        \node[FSC]  (a5) at ($(a4) + (\hsep, -\vsep)$)      {};
        \node[FSC]  (a6) at ($(a5) + (\hsep, \vsep)$)       {};
        \node[FSC]  (a7) at ($(a6) + (\hsep, -\vsep)$)      {};
        \node  (a8) at ($(a7) + (\hsep, \vsep)$)       {};


        \node (b0)  at ($(a8) + (2.5,0)$) {};
        \node[FSC]  (b1) at ($(b0) + (\hsep,-\vsep)$)                {};
        \node[FSC]  (b2) at ($(b1) + (\hsep, \vsep)$)       {};
        \node[FSC]  (b3) at ($(b2) + (\hsep, -\vsep)$)   {};
        \node[FSC]  (b4) at ($(b3) + (\hsep, \vsep)$)       {};
        \node[FSC_highlight]  (b5) at ($(b4) + (\hsep, -\vsep)$)      [label=below:{$\smallvertin{\psi}{X^\prime_W}$}] {};
        \node[FSC_highlight]  (b6) at ($(b5) + (\hsep, \vsep)$)          {};
        \node  (b7) at ($(b6) + (\hsep, -\vsep)$)      {};


        % see https://tex.stackexchange.com/questions/646915/how-add-1-in-math-expression-inside-tikz-foreach-loop
        \foreach \x in {5,7}
        \draw[m] (a\x) to (a\the\numexpr \x - 1\relax);

        \foreach \x in {3,5}
        \draw[not_m] (a\the\numexpr \x +1\relax) to (a\x);

        \foreach \x in {3}
        \draw[not_m] (a\x) to (a\the\numexpr \x -1\relax);

        \foreach \x in {1,3}
        \draw[m] (b\x) to (b\the\numexpr \x + 1\relax);

        \foreach \x in {3,5}
        \draw[not_m] (b\the\numexpr \x -1\relax) to (b\x);

        \foreach \x in {5}
        \draw[not_m] (b\x) to (b\the\numexpr \x +1\relax);

        \draw[dotted, thick] (a1) to (a2);
        \draw[dotted, thick] (b6) to (b7);
        \draw[dotted, thick] (a7) to (a8);
        \draw[dotted, thick] (b0) to (b1);

        \node[text width=2cm] (q) at ($0.5*(a8) + 0.5*(b0) + (0,-0.5)$) {Matching is not possible};
    \end{tikzpicture}
    \caption{C.f.~\cref{fig:acyclic_matching_fibre_components}. The central section between $\beta$ and $\psi$ is contained within $\mathcal{F}(K_W^\prime) \setminus \mathcal{F}(X^\prime_W)$. We see there is no way to form a matching on this fibre component such that all the critical cells are in the complement of this central section.}
    \label{fig:no_matching}
\end{figure}

We say that the matching on the section just to the right of $\beta$ in \cref{fig:no_matching} is \emph{oriented towards $\beta$}. Similarly, the matching on the section just to the left of $\psi$ is oriented towards $\psi$. In this terminology, to form the necessary matching on $f$, we need a way to \emph{change orientation} on a matching, as in \cref{fig:example_orientation_switch}.

\tikzmath{
	\hsep = 1.0;
	\vsep = 0.8;
}
\tikzstyle{m}=[green!70!black, arrow_me=stealth]
\tikzstyle{not_m}=[arrow_me=stealth]
\tikzstyle{FSC_highlight}=[circle, draw=black!50,fill=red!40,thick, inner sep=0pt,minimum size=1.5mm]
\begin{figure}[ht]
    \centering
    \vspace{0.3cm}
    \begin{tikzpicture}[scale=1.2, baseline=0.5cm]
        \node (d0) at (0,0) {};
        \node (d6) at ($ (d0) + 6*(\hsep,0)$) {};
        \foreach \x in {1,...,5}
        \node[FSC] (d\x) at ($ (d0) + \x*(\hsep,0)$) {};


        \node[FSC] (u0) at ($(d0) + (0,\vsep) + 0.5*(\hsep,0)$) {};
        \foreach \x in {1,...,5}
        \node[FSC] (u\x) at ($ (u0) + \x*(\hsep,0)$) {};

        \node[FSC] (cap) at ($0.5*(u2) + 0.5*(u3) + (0,\vsep)$) {};

        \draw[dotted, thick] (d0) to (u0);
        \draw[dotted, thick] (d6) to (u5);

        \foreach \x in {1,2,3}
        \draw[m] (d\x) to (u\the\numexpr \x -1\relax);

        \foreach \x in {4,5}
        \draw[m] (d\x) to (u\x);

        \foreach \x in {1,2,3}
        \draw[not_m] (u\x) to (d\x);

        \foreach \x in {3,4}
        \draw[not_m] (u\x) to (d\the\numexpr \x +1\relax);

        \draw[m] (u3) to (cap);
        \draw[not_m] (cap) to (u2);

    \end{tikzpicture}
    \caption{Certain cells, if added to the fibre component, allow us to change orientation in a matching. Note that this is only demonstrative and no such fibre component exists.}
    \label{fig:example_orientation_switch}
\end{figure}

The simple form of the fibre components in \cref{chap:dmt} makes them easy to deal with, but we see that they lack enough structure to allow for the orientation change that we have shown is necessary. In the following section we propose a different method of proof involving a conceptually simple change to the construction in \cref{sec:subcomplex_K_prime}, that changes the poset map $\mu \colon \mathcal{F}(K_W) \to \N$.

\end{document}