% !TeX root = ../../main.tex
\documentclass[class=article, crop=false]{standalone}
\begin{document}
\section{Fibre doubling and junctions}
In this section we will explore a potential method of proving $K^\prime \simeq X^\prime_W$ different to that in \cite[Section 8]{paolini_salvetti_kpi1_2021}. This will involve utilizing fibre components and the patchwork theorem, as in \cref{sec:subcomplex_K_prime}, but altering the poset map used to define the fibres. We will not prove this 


Let $h \colon \N \to \N$ be the poset map such that $h(n) = 2\cdot \lfloor n/2 \rfloor$ and recall $\mu$ from \eqref{eqn:mu_poset_map}. Consider the following map $\nu$.
\begin{equation*}
    \nu \, \colon \!\!\begin{tikzcd}
        \mathcal{F}(K_W) \ar[r, "\mu"] & \N \ar[r, "h"] &\N
    \end{tikzcd}
\end{equation*}
We have that $\nu$ is also a poset map.
If $d$ is odd then $\nu^{-1}(d) = \emptyset$. If $d$ is even then $\nu^{-1}(d) = \mu^{-1}(d) \cup \mu^{-1}(d+1)$. We call the connected (non--empty) components of such a $\nu^{-1}(d)$ \emph{double fibre components}. Let $G$ denote the set of double fibre components and $F$ denote the set of fibre components as in \cref{sec:subcomplex_K_prime}. To each $g \in G$ we associate a subset $D(g) \subseteq F$ that is the set of fibre components in $F$ that comprise the double fibre component $g$.

Given some fibre component $f \in F$. Define $\Sigma(f)$ to be a sequence $(\sigma_i)_{i \in \Z}$ of elements in $\mathcal{F}(K_W)$ such that $\bigcup_{i \in \Z} \sigma_i = f$, and such that $(\sigma_i)_{i \in \Z}$ respects the order in $f$. This sequence is unique up to translation of indices.

\begin{definition}
    Given some $g$ a connected component of $\nu^{-1}(d)$, let $f,f^\prime\in D(g)$ be two distinct fibre components comprising $g$ such that $f$ is a $d$--fibre component and $f^\prime$ is a $(d+1)$--fibre component. Let $\Sigma(f) = (\sigma_i)_{i \in \Z}$ and $\Sigma(f^\prime) = (\sigma^\prime_i)_{i \in \Z}$.
    Suppose there is a consecutive sequence of integers $(n, n+1, \ldots , n+j)$ and some constant of translation $m \in \Z$ such that $\sigma_i \lessdot \sigma^\prime_{i+m}$ for all $i \in (n, n+1, \ldots, n+j)$. We say that 
    \begin{equation*}
        \bigcup_{n\leq i \leq n+j}\sigma_i \qquad \cup \qquad \bigcup_{n+m \leq i \leq n+j+m} \sigma^\prime_i
    \end{equation*}
    is the \emph{junction} between $f$ and $f^\prime$ of length $j$, starting at $(\sigma_n,\sigma^\prime_{n+m})$.
\end{definition}

Note that by our construction, given such $g$, $f$ and $f^\prime$, a junction of some length necessarily exists. See \cref{fig:junction_example} for an example picture of a junction.

A useful property of junctions is that they enable us to change orientation in a matching, see \cref{fig:junction_matching_switch}. However, we see that this change of direction is necessarily paired with the other fibre component.
\tikzmath{
	\hsep = 1.4;
	\vsep = 1.0;
}
\contournumber{60}
\begin{figure}[ht]
    \centering
    \begin{tikzpicture}
        \tikzstyle{every label}=[font=\scriptsize]
        \tikzstyle{every node}=[font=\footnotesize]
        \node (x0x1) at (0,0) {$[x_0 | x_1]$};
        \node (x1) at ($(x0x1) + (\hsep,-\vsep)$) {$[x_1]$};
        \node (x0) at ($(x0x1) + (-\hsep,-\vsep)$) {$[x_0]$};
        \node (x-1x0) at ($(x0x1) + 2*(-\hsep,0)$) {$[x_{-1}|x_0]$};
        \node (x1x2) at ($(x0x1) + 2*(\hsep,0)$) {$[x_1|x_2]$};
        \node (x2) at ($(x1x2) + (\hsep,-\vsep)$) {$[x_2]$};


        \node (x0y1y2) at ($(x0x1) + 2.3*(0,\vsep)$) {$[y_0|y_1|y_2]$};
        \node (x0y1) at ($(x0y1y2) + (-\hsep,-\vsep)$) {$[y_0|y_1]$};
        \node (y1y2) at ($(x0y1y2) + (\hsep,-\vsep)$) {$[y_1|y_2]$};
        \node (y-1x0y1) at ($(x0y1y2) + (-2*\hsep,0)$) {$[y_{-1}|y_0|y_1]$};
        \node (y1y2y3) at ($(x0y1y2) + (2*\hsep,0)$) {$[y_1|y_2|y_3]$};
        \node (y2y3) at ($(y1y2y3) + (\hsep,-\vsep)$) {$[y_2|y_3]$};

        \node (ul) at ($ (y-1x0y1) + (-\hsep,-\vsep)$) {};
        \node (ur) at ($ (y2y3) + (\hsep,\vsep)$) {};

        \node (dl) at ($ (x-1x0) + (-\hsep,-\vsep)$) {};
        \node (dr) at ($ (x2) + (\hsep,\vsep)$) {};

        
        \draw (x-1x0) to (x0) to (x0x1) to (x1) to (x1x2) to (x2);
        \draw (y-1x0y1) to (x0y1) to (x0y1y2) to (y1y2) to (y1y2y3) to (y2y3);
        
        \draw (x0y1) to (x0);
        \draw (x0y1y2) to (x0x1);
        \draw (y1y2) to (x1);
        \draw (y1y2y3) to (x1x2);

        \draw[dotted, thick] (y-1x0y1) to (ul);
        \draw[dotted, thick] (y2y3) to (ur);
        \draw[dotted, thick] (x-1x0) to (dl);
        \draw[dotted, thick] (x2) to (dr);
            
    \end{tikzpicture}
    \caption{A length 4 junction starting at $([x_0],[y_0|y_1])$. This corresponds to a connected component of $\nu^{-1}(2)$. Here $f$ is the $2$--fibre component with sequence $(x_i)_{i \in \Z}$ and $f^\prime$ is the $3$--fibre component with sequence $(y_i)_{i \in \Z}$ such that $y_1y_2 = x_1$. This necessitates $y_0=x_0$ and $y_3=x_2$.}
    \label{fig:junction_example}
\end{figure}

\tikzmath{
	\hsep = 0.7;
	\vsep = 0.5;
}
\tikzstyle{m}=[green!70!black, arrow_me=stealth]
\tikzstyle{not_m}=[arrow_me=stealth]
\begin{figure}
    \centering
    \begin{tikzpicture}[scale=1.2, baseline=0.5cm]
        \node (dL) at (0,0) {};
        \foreach \x in {2,4,...,8}
        \node[FSC] (d\x) at ($(\x*\hsep, 0)$) {};
        \node (dR) at ($(d8) + (2*\hsep,0)$) {};
        \foreach \x in {1,3,...,9}
        \node[FSC] (d\x) at ($(\x*\hsep, \vsep)$) {};


        \node (uL) at ($(dL) + 2*(0,\vsep)$) {};
        \foreach \x in {2,4,...,8}
        \node[FSC] (u\x) at ($(uL) + (\x*\hsep, 0)$) {};
        \node (uR) at ($(u8) + (2*\hsep,0)$) {};
        \foreach \x in {1,3,...,9}
        \node[FSC] (u\x) at ($(uL) + (\x*\hsep, \vsep)$) {};

        \draw[dotted, thick] (dL) to (d1);
        \draw[dotted, thick] (d9) to (dR);

        \draw[dotted, thick] (uL) to (u1);
        \draw[dotted, thick] (u9) to (uR);

        \foreach \x in {2,4}
        \draw[m] (d\x) to (d\the\numexpr \x -1\relax);
        \foreach \x in {2,4}
        \draw[m] (u\x) to (u\the\numexpr \x -1\relax);

        \foreach \x in {3,5}
        \draw[not_m] (d\x) to (d\the\numexpr \x -1\relax);
        \foreach \x in {3,5}
        \draw[not_m] (u\x) to (u\the\numexpr \x -1\relax);

        \foreach \x in {6,8}
        \draw[m] (d\x) to (d\the\numexpr \x +1\relax);
        \foreach \x in {6,8}
        \draw[m] (u\x) to (u\the\numexpr \x +1\relax);
        
        \foreach \x in {5,7}
        \draw[not_m] (d\x) to (d\the\numexpr \x +1\relax);
        \foreach \x in {5,7}
        \draw[not_m] (u\x) to (u\the\numexpr \x +1\relax);

        \draw[m] (d5) to (u5);
    \end{tikzpicture}
    \caption{A demonstration of how a junction can be used to simultaneously change matching orientation on a pair of fibre components. In our case, we require one of the bridges of the junction to be between the upper rows of the respective fibre components.}
    \label{fig:junction_matching_switch}
\end{figure}
There are some immediate limitations to this technique. Let $d_{\max} \coloneq \max\Set{d\in\N \given \mu^{-1}(d)\neq \emptyset}$. Since $\mu^{-1}(n) = \emptyset$ for all $n\leq 0$, we see that if $d_{\max}$ is odd then we see that $\nu^{-1}(d_{\max}) = \mu^{-1}(d_{\max})$ and this construction leads to simple, linear fibre components as in \cref{fig:fibre_component_eg}. For which the necessary matching is impossible. For a given Coxeter system $(W,S)$ and Coxeter element $w$, the value of $d_{\max}$ is $\Abs{S}$. So this construction will not be useful for Coxeter systems for which $\Abs{S}$ is odd.



\vspace{5cm}
For fibre doubling to be useful in showing $K^\prime_W \simeq X^\prime_W$, we need the following to be true.
\begin{enumerate}
    \item For each section of a fibre component in $\mathcal{F}(K_W^\prime)\setminus\mathcal{F}(X_W^\prime)$ bounded between two elements of $\mathcal{F}(X^\prime_W)$, we require that there is a similar section belonging to a different fibre component such that a junction as in \cref{fig:junction_matching_switch}.
\end{enumerate}

\end{document}
