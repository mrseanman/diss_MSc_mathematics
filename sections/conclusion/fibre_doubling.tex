% !TeX root = ../../main.tex
\documentclass[class=article, crop=false]{standalone}
\begin{document}
\section{Fibre doubling and junctions}
We will now outline this idea for a proof of $K^\prime \simeq X^\prime_W$ different to that in \cite[Section 8]{paolini_salvetti_kpi1_2021}. This will involve utilizing fibre components and the patchwork theorem, as in \cref{sec:subcomplex_K_prime}, but we will alter the poset map used to define the fibres.


Let $h \colon \N \to \N$ be the poset map such that $h(n) = 2\cdot \lfloor n/2 \rfloor$ and recall $\eta$ from \eqref{eqn:mu_poset_map}. Consider the following map $\nu$.
\begin{equation}
    \nu \, \colon \!\!\begin{tikzcd}
        \mathcal{F}(K_W) \ar[r, "\eta"] & \N \ar[r, "h"] &\N
    \end{tikzcd}
    \label{eqn:nu_poset}
\end{equation}
We have that $\nu$ is also a poset map.
If $d$ is odd then $\nu^{-1}(d) = \emptyset$. If $d$ is even then $\nu^{-1}(d) = \eta^{-1}(d) \cup \eta^{-1}(d+1)$. We call the connected (non--empty) components of $\nu^{-1}(d)$ \emph{double fibre components}. Let $G$ denote the set of double fibre components and $F$ denote the set of fibre components as in \cref{sec:subcomplex_K_prime}. To each $g \in G$ we associate a subset $D(g) \subseteq F$ that is the set of fibre components in $F$ that comprise the double fibre component $g$.

Given some fibre component $f \in F$. Define $\Sigma(f)$ to be a sequence $(\sigma_i)_{i \in \Z}$ of elements in $\mathcal{F}(K_W)$ such that $\bigcup_{i \in \Z} \sigma_i = f$, and such that $(\sigma_i)_{i \in \Z}$ respects the order in $f$. This sequence is unique up to translation of indices. Note this is different but related to the sequence of group elements defined in \cref{sec:subcomplex_K_prime}.

\begin{definition}
    Given some $g \in G$, a connected component of $\nu^{-1}(d)$, let $f,f^\prime\in D(g)$ be two distinct fibre components within $g$ such that $f$ is a $d$--fibre component and $f^\prime$ is a $(d+1)$--fibre component. Let $\Sigma(f) = (\sigma_i)_{i \in \Z}$ and $\Sigma(f^\prime) = (\sigma^\prime_i)_{i \in \Z}$.
    Suppose there is a consecutive sequence of integers $(n, n+1, \ldots , n+j)$ and some constant of translation $m \in \Z$ such that $\sigma_i \lessdot \sigma^\prime_{i+m}$ for all $i \in (n, n+1, \ldots, n+j)$. We say that 
    \begin{equation*}
        \bigcup_{n\leq i \leq n+j}\sigma_i \qquad \cup \qquad \bigcup_{n+m \leq i \leq n+j+m} \sigma^\prime_i
    \end{equation*}
    is the \emph{junction} between $f$ and $f^\prime$ of length $j$, starting at $(\sigma_n,\sigma^\prime_{n+m})$.
\end{definition}

We say that the individual $\sigma_i \lessdot \sigma^\prime_{i+m}$ corresponding to a junction are the \emph{bridges} of that junction.
See \cref{fig:junction_example} for an example picture of a junction.

A useful property of junctions is that they enable us to change orientation in a matching, see \cref{fig:junction_matching_switch}. However, we see that this change of direction is necessarily paired with the other fibre component.
\tikzmath{
	\hsep = 1.4;
	\vsep = 1.0;
}
\contournumber{60}
\begin{figure}
    \centering
    \begin{tikzpicture}
        \tikzstyle{every label}=[font=\scriptsize]
        \tikzstyle{every node}=[font=\footnotesize]
        \node (x0x1) at (0,0) {$[x_0 | x_1]$};
        \node (x1) at ($(x0x1) + (\hsep,-\vsep)$) {$[x_1]$};
        \node (x0) at ($(x0x1) + (-\hsep,-\vsep)$) {$[x_0]$};
        \node (x-1x0) at ($(x0x1) + 2*(-\hsep,0)$) {$[x_{-1}|x_0]$};
        \node (x1x2) at ($(x0x1) + 2*(\hsep,0)$) {$[x_1|x_2]$};
        \node (x2) at ($(x1x2) + (\hsep,-\vsep)$) {$[x_2]$};
        \node (x2x3) at ($(x2) + (\hsep, \vsep)$) {$[x_2|x_3]$};

        \node (x0y1y2) at ($(x0x1) + 2.3*(0,\vsep)$) {$[y_0|y_1|y_2]$};
        \node (x0y1) at ($(x0y1y2) + (-\hsep,-\vsep)$) {$[y_0|y_1]$};
        \node (y1y2) at ($(x0y1y2) + (\hsep,-\vsep)$) {$[y_1|y_2]$};
        \node (y-1x0y1) at ($(x0y1y2) + (-2*\hsep,0)$) {$[y_{-1}|y_0|y_1]$};
        \node (y1y2y3) at ($(x0y1y2) + (2*\hsep,0)$) {$[y_1|y_2|y_3]$};
        \node (y2y3) at ($(y1y2y3) + (\hsep,-\vsep)$) {$[y_2|y_3]$};
        \node (y2y3y4) at ($(y2y3) + (\hsep,\vsep)$) {$[y_2|y_3|y_4]$};

        \node (ul) at ($ (y-1x0y1) + (-\hsep,-\vsep)$) {};
        \node (ur) at ($ (y2y3y4) + (\hsep,-\vsep)$) {};

        \node (dl) at ($ (x-1x0) + (-\hsep,-\vsep)$) {};
        \node (dr) at ($ (x2x3) + (\hsep,-\vsep)$) {};

        
        \draw (x-1x0) to (x0) to (x0x1) to (x1) to (x1x2) to (x2) to (x2x3);
        \draw (y-1x0y1) to (x0y1) to (x0y1y2) to (y1y2) to (y1y2y3) to (y2y3) to (y2y3y4);
        
        \draw (x0y1) to (x0);
        \draw (x0y1y2) to (x0x1);
        \draw (y1y2) to (x1);
        \draw (y1y2y3) to (x1x2);
        \draw (x2) to (y2y3);

        \draw[dotted, thick] (y-1x0y1) to (ul);
        \draw[dotted, thick] (y2y3y4) to (ur);
        \draw[dotted, thick] (x-1x0) to (dl);
        \draw[dotted, thick] (x2x3) to (dr);
            
    \end{tikzpicture}
    \caption{A length 5 junction starting at $([x_0],[y_0|y_1])$. This corresponds to a connected component of $\nu^{-1}(2)$. Here $f$ is the $2$--fibre component with sequence $(x_i)_{i \in \Z}$ and $f^\prime$ is the $3$--fibre component with sequence $(y_i)_{i \in \Z}$ such that $y_1y_2 = x_1$. This necessitates $y_0=x_0$ and $y_3=x_2$.}
    \label{fig:junction_example}
\end{figure}

\tikzmath{
	\hsep = 0.7;
	\vsep = 0.5;
}
\tikzstyle{m}=[green!70!black, arrow_me=stealth]
\tikzstyle{not_m}=[arrow_me=stealth]
\begin{figure}
    \centering
    \begin{tikzpicture}[scale=1.2, baseline=0.5cm]
        \node (dL) at (0,0) {};
        \foreach \x in {2,4,...,8}
        \node[FSC] (d\x) at ($(\x*\hsep, 0)$) {};
        \node (dR) at ($(d8) + (2*\hsep,0)$) {};
        \foreach \x in {1,3,...,9}
        \node[FSC] (d\x) at ($(\x*\hsep, \vsep)$) {};


        \node (uL) at ($(dL) + 2*(0,\vsep)$) {};
        \foreach \x in {2,4,...,8}
        \node[FSC] (u\x) at ($(uL) + (\x*\hsep, 0)$) {};
        \node (uR) at ($(u8) + (2*\hsep,0)$) {};
        \foreach \x in {1,3,...,9}
        \node[FSC] (u\x) at ($(uL) + (\x*\hsep, \vsep)$) {};

        \draw[dotted, thick] (dL) to (d1);
        \draw[dotted, thick] (d9) to (dR);

        \draw[dotted, thick] (uL) to (u1);
        \draw[dotted, thick] (u9) to (uR);

        \foreach \x in {2,4}
        \draw[m] (d\x) to (d\the\numexpr \x -1\relax);
        \foreach \x in {2,4}
        \draw[m] (u\x) to (u\the\numexpr \x -1\relax);

        \foreach \x in {3,5}
        \draw[not_m] (d\x) to (d\the\numexpr \x -1\relax);
        \foreach \x in {3,5}
        \draw[not_m] (u\x) to (u\the\numexpr \x -1\relax);

        \foreach \x in {6,8}
        \draw[m] (d\x) to (d\the\numexpr \x +1\relax);
        \foreach \x in {6,8}
        \draw[m] (u\x) to (u\the\numexpr \x +1\relax);
        
        \foreach \x in {5,7}
        \draw[not_m] (d\x) to (d\the\numexpr \x +1\relax);
        \foreach \x in {5,7}
        \draw[not_m] (u\x) to (u\the\numexpr \x +1\relax);

        \draw[m] (d5) to (u5);
    \end{tikzpicture}
    \caption{A demonstration of how a junction can be used to simultaneously change matching orientation on a pair of fibre components. In our case, we require one of the bridges of the junction to be between the upper rows of the respective fibre components.}
    \label{fig:junction_matching_switch}
\end{figure}
There are some immediate limitations to this technique. Let $d_{\max} \coloneq \max\Set{d\in\N \given \eta^{-1}(d)\neq \emptyset}$. Recalling that $\nu^{-1}(d) \neq \emptyset$ only if $d$ is even, we see that if $d_{\max}$ is even then $\nu^{-1}(d_{\max}) = \eta^{-1}(d_{\max})$ and connected components of $\nu^{-1}(d_{\max})$ will be simple and linear $(d_{\max})$--fibre components, as in \cref{fig:fibre_component_eg}. So, the required matching would be is impossible on $\nu^{-1}(d_{\max})$

For a given Coxeter system $(W,S)$ and Coxeter element $w$, the value of $d_{\max}$ is $\Abs{S}$. So this particular construction will not be useful for Coxeter systems for which $\Abs{S}$ is even. However, given $\Abs{S}$ is odd, we always have junctions between fibre components.

\begin{lemma}[{\hspace{1sp}\cite[Lemma 5.1]{paolini_salvetti_kpi1_2021}}]
    Fix a Coxeter system $(W,S)$ with set of reflections $R \subseteq W$ and let $l_R \colon W \to \N$ be the length function on $W$ with respect to the generating suet $R$. For any Coxeter element $w \in W$, we have $l_R(w) = \Abs{S}$.
    \label{lem:constant_reflection_length_of_cox_elt}
\end{lemma}
\begin{lemma}
   Fix a Coxeter system $(W,S)$ such that $\Abs{S}\geq 3$ is odd and fix a Coxeter element $w \in W$. For some $d\geq 1$, either $\nu^{-1}(d) = \emptyset$ or $\nu^{-1}(d)$ contains a junction.
   \label{lem:junctions_exist}
\end{lemma}
\begin{proof}
    Denote the set of reflections in $W$ by $R$.
    Suppose $\nu^{-1}(d)  = \eta^{-1}(d) \cup \eta^{-1}(d +1)\neq \emptyset$. We have that $d$ is even. Bear in mind that for any $n$, $\eta^{-1}(n) \neq \emptyset \iff 1 \leq n \leq \Abs{S}$. At least one of $\eta^{-1}(d)$ or $\eta^{-1}(d+1)$ are non--empty. Given our restrictions, $\eta^{-1}(d)$ is definitely non--empty, so $d \leq \Abs{S}$. Since $d$ is even, $d \leq \Abs{S}$ and $\Abs{S}$ is odd, we have $d < \Abs{S}$.

    Now let $\sigma = [x_1 | \cdots |x_{d-1}]$ be in some $d$--fibre component such that $x_1x_2\cdots x_{d} = w$. There is a factorisation of $w$ in $R$ that respects the factors $x_1x_2\cdots x_{d} = w$. By \cref{lem:constant_reflection_length_of_cox_elt}, this factorisation contains $\Abs{S}$ factors. Since $\Abs{S} > d$, by pigeonhole principle there is at least one $x_i \in \Set{x_1, \ldots, x_{d}}$ that non--trivially factors in $R$. Suppose $x_i = r_1r_2\cdots r_k$ is this factorisation, where the $r_j$ are in $R$. We have that
    \begin{equation*}
        \tau = [x_1|\cdots |x_{i-1}|r_1|r_2r_3\cdots r_k|x_{i+1}|\cdots|x_{d}] \in \eta^{-1}(d+1)
    \end{equation*}
    and $\sigma \lessdot \tau$.
\end{proof}

\begin{remark}
    The construction of this proof is doing the construction in the proof of \cref{lem:K_prime_valid_subcomplex} with $s^i$, but in the inverse order (i.e.~the inverse of the map $s \mapsto s^i$ in the proof of \cref{lem:K_prime_valid_subcomplex}). The above construction results in a junction of length at least $2d +1$. See \cref{fig:junction_example} for an example of such a junction.
    \label{rmk:junction_length}
\end{remark}

At this point it is worth noting why we have chosen to define $\nu$ as in \eqref{eqn:nu_poset}. Suppose in \eqref{eqn:nu_poset} we had instead chosen $h$ to be $h(n) = 2\lceil n/2 \rceil$ instead of  $h(n) = 2\lfloor n/2 \rfloor$. The non--empty fibres of $\nu$ would instead look like $\eta^{-1}(d) \cup \eta^{-1}(d-1)$ and this construction could potentially work for $(W,S)$ where $\Abs{S}$ is even. However, consider now $\eta^{-1}(1)$. This comprises two elements $[\,\,]$ and $[w]$. If we chose $h(n) = 2\lceil n/2 \rceil$ then we would have $\nu^{-1}(2) = \eta^{-1}(2) \cup \eta^{-1}(1)$. There would be no way to use the junctions between $[w] \in \eta^{-1}(1)$ and all the 2--fibre components to construct a useful matching. The relevant picture would be to imagine \cref{fig:junction_matching_switch} but with only one vertex in the bottom row, to be shared between all 2--fibre components. Hence, why we made our specific choice for $h$ in defining $\nu$.

Also note that $\nu^{-1}(0) = \eta^{-1}(1)$ is missed out by \cref{lem:junctions_exist}. This poses no issues since $\eta^{-1}(1)$ is always a subset of $X^\prime_W$ by our previous remarks and so no matching construction is necessary on this fibre. 

At this point we should address how this construction could avoid using \cref{lem:intersection_fibre_components_X_prime}, which is only proven for affine Coxeter groups. Suppose we had some infinite fibre component $f$ such that $f \cap X^\prime_W = \emptyset$. It may seem possible that we could form a matching as in \cref{fig:improper_infinite_matching} which has no critical cells. However, the number of cells reachable by a directed path from any of the cells in this matching is infinite. So the matching is improper and is not useful for our construction.

\tikzmath{
	\hsep = 1.0;
	\vsep = 0.8;
}
\tikzstyle{m}=[green!70!black, arrow_me=stealth]
\tikzstyle{not_m}=[arrow_me=stealth]
\tikzstyle{FSC_highlight}=[circle, draw=black!50,fill=red!40,thick, inner sep=0pt,minimum size=1.5mm]
\begin{figure}[ht]
    \centering
    \vspace{0.3cm}
    \begin{tikzpicture}[scale=1.2, baseline=0.5cm]
        \node (d0) at (0,0) {};
        \node (d6) at ($ (d0) + 6*(\hsep,0)$) {};
        \foreach \x in {1,...,5}
        \node[FSC] (d\x) at ($ (d0) + \x*(\hsep,0)$) {};


        \node[FSC] (u0) at ($(d0) + (0,\vsep) + 0.5*(\hsep,0)$) {};
        \foreach \x in {1,...,5}
        \node[FSC] (u\x) at ($ (u0) + \x*(\hsep,0)$) {};

        \draw[dotted, thick] (d0) to (u0);
        \draw[dotted, thick, green!70!black] (d6) to (u5);

        \foreach \x in {1,2,3,4,5}
        \draw[m] (d\x) to (u\the\numexpr \x -1\relax);


        \foreach \x in {1,2,3,4,5}
        \draw[not_m] (u\x) to (d\x);


    \end{tikzpicture}
    \caption{A matching on an infinite component with no critical cells. However, this matching is not proper.}
    \label{fig:improper_infinite_matching}
\end{figure}

However, consider the case where there is a junction $j$ between $f$ and another fibre component such that a bridge of $j$ attaches to the bottom row of $f$, then we could construct a matching as in \cref{fig:proper_infinite_matching}. This matching is proper on $f$ and no critical cells are in $f$.

\begin{figure}[ht]
    \centering
    \vspace{0.3cm}
    \begin{tikzpicture}[scale=1.2, baseline=0.5cm]
        \node (d0) at (0,0) {};
        \node (d6) at ($ (d0) + 6*(\hsep,0)$) {};
        \foreach \x in {1,...,5}
        \node[FSC] (d\x) at ($ (d0) + \x*(\hsep,0)$) {};

        \node[FSC] (u0) at ($(d0) + (0,\vsep) + 0.5*(\hsep,0)$) {};
        \foreach \x in {1,...,5}
        \node[FSC] (u\x) at ($ (u0) + \x*(\hsep,0)$) {};

        \node[FSC] (cap mid) at ($(d3) + 2.1*(0, \vsep)$) {};
        \node (capl) at ($(cap mid) + (-\hsep,0)$) {};
        \node (capr) at ($(cap mid) + (\hsep,0)$) {};

        \draw[dotted, thick, green!70!black] (d0) to (u0);
        \draw[dotted, thick, green!70!black] (d6) to (u5);
        \draw[dotted, thick] (capl) to (cap mid) to (capr);

        \foreach \x in {1,2}
        \draw[m] (d\x) to (u\x);

        \foreach \x in {1,2,3}
        \draw[not_m] (u\the\numexpr \x -1\relax) to (d\x);

        \foreach \x in {4,5}
        \draw[m] (d\x) to (u\the\numexpr \x -1\relax);

        \foreach \x in {3,4,5}
        \draw[not_m] (u\x) to (d\x);

        \draw[m] (d3) to (cap mid);

    \end{tikzpicture}
    \caption{A proper matching on an infinite component $f$ such that no critical cells are in $f$. This utilises a junction from the bottom row of $f$ to some other fibre component.}
    \label{fig:proper_infinite_matching}
\end{figure}

\end{document}
