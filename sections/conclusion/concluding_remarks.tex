% !TeX root = ../../main.tex
\documentclass[class=article, crop=false]{standalone}
\begin{document}
\section{Concluding remarks}

Classifying spaces are useful tools in studying the properties of groups. There are several algebraic properties of Artin groups that can be verified from an understanding of their classifying spaces, see \cite{charney_problems_2008}. For instance, proving the $K(\pi,1)$ conjecture would give us that Artin groups are torsion free by \cite[Proposition 2.45]{hatcher_algebraic_2001}. However, the difficulty in using classifying spaces to understand groups is that we move from the discrete mathematics of algebra to (typically high dimensional) topology. Sometimes this leads to a beneficial geometric intuition, but sometimes it leads to seemingly intractable problems. In either case, the power of studying groups using topology is well established.

The methods we have presented here demonstrate ways to translate problems of topology back in to problems of discrete mathematics and combinatorics, allowing us to use tools such as induction or to leverage intuition of graphs. 
The particularly prominent role of posets is intriguing. We note the many roles they play in this work. In \cref{sec:poset_cx}, we showed ways to construct spaces from posets such that the fundamental group of this space is encoded by the geometry of the poset. In \cref{sec:interval_cx} we showed that for certain Coxeter groups $W$, the isomorphism class of $W$ is encoded by the geometry of a subset of its Cayley graph that behaves well as a poset. In \cref{chap:dmt} we formed a poset using the CW--structure of a space. We then proved facts about the space using combinatoric and geometric properties of that poset.

Even ignoring the role of discrete Morse theory in this work, it is clear of the importance of posets in understanding Coxeter groups and Artin groups. This is made even clearer now that the seemingly roundabout route taken by Paolini and Salvetti, going via dual Artin group and its defining poset, has proved so fruitful. A better understanding of the geometry of interval posets $[1,w]^W$ would be a valuable tool, and it seems very possible that a similar route may be used to show the $K(\pi,1)$ conjecture in more general cases in the future.
\end{document}