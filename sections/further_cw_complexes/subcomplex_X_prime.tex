% !TeX root = ../../main.tex
\documentclass[class=article, crop=false]{standalone}
\begin{document}
\section{The subcomplex \texorpdfstring{$X_{W}^\prime$}{X W Prime}}
The definition of $K_{W}$ depends on the data of some Coxeter element $w$ and thus implicitly on some generating set $S$. For some $w = s_1s_2\cdots s_n$ and some $T\subseteq S$, define $w_T\in W$ to be the (non--consecutive) subword of $w$ consisting of elements that are in $T$, respecting the original order in $w$. 
\begin{definition}
    Define $X_{W}^\prime$ to be the finite subcomplex of $K_{W}$ consisting only of simplices $[x_1 | \cdots | x_n]$ such that $x_1x_2\cdots x_n \in [1,w_T]^W$ for some $T \in \mathcal{S}_W$. (Recalling $\mathcal{S}_W$ from \cref{def:delta_W_finite}).
    \label{def:subcomplex_X_prime}
\end{definition}
Note again the absence of $w$ from the notation of $X^\prime_W$. This will be justified in \cref{sec:adj_hom_equiv} where we will show that the homotopy type of $X_W^\prime$ has no dependence on $w$.

\begin{lemma}[{\hspace{1sp}\cite[Lemma 5.2]{paolini_salvetti_kpi1_2021}}]
    % \what[As far as I can tell, what I write here is all we need, but there they seem to go in to more detail unnecessarily]
    Fix a Coxeter group $W$ and any parabolic subgroup $W_T$, with sets of reflections $R_W$ and $R_{W_T}$ respectively. Given some Coxeter element $w \in W$ and corresponding $w_T \in W_T$, all minimal factorisations of $w_T$ by factors in $R_W$ consist only of factors in $R_{W_T}$.
    \label{lem:all_decompositions_of_w_T_are_R}
\end{lemma}
An immediate consequence of this is that the intervals $[1,w_T]^W$ and $[1,w_T]^{W_T}$ agree. This allows us to decompose $X^\prime_W$ in a useful way. For each $T \in \mathcal{S}_W$ and corresponding $W_T$, the space corresponding to the whole interval $[1, w_T]^W=[1,w_T]^{W_T}$ is a subspace inside $X^\prime_W$. This subspace is exactly $X^\prime_{W_T}$ (with respect to $w_T$). The union of all such $[1,w_T]^{W_T}$ contains all the simplices of $X_W^\prime$ by \cref{def:subcomplex_X_prime}, thus, we can think of $X^\prime_W$ as the union (with appropriate gluing) of all $X^\prime_{W_T}$ for $T \in \mathcal{S}_W$.

For $T \in \mathcal{S}_W$, each $X^\prime_{W_T}$ is exactly the same as its interval complex $K_{{W_T}}$ since all subgroups of $T$ would also generate finite Coxeter groups. Thus, using known results for finite Coxeter groups, $X^\prime_{W_T}$ is a classifying space for the dual Artin group $W_{w_T}$ by \cref{thm:interval_cx_k_pi_1_finite}. Furthermore, $W_{w_T}$ is isomorphic to the Artin group $G_{W_T}$ by \cref{thm:dual_artin_iso_artin}.

In a very similar way, the Salvetti complex consists of subspaces corresponding to elements of $\mathcal{S}_W$. For each $T \in \mathcal{S}_W$, the Salvetti complex $X_{W_T}$ is a $\Abs{T}$--cell attached to all cells corresponding to $R \subseteq T$ is the appropriate way. This is a cellular subspace of $X_W$, and since $W_T$ is finite, by \cref{thm:k_pi_1_finite}, $Y_{W_T} \simeq X_{W_T}$ is a $K(G_{W_T},1)$. The following remark summarises these observations.

\begin{remark}
    The Salvetti complex $X_W$ decomposes in to cellular subspaces $X_{W_T}$ which are $K(G_{W_T},1)$ spaces. These subspaces are in bijection with cellular subspaces $X^\prime_{W_T}$ of $X^\prime_W$, which are also $K(G_{W_T},1)$ spaces.
    \label{rmk:salvetti_X_prime_similarities}
\end{remark}

We now develop some theory to help us exploit this similarity and eventually show that $X_W \simeq X^\prime_{W}$, which is an intermediate step in \eqref{eqn:proof_overview}.

\end{document}