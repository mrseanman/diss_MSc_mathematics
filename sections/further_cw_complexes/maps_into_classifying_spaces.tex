% !TeX root = ../../main.tex
\documentclass[class=article, crop=false]{standalone}
\begin{document}

\section{Maps in to classifying spaces}

Here we will show a fundamental link between homomorphisms in to groups $G$ and maps between certain classifying spaces for $G$.

\begin{lemma}
    \label{lem:null_homotopic_extension}
    A null homotopic map $\rho \colon S^n \to X$ can be extended to a map $\sigma \colon D^{n+1} \to X$.
\end{lemma}
\begin{proof}
    Let $H \colon S^n \times I \to X$ witness the null homotopy with $H|_{S^n \times \Set{1}} \colon S^n \to \Set{x_0} \in X$. We have that $H$ factors uniquely through $(S^n \times I)/(S^n \times \Set{1}) \cong D^{n+1}$. With $\sigma$ being the unique induced map as below.

    \begin{equation*}
        \begin{tikzcd}
            &S^n \times I       \arrow[d, "q"']      \arrow[r, "H"]                                &X \\
            &(S^n \times I)/(S^n \times \Set{1})    \arrow[ru, "\exists ! \sigma", swap, dotted]
        \end{tikzcd}
    \end{equation*}

\end{proof}

\begin{theorem}[{\hspace{1sp}\cite[Proposition 1B.9]{hatcher_algebraic_2001}}]
    Let $Y$ be a $K(G,1)$ space and $X$ a finite dimensional CW--complex with only one 0--cell, the point $x_0$. Any homomorphism $\phi \colon \pi_1(X,x_0) \to \pi_1(Y,y_0)$ is induced by a map $\overline{\phi} \colon X \to Y$ where $\overline{\phi}$ is unique up to homotopy fixing $x_0$.
    \label{thm:iso_induces_unique_map_for_K_Pi_1}
\end{theorem}

\begin{proof}
    We construct the map $\overline{\phi}$ by inductively constructing it on the $n$--skeletons of $X$. We must have $\overline{\phi}(x_0) = y_0$. The 1--skeleton $X^1$ will be a wedge of circles. Let $A$ be an indexing set for the 1--cells of $X$. For each $\alpha \in A$, let each $\lambda_\alpha$ be a loop going around the closure of the 1--cell $e^1_\alpha$ in either orientation. We restrict that each $\lambda_\alpha$ be a homeomorphism witnessing $S^1 \cong e^1_\alpha \cup \Set{x_0}$. Let $B$ be an indexing set for the 2--cells of $X$. For each $\beta \in B$, let $r_\beta$ be a factorisation in the set $\Set{[\lambda_\alpha]}_{\alpha \in A} \cup \Set{[\lambda_\alpha]^{-1}}_{\alpha \in A}$ of the homotopy class of the attaching map $\psi_\beta \colon S^1 \to X^1$ for a 2--cell $e^2_\beta$. We have the following presentation for $\pi_1(X, x_0)$, as in \eqref{eqn:CW_fund_group_presentation}.
    \begin{equation}
        \pi_1(X,x_0) \cong \GroupPres{\Set{[\lambda_\alpha]}_{\alpha \in A} \relations r_\beta}
        \label{eqn:fund_group_presentation_1_skeleton}
    \end{equation}
    We can choose $\overline{\phi}(e^1_\alpha)$ to trace out any path in the homotopy class $\phi([\lambda_\alpha]) \in \pi_1(Y,y_0)$ for each $e_\alpha^1 \in X^1$.
    % \what[I disagree that this requires any specific structure on $Y$. In Hatcher, $Y$ is an arbitrary space and thus doesn't necessarily have a 1--skeleton. My previous statement here wasn't very clear and hopefully that's been amended.]
   Let $i \colon X^1 \hookrightarrow X$ be the inclusion. We have that $i_*$ is the surjection from the free group generated by $\Set{[\lambda_\alpha]}_{\alpha \in A}$ to $\pi_1(X,x_0)$. The induced map $(\overline{\phi})_*$ is determined by its action on the generators $\Set{[\lambda_\alpha]}_{\alpha \in A}$ and $(\overline{\phi})_*([\lambda_\alpha]) = \phi([\lambda_\alpha])$. Thus, our construction so far gives us the following commutative diagram.
    \begin{equation}
        \begin{tikzcd}[column sep=normal]
            \pi_1(X^1,x_0) \ar[dr, "i_*"'] \ar[rr, "(\overline{\phi})_*"] && \pi_1(Y,y_0) \\
            &\pi_1(X,x_0) \ar [ur, "\phi"']
        \end{tikzcd}
        \label{eqn:com_diagram_1_skeleton}
    \end{equation}
    
    Let $\psi_\beta \colon S^1 \to X^1$ be an attaching map for a 2--cell $e_\beta^2 \subseteq X$. We wish to show that $\overline{\phi} \circ \psi_\beta$ is null homotopic. The attaching of the 2--cell $e_\beta^2$ provides a null homotopy for the loop traced by $\psi_\beta$. In the presentation of $\pi_1(X, x_0)$ in \eqref{eqn:fund_group_presentation_1_skeleton}, each relation corresponds to the path traced out by some $\psi_\beta$. Thus, $i_*([\psi_\beta]) = 1$ and so $\overline{\phi}_*([\psi_\beta]) = \phi \circ i_*([\psi_\beta]) = 1$ by \eqref{eqn:com_diagram_1_skeleton}. Thus, $\overline{\phi} \circ \psi_\beta$ is null homotopic and so can be extended over all of the closure of $e_\beta^2$ by \cref{lem:null_homotopic_extension}. This is an extension of $\overline{\phi}$ and repeating this for each 2--cell allows us to extend $\overline{\phi}$ over all of $X^2$.
    
    To extend $\overline{\phi}$ over some $e_\gamma^3$ we use that $\pi_2(Y,y_0) = \Set{1}$. Let $\psi_\gamma \colon S^2 \to X^2$ be the attaching map for $e_\gamma^3$. Since $\pi_2(Y,y_0) = \Set{1}$, we have that $\overline{\phi} \circ \psi_\gamma \colon S^2 \to X^2 \to Y$ is nullhomotopic and so can be extended over all of the closure of $e_\gamma^3$. This same argument applies for any $e_\delta^n$ for $n\geq 3$ since $\pi_n(Y,y_0) = \Set{1}$ for all $n\geq 2$. We can thus extend $\overline{\phi}$ over the 3--cells and proceeding inductively on the $n$--skeletons, over all of $X$.

    Now we turn to the uniqueness of $\overline{\phi}$ up to homotopy. Let $\phi \colon \pi_1(X,x_0) \to \pi_1(Y,y_0)$ be some homomorphism and $\overline{\phi}_0$ and $\overline{\phi}_1$ be any maps from $X$ to $Y$ such that $(\overline{\phi}_0)_* =(\overline{\phi}_1)_* = \phi$. We have that $\overline{\phi}_0(x_0) = \overline{\phi}_1(x_0)$.
    To agree as maps, $(\overline{\phi}_0)_*$ and $(\overline{\phi}_1)_*$ must agree on generators $\Set{[\lambda_\alpha]}_{\alpha \in A}$. Thus, we have $\overline{\phi}_0 \circ \lambda_\alpha \sim \overline{\phi}_1 \circ \lambda_\alpha$ for all $\alpha \in A$. Since we restricted each $\lambda_\alpha$ to be a homeomorphism, this gives us that $\overline{\phi}_0|_{\image(\lambda_\alpha)} \sim \overline{\phi}_1|_{\image(\lambda_\alpha)}$. Let $H_\alpha$ witness this homotopy. Each $\image(\lambda_\alpha)$ is the closure of the 1--cell $e^1_\alpha$, thus $\bigcup_{\alpha \in A}\image(\lambda_\alpha) = X^1$ and the union $H \coloneq \bigcup_{\alpha \in A}H_\alpha$ witnesses the homotopy $\overline{\phi}_0|_{X^1} \sim \overline{\phi}_1|_{X^1}$.
    % \what[This is homotopy not equality, since there is a choice of loop ``\emph{in the homotopy class of $\phi([e_\alpha^1]) \in \pi_1(Y,y_0)$ for each $e_\alpha^1 \in X^1$}'']
  
    Give $X \times I$ the following CW structure. Let $X\times \Set{0}$ and $X\times \Set{1}$ both have the same cell structure as $X$ with cells denoted $d_\alpha^n$ and $e_\alpha^n$ respectively. Connect $d^0$ to $e^0$ with a 1--cell $s^1$, called the \emph{spine}. Connect a 2--cell $s_\alpha^2$ along $d_\alpha^1$, then $s^1$ then $e_\alpha^1$ then back along $s^1$ with opposite orientations on $d_\alpha^1$ and $e_\alpha^1$ such that $d^0 \cup e^0 \cup s^1 \cup d_\alpha^1 \cup e_\alpha^1 \cup s_\alpha^2 \cong S^1 \times I$. The spine now consists of $s_1 \cup s_\alpha^2$. Repeat this for each 1--cell in $X$ and then repeat for each 2--cell and so on, attaching an $s_\beta^n$ along $d_\beta^{n-1}$, $e_\beta^{n-1}$ and $s_\beta^{n-1}$, inductively building up the spine. A picture of this CW--complex completed for one $s_\alpha^2$ is below.
    \begin{equation*}
    \begin{tikzpicture}
        \tikzstyle{every label}=[font=\scriptsize]
        \tikzstyle{every node}=[font=\scriptsize]

        \node[FSC] (d_0) at (0,0)               [label={[label distance=-4pt]-60:{$d^0$}}]      {};
        \node[FSC] (e_0) at ($(d_0) + (3,0)$)   [label={[label distance=-2pt]240:{$e_0$}}]      {};

        \node            at ($(d_0) + (-0.5,1.5)$)                                              {$d_\alpha^1$};
        \node            at ($(e_0) + (0.5,1.5)$)                                               {$e_\alpha^1$};
        \node            at ($(d_0)!0.5!(e_0) + (0,1.8)$)                                       {$s_\alpha^2$};
        \draw (d_0) to node[below] {$s^1$} (e_0);

        \draw[rotate = 0, name path=d_11]   (d_0) .. controls ($(d_0) + (-1,2)$) and ($(d_0) + (1,2)$) .. (d_0);
        \draw[rotate = 60, name path=d_12]  (d_0) .. controls ($(d_0) + (-1,2)$) and ($(d_0) + (1,2)$) .. (d_0);
        \draw[rotate = 120, name path=d_12] (d_0) .. controls ($(d_0) + (-1,2)$) and ($(d_0) + (1,2)$) .. (d_0);


        \draw[rotate = 0, name path=e_11]    (e_0) .. controls ($(e_0) + (-1,2)$) and ($(e_0) + (1,2)$) .. (e_0);
        \draw[rotate = -60, name path=e_12]  (e_0) .. controls ($(e_0) + (-1,2)$) and ($(e_0) + (1,2)$) .. (e_0);
        \draw[rotate = -120, name path=e_12] (e_0) .. controls ($(e_0) + (-1,2)$) and ($(e_0) + (1,2)$) .. (e_0);


        \foreach \y in {0.1,0.4,...,1.3}{
            \path[name path=s_1] ($(d_0) + (-2,\y)$)          to ($(e_0) + (2,\y)$);
            \path[name path=s_2] ($(d_0) + (-2,\y)+ (0,0.1)$) to ($(e_0) + (2,\y) + (0,0.1)$);
            \node [coordinate, name intersections = {of = d_11 and s_1}] (s_l_1) at (intersection-1) {};
            \node [coordinate, name intersections = {of = d_11 and s_2}] (s_l_2) at (intersection-2) {};

            \node [coordinate, name intersections = {of = e_11 and s_1}] (s_r_1) at (intersection-1) {};
            \node [coordinate, name intersections = {of = e_11 and s_2}] (s_r_2) at (intersection-2) {};

            \draw[white!80!black, line width=0.1mm] (s_l_1) to (s_r_1);
            \begin{pgfonlayer}{background}
                \draw[white!80!black, line width=0.05mm, dashed] (s_l_2) to (s_r_2);
            \end{pgfonlayer}
        }
    \end{tikzpicture}
    \vspace{-0.8cm}
    \end{equation*}
    We can now extend the domain of $H$ from $X^1 \times I$ to all of $X \times I$ using this cell structure in the following way. Note that now we have two 0--cells, but this does not cause any issues. Let $H$ have domain $X^1 \times I \subseteq X \times I$. Now extend $H$ to have domain $X^1 \times I \cup X \times \Set{0} \cup X \times \Set{1}$ such that $H|_{X\times \Set{0}}$ agrees with $\overline{\phi}_0$ and $H|_{X\times \Set{1}}$ agrees with $\overline{\phi}_1$. This is possible because $H$ is a homotopy between restrictions of these maps. Note that now $H$ is defined on the whole 2--skeleton of $X \times I$. We can extend $H$ to all the higher dimensional cells by the exact same argument as before, using that $\pi_n(Y,y_0)=\Set{1}$ for $n \geq 2$. Thus, we have a continuous function $H \colon X \times I \to Y$ witnessing the homotopy $\overline{\phi}_0 \sim \overline{\phi}_1$.
\end{proof}

\begin{corollary}[{\hspace{1sp}\cite[Theorem 1B.8]{hatcher_algebraic_2001}}]
    Let $X$ and $Y$ both be $K(G,1)$ spaces. If both spaces are CW--complexes with only one 0--cell, then any isomorphism $\phi \colon \pi_1(X,x_0) \to \pi_1(Y,y_0)$ induces a homotopy equivalence witnessing $X \simeq Y$.
    \label{cor:iso_K_G_1_induces_hom_equiv}
\end{corollary}
\begin{proof}
    We have maps $\overline{\phi} \colon X \to Y$ and $\overline{(\phi^{-1})} \colon Y \to X$ with $(\overline{\phi} \circ \overline{(\phi^{-1})})_* = \id_{\pi_1(Y,y_0)}$. Thus, since the homotopy class of such maps is determined by the induced action on their fundamental groups $\overline{\phi}\circ \overline{(\phi^{-1})} \sim \id_Y$. Similarly, $\overline{(\phi^{-1})} \circ \overline{\phi} \sim \id_X$.
\end{proof}

The following is a technical lemma that uses arguments very similar to those in the proof of \cref{lem:null_homotopic_extension} to give extensions of certain homotopy equivalences between unions of $X_{W_Q}$ and unions of $X^\prime_{W_Q}$.

\begin{lemma}
    \label{lem:extendability_of_maps}
    Let $T \in \mathcal{S}_W \setminus \emptyset$. Let $\phi \colon \bigcup_{Q \subsetneq T} X_{W_Q} \to \bigcup_{Q \subsetneq T} X^\prime_{W_Q}$ be a homotopy equivalence. And $i \colon \bigcup_{Q \subsetneq T} X_{W_Q} \hookrightarrow X_{W_T}$ and $j \colon \bigcup_{Q \subsetneq T} X^\prime_{W_Q} \hookrightarrow X^\prime_{W_T}$ be inclusions.
    If:
    \begin{enumerate}
        \item $\Abs{T} \in \Set{1, 3,4,5, \ldots}$ or
        \item $\Abs{T}=2$ with $T=\Set{u,v}$ and $\phi \colon X_{W_{\Set{u}}} \cup X_{W_{\Set{v}}} \to \colon X^\prime_{W_{\Set{u}}} \cup X^\prime_{W_{\Set{v}}}$ restricts to the homotopy equivalences $\phi_{X_{W_{\Set{u}}}} \colon X_{W_{\Set{u}}}  \to X^\prime_{W_{\Set{u}}}$ and $\phi_{X_{W_{\Set{v}}}} \colon  X_{W_{\Set{v}}} \to X^\prime_{W_{\Set{v}}}$
    \end{enumerate}
    then we can extend $\phi$ to a homotopy equivalence $\psi \colon X_{W_T} \to X^\prime_{W_T}$ such that the following diagram commutes.
    \begin{equation}
        \begin{tikzcd}
\bigcup_{Q \subsetneq T} X_{W_Q}  \ar[r, "\phi"]             &   \bigcup_{Q \subsetneq T} X^\prime_{W_Q}                                                   \\
X_{W_T} \ar[r, "\psi" ] \ar[from=u, hookrightarrow, "i"']                                  &   X^\prime_{W_T}  \ar[from = u, hookrightarrow, "j"]      \\
        \end{tikzcd}
        \label{eqn:extensability_maps}
    \end{equation}
\end{lemma}
\begin{proof}
    We prove this by cases. By \cref{rmk:salvetti_X_prime_similarities}, $X_{W_T}$ and $X^\prime_{W_T}$ are classifying spaces.
    \begin{enumerate}[i)]
        \item If $\Abs{T}=1$ then any $Q\subsetneq T$ is uniquely $\emptyset$.
        % \what[I disagree that $Q$ is $\Set{\emptyset}$. $\mathcal{S}_W$ is a collection of sets, $T \in \mathcal{S}_W$ is a set, so is a subset $Q \subsetneq T$.]
        We have that $X_{W_T} \cong X^\prime_{W_T} \cong S^1$. Let $\psi$ be any map witnessing $X_{W_T} \simeq X^\prime_{W_T}$ that restricts to the map $\psi|_{X_{W_\emptyset}} \colon X_{W_\emptyset} \to X^\prime_{W_\emptyset}$.
        
        \item Suppose $\Abs{T}=2$ with $T = \Set{u,v}$. By \cref{rmk:fundamental_grp_of_salvetti}, there is an isomorphism $\sigma \colon \pi_1(X_{W_T}, X_{W_{\emptyset}}) \to G_{W_T}$ that sends the loops going around $X_{W_{\Set{u}}}$ and $X_{W_{\Set{v}}}$ to the two generators of $G_{W_T}$.
        Let $\chi\colon G_{W_T} \to W_{w_T}$ be the isomorphism from \cref{thm:dual_artin_iso_artin} that restricts to the inclusion $T \hookrightarrow W_{w_T}$.
        Furthermore, let $\tau \colon W_{w_T} \to \pi_1(X^\prime_{W_T},  X_{W_{\emptyset}})$ be the isomorphism given to us by \cref{thm:fund_group_poset_complex_poset_group}.


        Let $[\lambda_u]$ and $[\lambda_v]$ be generators of $\pi_1(X_{W_{\Set{u}}}, X_{W_\emptyset})$ and $\pi_1(X_{W_{\Set{v}}}, X_{W_\emptyset})$ respectively. By assumption, $\phi$ witnesses the homotopy equivalences $X_{W_{\Set{u}}}  \simeq X^\prime_{W_{\Set{u}}}$ and $X_{W_{\Set{v}}} \simeq X^\prime_{W_{\Set{v}}}$. Let $u_\pm \acts \pi_1(X_{W_T}, X_{W_\emptyset})$ be the automorphisms sending $[\lambda_u]$ to either $[\lambda_u]$ or $[\lambda_u]^{-1}$ and similarly define $v_\pm$. With an appropriate choice for $u_\pm$, we see that
        \begin{equation*}
            (j \circ \phi)_*([\lambda_u]) = (\tau \circ \chi \circ \sigma \circ u_\pm \circ i_*)([\lambda_u])
        \end{equation*}
        and similarly for $[\lambda_v]$. The use of $u_\pm$ and $v_\pm$ is necessary as we have no control over possible orientation flips in the homotopy equivalences $X_{W_{\Set{u}}}  \simeq X^\prime_{W_{\Set{u}}}$ and $X_{W_{\Set{v}}} \simeq X^\prime_{W_{\Set{v}}}$ witnessed by $\phi$. Together, $[\lambda_u]$ and $[\lambda_v]$ generate $\pi_1(X_{W_{\Set{u}}} \cup X_{W_{\Set{v}}}, X_{W_\emptyset})$, so we have completely specified $(j \circ \phi)_*$. Considering all of these maps together, we have the following commutative diagram.
        \vspace{0.3cm}
        \begin{equation}
            \begin{tikzcd}[column sep=small]
                \pi_1(X_{W_{\Set{u}}} \cup X_{W_{\Set{v}}}, X_{W_\emptyset}) \ar[rrr, "(j\circ \phi)_*"] \ar[d, "i_*"'] &&& \pi_1(X^\prime_{W_T}, X^\prime_{W_\emptyset}) \\
                \pi_1(X_{W_T}, X_{W_\emptyset}) \ar[r, "u_\pm \circ v_\pm"] &\pi_1(X_{W_T}, X_{W_\emptyset}) \ar[r, "\sigma"] &G_{W_T} \ar[r, "\chi"] &W_{w_T} \ar[u, "\tau"']
            \end{tikzcd}
            \label{eqn:big_com_diagram_extensability}
            \vspace{0.1cm}
        \end{equation}
        We compare this with \eqref{eqn:com_diagram_1_skeleton} and continue as in the proof of \cref{thm:iso_induces_unique_map_for_K_Pi_1}.
        Let $\beta \colon S^1 \to X_{W_{\Set{u}}} \cup X_{W_{\Set{v}}}$ be the attaching map of the unique 2--cell $e^2$ in $X_{W_T}$. This 2--cell provides a null homotopy for the loop $\beta$ in $X_{W_T}$, therefore $i_*([\beta]) = 1$. Thus, \newline$(j\circ\phi)_*([\beta])=1$ by \eqref{eqn:big_com_diagram_extensability} and so $j\circ \phi \circ \beta$ can be extended to all of the closure of $e^2$, which is all of $X_{W_T}$.
        Let $\psi\colon X_{W_T} \to X^\prime_{W_T}$ be this extension. As a map, $\psi$ satisfies the commutative diagram in \eqref{eqn:extensability_maps}. The action of the map $\psi_*$ is defined by the action of the map $(j \circ \phi)_*$. We see from \eqref{eqn:big_com_diagram_extensability} that $\psi_* =\tau \circ \chi \circ \sigma \circ t_\pm \circ s_\pm$ which is an isomorphism. Both $X_{W_T}$ and $X^\prime_{W_T}$ are CW--complexes with only one 0--cell so by \cref{cor:iso_K_G_1_induces_hom_equiv} $\psi$ is a homotopy equivalence.

        \item If $\Abs{T} \geq 3$ then we can extend $j \circ \phi$ to some map $\psi$ using the same methods as in the proof of \cref{thm:iso_induces_unique_map_for_K_Pi_1}, utilising that $\pi_n(X^\prime_{W_T}, X^\prime_{W_\emptyset}) = \Set{1}$ for all $n\geq 2$. Now we show that $\psi$ is a homotopy equivalence. In this case, $\bigcup_{Q \subsetneq T} X_{W_Q}$ contains the 2--skeleton of $X_T$ and similarly for $\bigcup_{Q \subsetneq T} X^\prime_{W_Q}$ and $X^\prime_T$. So the induced map on the inclusion $i_* \colon \pi_1(\bigcup_{Q \subsetneq T} X_{W_Q}, X_{W_\emptyset})  \to \pi_1(X_T, X_{W_\emptyset})$ is an isomorphism and similarly for $j_*$ and $X^\prime_T$ \cite[Corollary 4.12]{hatcher_algebraic_2001}. By assumption $\phi$ is a homotopy equivalence and so $\phi_*$ is an isomorphism. Therefore, $\psi_*$ is an isomorphism. Both $X_{W_T},$ and $X^\prime_{W_T}$ are CW--complexes with only one 0--cell so by \cref{cor:iso_K_G_1_induces_hom_equiv} $\psi$ is a homotopy equivalence.\qedhere
    \end{enumerate}
\end{proof}

\end{document}