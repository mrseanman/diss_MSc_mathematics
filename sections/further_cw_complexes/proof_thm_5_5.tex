% !TeX root = ../../main.tex
\documentclass[class=article, crop=false]{standalone}
\begin{document}
\section{Adjunction spaces and proving \texorpdfstring{$X_W \simeq X^\prime_W$}{X W homotopy equiv. to X prime W}}
\label{sec:adj_hom_equiv}

Here we will develop a formalism for gluing together two spaces in to what is called an \emph{adjunction space}. Adjunction spaces inherit homotopic properties from their two parent spaces in a useful way, that will help us to exploit \cref{rmk:salvetti_X_prime_similarities}.

\begin{definition}[Adjunction Space]
    For two spaces $X$ and $U$, with a continuous map $f \colon A \to U$ for some subspace $A \subseteq X$. The \emph{adjunction space} $ X \sqcup_f U$  is the space formed by gluing $X$ and $U$ via the map $f$.
    \begin{equation*}
        X \sqcup_f U \coloneq (X \sqcup Y)/(a \sim f(a))
    \end{equation*}
\end{definition}

An adjunction space is associated to the commutative diagram
\begin{equation}
    \begin{tikzcd}
            A \ar[r, "f" ] \ar[d, "i"]  &   U  \ar[d, "\bar{i}"]    \\
            X \ar[r, "\bar{f}"]         &   X \sqcup_f U            \\
    \end{tikzcd}
\label{eqn:adjunction_diagram}
\end{equation}
where $i$ is inclusion of $A$ in to $X$ and $\bar{i}$ is inclusion of $U$ in to $X \sqcup_f U $. Suppose we also have the adjunction space $Y \sqcup_g V$ with $g \colon B \to V$ and $B\subseteq Y$. Suppose further that we have maps $ \phi_1 \colon X \to Y$, $\phi_2 \colon A \to B$ and $\phi_3 \colon U \to V$ such that the following diagram commutes.
\begin{equation}
    \begin{tikzcd}[row sep=scriptsize, column sep=scriptsize]
A \ar[dr, "\phi_2"'] \ar[rr, "f"] \ar[dd, "i"']     &                                                                           &   U \ar[dr, "\phi_3"] \ar[dd, "\bar{i}", near start]  &                         \\
                                                &   B \ar[rr, crossing over, "g" near start]                                &                                                       &   V \ar[dd, "\bar{j}"]  \\
X \ar[dr, "\phi_1"'] \ar[rr, "\bar{f}"', near end]  &                                                                           &   X \sqcup_f U \ar[dr, "\phi"]                        &                         \\
                                                &   Y \ar[rr, "\bar{g}"']  \ar[from=uu, crossing over, "j"', near start]    &                                                       &   Y\sqcup_gV            \\
    \end{tikzcd}
\label{eqn:adjunction_gluing_diagram}
\end{equation}
If all $\phi_1$, $\phi_2$ and $\phi_3$ are homotopy equivalences then the following lemma tells us that $\phi$ is also a homotopy equivalence.


\begin{lemma}[{\cite[Theorem 7.5.7]{brown_topology_2006}}]
    Consider a commutative diagram as in \eqref{eqn:adjunction_gluing_diagram} where the front and back faces define an adjunction space as in \eqref{eqn:adjunction_diagram}. If $i$ and $j$ are closed cofibrations and $\phi_1$, $\phi_2$ and $\phi_3$ are homotopy equivalences, then the $\phi$ as determined by the diagram is also a homotopy equivalence.
    \label{lem:adjunction_gluing}
\end{lemma}

The restriction of $i$ and $j$ being closed cofibrations is quite mild. In the cases important to us, $i$ and $j$ will be cellular inclusions in to finite CW--complexes, and thus closed cofibrations. See \cite{brown_topology_2006} for more details on pushout squares and adjunction spaces.

To use \cref{lem:adjunction_gluing} we must be able to construct $X_W$ and $X_W^\prime$ as a sequence of adjunction spaces. Consider $X_W$ in the following example.

\begin{example}
    \label{eg:inductive_construction}
    Let $\Delta_W = \Set{\emptyset, \Set{s}, \Set{t}, \Set{u}, \Set{s,t}, \Set{s,u}, \Set{t,u}}$ and let $\Delta_W^n \coloneq \Set{T \in \Delta_W \given \Abs{T} = n }$. Clearly we have that $X_W = \bigcup_{T \in \Delta_W^2} X_{W_T}$ with the appropriate gluing. Suppose we had the 1--skeleton $X^1_W \subsetneq X_W$, and some ordering on $\Delta_W^2 = (\Set{s,t}, \Set{s,u}, \Set{t,u})$. To construct $X_W$, we would first glue $X_{\Set{s,t}}$ to $X_W^1$ as an adjunction space in the following way.
    \begin{equation*}
        \begin{tikzcd}
                X_{\Set{s}} \cup X_{\Set{t}} \ar[r, "f" ] \ar[d, "i_1"]   &   X_{\Set{s,t}}  \ar[d, "\overline{i_1}"]    \\
                X_W^1 \ar[r, "\bar{f}"]                                 &   X^1_W \sqcup_f X_{\Set{s,t}}            \\
        \end{tikzcd}
    \end{equation*}
    Where $f$ is inclusion of those 1--cells in to $X_{\Set{s,t}}$, which in this case makes $X^1_W \sqcup_f X_{\Set{s,t}} \cong X_{\Set{s,t}}$. Note that $ X_{\Set{s}} \cup X_{\Set{t}}$ is really shorthand for another adjunction space, which we assume to have been already constructed.
    We can then add $X_{\Set{t,u}}$ to the preceding adjunction space in the following way.
    \begin{equation}
        \begin{tikzcd}
                X_{\Set{t}} \cup X_{\Set{u}} \ar[r, "g" ] \ar[d, "i_2"]   &   X_{\Set{t,u}}  \ar[d, "\overline{i_2}"]    \\
                X^1_W \sqcup_f X_{\Set{s,t}} \ar[r, "\bar{g}"]            &   (X^1_W \sqcup_f X_{\Set{s,t}}) \sqcup_g X_{\Set{t,u}}            \\
        \end{tikzcd}
        \label{eqn:inductive_gluing_step_2}
    \end{equation}
    After which we would continue with $\Set{u,v}$ in the same manner. In the final space, $X_{\Set{s,t}}$ is glued to $X_{\Set{t,u}}$ along $X_{\Set{t}}$. In general, for $T_1,T_2 \in \Delta_W^n$, we have that $X_{T_1}$ and $X_{T_2}$ are glued along $X_{T_1 \cap T_2} \subseteq X^{n-1}_W$ where $T_1 \cap T_2 \in \Delta_W^{n-1}$. We can always construct the $n$--skeleton from the $(n-1)$--skeleton in exactly this way. 
\end{example}

The exact same construction works for $X^\prime_W$. This construction may seem too abstracted, in that much of the structure is hidden away in the maps $f$ and $g$. However, as it turns out, we can use this adjunction structure without considering the details of these maps.

\begin{theorem}[{\cite[Theorem 5.5]{paolini_salvetti_kpi1_2021}}]
    \label{thm:salvetti_cx_equiv_X_prime}
    For a Coxeter group $W$, the space $X^\prime_W$ as in \cref{def:subcomplex_X_prime} is homotopy equivalent to the Salvetti complex $X_W$. 
\end{theorem}
\begin{proof}
    We achieve this inductively. To tidy our notation, in this proof we drop $W$ so that $X$, $X^\prime$, $X_T$ and $X^\prime_T$ correspond to $X_W$, $X^\prime_W$, $X_{W_T}$ and $X^\prime_{W_T}$ respectively. Let $\Delta_W^n$ be as in \cref{eg:inductive_construction}.
    
    Suppose we have the $(n-1)$--skeletons $X^{n-1}$ and $(X^\prime)^{n-1}$ and a homotopy equivalence $\phi \colon X^{n-1} \to (X^\prime)^{n-1}$.
    We wish to show that we can extend $\phi$ to a homotopy equivalence for the respective $n$--skeletons. We do so by constructing the $n$--skeletons as adjunction spaces of the $(n-1)$--skeletons as in \cref{eg:inductive_construction} and use \cref{lem:adjunction_gluing}.

    Suppose we are gluing on the cells corresponding to some $T \in \Delta_W^n$. So any $Q \subsetneq T$ will correspond to a cell in $X^{n-1}$. Let $Y$ and $Y^\prime$ be some intermediate steps in the adjunction gluing, such as the bottom--left term of \eqref{eqn:inductive_gluing_step_2}. Suppose we have a homotopy equivalence $\phi_1 \colon Y \to Y^\prime$ such that ${\phi_1}|_{X^{n-1}} = \phi$ so the following commutes.
    \begin{equation}
        \begin{tikzcd}
\bigcup_{Q \subsetneq T} X_{Q} \ar[r, "\phi" ] \ar[d, hookrightarrow]     &   \bigcup_{Q \subsetneq T} X^\prime_{Q}  \ar[d, hookrightarrow]       \\
Y  \ar[r, "\phi_1"]                                                                 &   Y^\prime                                                  \\
        \end{tikzcd}
        \label{eqn:commutative_Y_inductive_hyp}  
    \end{equation}
We wish to extend $\phi$ to a homotopy equivalence $\psi$ such that the following commutes.
    \begin{equation*}
        \begin{tikzcd}
\bigcup_{Q \subsetneq T} X_{Q}  \ar[r, "\phi"]             &   \bigcup_{Q \subsetneq T} X^\prime_{Q}                                  \\
X_T \ar[r, "\psi" ] \ar[from = u, hookrightarrow]                                  &   X^\prime_T  \ar[from = u, hookrightarrow]      \\
        \end{tikzcd}
    \end{equation*}
    This is possible by \cref{lem:extendability_of_maps}. Now we have the following commutative diagram.

\begin{equation}
    \begin{tikzcd}[row sep=scriptsize, column sep=scriptsize]
\bigcup_{Q \subsetneq T} X_{Q} \ar[dr, "\phi"'] \ar[rr, "f"] \ar[dd, hookrightarrow]     &  &   X_T \ar[dr, "\psi"] \ar[dd, hookrightarrow]  &  \\
 &   \bigcup_{Q \subsetneq T} X^\prime_{Q} \ar[rr, crossing over, "g" near start] & & X^\prime_T \ar[dd, hookrightarrow]                                    \\
Y \ar[dr, "\phi_1"'] \ar[rr, "\bar{f}"', near end]  &                                                                           &   Y \sqcup_f X_T \ar[dr, "\sigma"]                        &                                   \\
                                                &   Y^\prime \ar[rr, "\bar{g}"']  \ar[from=uu, crossing over, hookrightarrow]    &                                                       &   Y^\prime\sqcup_g X^\prime_T        \\
    \end{tikzcd}
\label{eqn:gluing_commutative_diagram_proof}
\end{equation}

    Where the induced map $\sigma$ is a homotopy equivalence by \cref{lem:adjunction_gluing}. At the next inductive step, $Y \sqcup_f X_T$ and $Y^\prime\sqcup_g X^\prime_T$ replace $Y$ and $Y^\prime$ respectively. Accordingly, $\sigma$ replaces $\phi_1$. Suppose we are next going to glue the cells corresponding to $\tilde{T} \in \Delta_W$. To proceed inductively, there are two possible outcomes:
    \begin{enumerate}
        \item We are still constructing $X^n \simeq (X^\prime)^n$ and $\tilde{T} \in \Delta_W^n$.
        \item We completely constructed $X^n \simeq (X^\prime)^n$ in the previous step and $\tilde{T} \in \Delta_W^{n+1}$.
    \end{enumerate}
    
    In Case 1, we have that any $Q \subsetneq \tilde{T}$ corresponds to cells in $X^{n-1}$. By the inductive hypothesis, we can restrict $\phi_1 \colon Y \to Y^\prime$ to $X^{n-1}$, and thus we can do the same for $\sigma$ and so the restriction $\sigma|_{X^{n-1}}$ is well--defined, we can get \eqref{eqn:commutative_Y_inductive_hyp} with the appropriate replacements and proceed inductively. 

    In Case 2, $Y \sqcup_f X_T$ and $Y^\prime\sqcup_g X^\prime_T$ are $X^n$ and $(X^\prime)^n$ respectively. Some $Q\subsetneq \tilde{T}$ will correspond to cells in $X^n$, but $\sigma$ is exactly the restriction $\sigma|_{X^n}$ so, the restriction is well--defined. We get \eqref{eqn:commutative_Y_inductive_hyp} with the appropriate replacements and proceed inductively.

    The base case is $X_\emptyset \simeq X^\prime_\emptyset \simeq \Set{\bullet}$.
\end{proof}

\end{document}