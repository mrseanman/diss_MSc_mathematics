% !TeX root = ../../main.tex
\documentclass[class=article, crop=false]{standalone}
\begin{document}
\section{Adjunction spaces and proving \texorpdfstring{$X_W \simeq X^\prime_W$}{X W homotopy equiv. to X prime W}}
\label{sec:adj_hom_equiv}

Here we will develop a formalism for gluing together two spaces in to what is called an \emph{adjunction space}. Adjunction spaces inherit homotopic properties from their two parent spaces in a useful way, that will help us to exploit \cref{rmk:salvetti_X_prime_similarities}.

\begin{definition}[Adjunction Space]
    Given two spaces $X$ and $U$, with a continuous map $f \colon A \to U$ for some subspace $A \subseteq X$, the \emph{adjunction space} $ X \sqcup_f U$  is the space formed by gluing $X$ and $U$ via the map $f$.
    \begin{equation*}
        X \sqcup_f U \coloneq (X \sqcup Y)/(a \sim f(a))
    \end{equation*}
\end{definition}

An adjunction space is associated to the commutative diagram
\begin{equation}
    \begin{tikzcd}
            A \ar[r, "f" ] \ar[d, "i"]  &   U  \ar[d, "\widetilde{i}"]    \\
            X \ar[r, "\widetilde{f}"]         &   X \sqcup_f U            \\
    \end{tikzcd}
\label{eqn:adjunction_diagram}
\end{equation}
where $i$ is inclusion of $A$ in to $X$ and $\widetilde{i}$ is inclusion of $U$ in to $X \sqcup_f U $. Suppose we also have the adjunction space $Y \sqcup_g V$ with $g \colon B \to V$ and $B\subseteq Y$. Suppose further that we have maps $ \phi_1 \colon X \to Y$, $\phi_2 \colon A \to B$ and $\phi_3 \colon U \to V$ such that the following diagram commutes.
\begin{equation}
    \begin{tikzcd}[row sep=scriptsize, column sep=scriptsize]
        A \ar[dr, "\phi_2"'] \ar[rr, "f"] \ar[dd, "i"'] & & U \ar[dr, "\phi_3"] \ar[dd, "\widetilde{i}", near start]  & \\
        &   B \ar[rr, crossing over, "g" near start] & & V \ar[dd, "\widetilde{j}"]                                     \\
        X \ar[dr, "\phi_1"'] \ar[rr, "\widetilde{f}"', near end] & & X \sqcup_f U \ar[dr, "\phi"] &                     \\
        &   Y \ar[rr, "\widetilde{g}"'] \ar[from=uu, crossing over, "j"', near start] & & Y\sqcup_gV                    \\
    \end{tikzcd}
    \label{eqn:adjunction_gluing_diagram}
\end{equation}
If all $\phi_1$, $\phi_2$ and $\phi_3$ are homotopy equivalences then the following lemma tells us that $\phi$ is also a homotopy equivalence.


\begin{lemma}[{\hspace{1sp}\cite[Theorem 7.5.7]{brown_topology_2006}}]
    Consider a commutative diagram as in \eqref{eqn:adjunction_gluing_diagram} where the front and back faces define an adjunction space as in \eqref{eqn:adjunction_diagram}. If $i$ and $j$ are closed cofibrations and $\phi_1$, $\phi_2$ and $\phi_3$ are homotopy equivalences, then the $\phi$ as determined by the diagram is also a homotopy equivalence.
    \label{lem:adjunction_gluing}
\end{lemma}

In the cases important to us, $i$ and $j$ will be cellular inclusions in to finite CW--complexes, and thus closed cofibrations. See \cite{brown_topology_2006} for more details on adjunction spaces.

To use \cref{lem:adjunction_gluing} we must be able to construct $X_W$ and $X_W^\prime$ as a sequence of adjunction spaces. Consider $X_W$ in the following example.

\begin{example}
    \label{eg:inductive_construction}
    To clean our notation in this example let $X$ and $X_T$ be shorthand for $X_W$ and $X_{W_T}$ respectively. Accordingly, $X^n$ is the $n$--skeleton for $X_W$ and $X_T^n$ is the $n$--skeleton for $X_{W_T}$.
    Suppose $W$ is generated by $S = \Set{s,t.u}$ such suppose $\mathcal{S}_W  = \Set{Q \subsetneq S}$. For convenience define $\mathcal{S}_W^n \coloneq \Set{T \in \mathcal{S}_W \given \Abs{T} = n }$. When denoting adjunction spaces, we will use $\sqcup_f$ to denote a specific adjunction space that is currently being constructed. We use the $\cup$ symbol to denote some adjunction space or space resulting from a sequence of adjunctions without specifying the map $f$. We have that $X = \bigcup_{T \in \mathcal{S}_W^2} X_T$. Suppose we had the 1--skeleton $X^1 \subsetneq X$, and some ordering on $\mathcal{S}_W^2 = (\Set{s,t}, \Set{s,u}, \Set{t,u})$. To construct $X$, we would first glue $X_{\Set{s,t}}$ to $X^1$ as an adjunction space in the following way.
    \begin{equation}
        \begin{tikzcd}
                X_{{\Set{s}}} \cup X_{{\Set{t}}} \ar[r, "f" ] \ar[d, "i_1"]   &    X_{{\Set{s,t}}} \ar[d, "\widetilde{i_1}"]    \\
                X^1 \ar[r, "\widetilde{f}"]                                 &   X^1 \sqcup_f X_{{\Set{s,t}}}            \\
        \end{tikzcd}
        \label{eqn:inductive_gluing_step_1}
    \end{equation}
    Where $f$ is inclusion of the 1--cells $X_{{\Set{s}}}$ and $X_{{\Set{t}}}$ in to $X_{\Set{s,t}}$ and $i_1$ is the inclusion in to $X^1$, which in this case makes $X^1 \sqcup_f X_{{\Set{s,t}}} \cong X_{{\Set{s,t}}}$. If we were using this construction in the proof of \cref{thm:salvetti_cx_equiv_X_prime} we would inductively assume to have already constructed $ X_{{\Set{s}}} \cup X_{{\Set{t}}}$.
    We can now add $X_{\Set{t,u}}$ to the preceding adjunction space in the following way.
    \begin{equation}
        \begin{tikzcd}
                X_{{\Set{t}}} \cup X_{{\Set{u}}} \ar[r, "g" ] \ar[d, "i_2"]   &  X_{{\Set{t,u}}}   \ar[d, "\widetilde{i_2}"]    \\
                X^1 \sqcup_f X_{{\Set{s,t}}} \ar[r, "\widetilde{g}"]            &   (X^1 \sqcup_f X_{{\Set{s,t}}}) \sqcup_g X_{{\Set{t,u}}}            \\
        \end{tikzcd}
        \label{eqn:inductive_gluing_step_2}
    \end{equation}
    Again, $g$ and $i_2$ are just inclusions. After this step, we would continue with $\Set{u,v}$ in the same manner. In the final space, $X_{\Set{u,v}}$ is glued to $(X^1_W \sqcup_f X_{\Set{s,t}}) \sqcup_g X_{\Set{t,u}}$ along $X_{\Set{u}} \cup X_{\Set{v}}$. Given the $(n-1)$--skeleton, we can always choose an order on $\Set{X_{T} \given T \in \mathcal{S}^n_W}$ and attach each such $X_{T}$ to the previous adjunction space along the $(n-1)$--skeleton of $X_{T}$ (which is always a subspace of the adjunction space of the previous step).
\end{example}

The exact same construction works for $X^\prime_W$. Note that none of the maps in \eqref{eqn:inductive_gluing_step_1} or \eqref{eqn:inductive_gluing_step_2} are the attaching maps of cells as in the CW--complex. It is possible to construct CW--complexes as a sequence of adjunction spaces, but that is not the construction presented in \cref{eg:inductive_construction}. In this construction we assume to already have fully-constructed $\Abs{T}$--dimensional CW--complexes $X_T$, then we just glue these on to the $(\Abs{T}-1)$--skeleton which is a subspace of the adjunction space formed in the previous step. All the internal structure of the CW--complexes $X_T$ is ignored. This construction may seem too abstracted to be useful, but we will soon see otherwise.

The following proof uses induction on the steps presented in \cref{eg:inductive_construction}. Note that a step is the gluing of a single $X_T$. In principle, to construct the $n$--skeleton from the $(n-1)$--skeleton takes multiple steps.

\begin{theorem}[{\hspace{1sp}\cite[Theorem 5.5]{paolini_salvetti_kpi1_2021}}]
    \label{thm:salvetti_cx_equiv_X_prime}
    For a Coxeter group $W$, the space $X^\prime_W$ as in \cref{def:subcomplex_X_prime} is homotopy equivalent to the Salvetti complex $X_W$. 
\end{theorem}
\begin{proof}
    We achieve this by induction on adjunction gluing steps. To tidy our notation, as in \cref{eg:inductive_construction}, we drop $W$ so that $X$, $X^\prime$, $X_T$ and $X^\prime_T$ correspond to $X_W$, $X^\prime_W$, $X_{W_T}$ and $X^\prime_{W_T}$ respectively. Let $\mathcal{S}_W^n$ be as in \cref{eg:inductive_construction}.

    Our inductive hypotheses are as follows:

    \begin{enumerate}
        \item We have a homotopy equivalence between $(n-1)$--skeletons $\alpha \colon X^{n-1} \to (X^\prime)^{n-1}$.
        \item For any subset $\mathcal{T} \subseteq \mathcal{S}_W^{n-1}$ we have that $\alpha$ restricts to a homotopy equivalence $\alpha^\prime \colon \bigcup_{Q \in \mathcal{T}}X_Q \to \bigcup_{Q \in \mathcal{T}}X^\prime_Q$ such that the following diagram commutes.
        \begin{equation*}
            \begin{tikzcd}
    \bigcup_{Q \in \mathcal{T}} X_{Q} \ar[r, "\alpha^\prime" ] \ar[d, hookrightarrow]     &   \bigcup_{Q \in \mathcal{T}} X^\prime_{Q}  \ar[d, hookrightarrow]       \\
    X^{n-1}  \ar[r, "\alpha"]                                                                 &   (X^\prime)^{n-1}                                                  \\
            \end{tikzcd}
            \label{eqn:commutative_skeleton_inductive_hyp}  
        \end{equation*}
        \item We have a homotopy equivalence $\beta \colon Y \to Y^\prime$ where $X^{n-1} \subseteq Y \subseteq X^n$ and $(X^\prime)^{n-1} \subseteq Y^\prime \subseteq (X^\prime)^n$ so $Y$ and $Y^\prime$ are intermediate steps in constructing $X^n$ and $(X^\prime)^n$ respectively. For example, $Y$ could be the upper-right term in \eqref{eqn:inductive_gluing_step_2} for $n=2$. This homotopy equivalence $\beta$ restricts to the homotopy equivalence $\alpha$ on $X^{n-1}$.
    \end{enumerate}

    The base case is $X_\emptyset \simeq X^\prime_\emptyset \simeq \Set{\bullet}$.
    Fix an ordering $(T_1, T_2, \ldots , T_k)$ on $\mathcal{S}^n_W$.
    For each $T_i$ we can extend the map $\phi_i \colon \bigcup_{Q \subsetneq T_i} X_{Q} \to \bigcup_{Q \subsetneq T_i} X^\prime_{Q}$ to $\psi_i \colon X_{T_i} \to X^\prime_{T_i}$ such that we have the following commutative diagram.
    \vspace{0.2cm}
    \begin{equation*}
        \begin{tikzcd}
\bigcup_{Q \subsetneq T_i} X_{Q}  \ar[r, "\phi_i"]             &   \bigcup_{Q \subsetneq T_i} X^\prime_{Q}                                  \\
X_{T_i} \ar[r, "\psi_i" ] \ar[from = u, hookrightarrow]                                  &   X^\prime_{T_i}  \ar[from = u, hookrightarrow]      \\
        \end{tikzcd}
    \end{equation*}
    \vspace{-0.2cm}
    This is possible by \cref{lem:extendability_of_maps}. The extra axioms in the case where $\Abs{T}=2$ are satisfied due to inductive hypothesis 2. Fix each $\psi_i$.

    $\bullet$ Case 1: Suppose $Y = \bigcup_{m < i} X_{T_m}$ and $Y^\prime = \bigcup_{m < i} X^\prime_{T_m}$ and $i\leq k-1$, i.e.~in the next step, we \emph{will not} complete $X^n \simeq (X^\prime)^n$.

    In this step, we will glue $X_{T_{i}}$ to $Y$ and $X^\prime_{T_{i}}$ to $Y^\prime$ such that the inductive hypotheses remain for these resulting adjunction spaces.
    By combining inductive hypothesis 3 and 2, we have the following commutative diagram.

    \begin{equation*}
        \begin{tikzcd}
\bigcup_{Q \subsetneq T_i} X_{Q} \ar[r, "\phi_i" ] \ar[d, hookrightarrow]     &   \bigcup_{Q \subsetneq T_i} X^\prime_{Q}  \ar[d, hookrightarrow]       \\
Y  \ar[r, "\beta"]                                                                 &   Y^\prime                                                  \\
        \end{tikzcd}
        \label{eqn:commutative_Y_inductive_hyp}  
    \end{equation*}

    Where $\phi_i$ and $\beta$ are homotopy equivalences and the vertical maps are inclusions. We now have the following commutative diagram.

    \begin{equation}
        \begin{tikzcd}[row sep=scriptsize, column sep=scriptsize]
            \bigcup_{Q \subsetneq T_i} X_{Q} \ar[dr, "\phi_i"'] \ar[rr, hookrightarrow, "f"] \ar[dd, hookrightarrow]     &  &   X_{T_i} \ar[dr, "\psi_i"] \ar[dd, hookrightarrow]  &  \\
            &   \bigcup_{Q \subsetneq T_i} X^\prime_{Q} \ar[rr, crossing over, hookrightarrow, "g" near start] & &  X^\prime_{T_i}\ar[dd, hookrightarrow]                         \\
            Y \ar[dr, "\beta"'] \ar[rr, hookrightarrow, "\widetilde{f}"', near end]  & &   Y \sqcup_f X_{T_i} \ar[dr, "\sigma"] &                                              \\
            &   Y^\prime \ar[rr, hookrightarrow, "\widetilde{g}"']  \ar[from=uu, crossing over, hookrightarrow]    &    &   Y^\prime\sqcup_g X^\prime_{T_i}                     \\
        \end{tikzcd}
    \label{eqn:gluing_commutative_diagram_proof}
    \end{equation}
    Where all maps not coming out of the plane of the page are inclusions and front and back faces determine the adjunction spaces $ Y \sqcup_f X_{T_i}$ and $Y^\prime\sqcup_g X^\prime_{T_i}$ respectively. By \cref{lem:adjunction_gluing}, the map $\sigma$ determined by the maps $\beta$, $\phi_i$ and $\psi_i$ is a homotopy equivalence.
    In the next step, we will still be constructing $X^n \simeq (X^\prime)^n$ and will be gluing $X_{T_{i+1}}$ to $Y$ and $X^\prime_{T_{i+1}}$ to $Y^\prime$. From \eqref{eqn:gluing_commutative_diagram_proof}, we see that $\sigma$ restricts to $\beta$ and so by induction also restricts to $\alpha \colon X^{n-1} \to (X^\prime)^{n-1}$. We replace $Y$ with $ Y \sqcup_f X_{T_i}$ and $Y^\prime$ with $Y^\prime\sqcup_g X^\prime_{T_i}$. Accordingly, we replace $\beta$ with $\sigma$. 

    With these replacements, we maintain all our inductive hypotheses and can continue.

    $\bullet$ Case 2: Suppose $Y = \bigcup_{m < i} X_{T_m}$ and $Y^\prime = \bigcup_{m < i} X^\prime_{T_m}$ and $i = k$, i.e.~in the next step, we \emph{will} complete $X^n \simeq (X^\prime)^n$.

    The steps up to creating the commutative diagram \eqref{eqn:gluing_commutative_diagram_proof} are exactly the same as in Case 1. However, we must make a further argument to ensure the inductive hypotheses continue to be true for the next inductive step. Suppose we have completed the adjunction spaces $ Y \sqcup_f X_{T_i}$ and $Y^\prime\sqcup_g X^\prime_{T_i}$ and have a homotopy equivalence $\sigma$ between them. In this case, $ Y \sqcup_f X_{T_i} \cong X^n$ and $Y^\prime\sqcup_g X^\prime_{T_i} \cong (X^\prime)^{n}$, so $\sigma$ is a homotopy equivalence $\sigma \colon X^n \to (X^\prime)^n$. For the next step we will replace both $\alpha$ and $\beta$ with $\sigma$. We immediately achieve inductive hypotheses 1 and 3. But 2 is not obvious.
    
    Recall that we had fixed an ordering $(T_1, T_2, \ldots , T_k)$ for $\mathcal{S}^{n}_W$. We see that in the intermediate steps before constructing the $n$--skeletons, we constructed all $\bigcup_{m \leq l} X_{T_m}$ and $\bigcup_{m \leq l} X_{T_m}$ for each $l\leq k$. Furthermore, we constructed homotopy equivalences between $\bigcup_{m \leq l} X_{T_m}$ and $\bigcup_{m \leq l} X_{T_m}$ that are restrictions of our homotopy equivalence $\sigma \colon X^n \to (X^\prime)^n$. Therefore, we have inductive hypothesis 2, but only for subsets of $\mathcal{S}_W^n$ that correspond to a prefix of our ordering $(T_1, T_2, \ldots , T_k)$. Clearly this is not all subsets of $\mathcal{S}_W^n$.
    
    However, suppose we are given an arbitrary subset $\mathcal{T} \subseteq \mathcal{S}_W^n$. We could have chosen an ordering $\chi = (T_{a_1}, T_{a_2}, \ldots, T_{a_k})$ of $\mathcal{S}^n_W$ such that $\mathcal{T}$ is some prefix of $\chi$. We would have continued through the same steps and obtained a (potentially different) homotopy equivalence $\sigma_\chi \colon X^n \to (X^\prime)^n$. We observe from \eqref{eqn:gluing_commutative_diagram_proof} that $\sigma_\chi$ must restrict to the homotopy equivalence $\psi_i \colon X_{T_i} \to X^\prime_{T_i}$ for all $1\leq i \leq k$. These $\psi_i$ are each fixed. Since $ \bigcup_{1\leq m\leq k} X_{T_m} = X^n$, these $\psi_i$ completely determine $\sigma$ (or $\sigma_\chi$) as a map, so $\sigma = \sigma_\chi$. Thus, $\sigma$ is independent of our choice of ordering and the restriction of $\sigma$ to $\bigcup_{Q \in \mathcal{T}}X_Q$ for \emph{any} $\mathcal{T}\subseteq \mathcal{S}^n_W$ is the necessary homotopy equivalence to satisfy inductive hypothesis 2.
    
    Our proof concludes by observing that there is a minimal $n$ such that $\mathcal{S}^{n+1}_W = \emptyset$, thus giving $X^n = X$ and $(X^\prime)^n = X^\prime$. The inductive process outlined above eventually ends with a homotopy equivalence witnessing $X \simeq X^\prime$.
\end{proof}

\end{document}