% !TeX root = ../../main.tex
\documentclass[class=article, crop=false]{standalone}
\begin{document}

\section{An Adjunction Homotopy Equivalence}

Here we will show a fundamental link between homomorphisms in to groups $G$ and maps in to classifying spaces for $G$. This result shows that in a certain way, the homotopy type of classifying spaces is unique. We will use this result to then show an important homotopy equivalence, which is an intermediate step in proving the main result of \cite{paolini_salvetti_kpi1_2021}.

\begin{lemma}
    A null homotopic map $\rho \colon S^n \to X$ can be extended to a map $\sigma \colon D^{n+1} \to X$.
    \label{lem:null_homotopic_extension}
\end{lemma}
\begin{proof}
    Let $H \colon S^n \times I \to X$ witness the null homotopy with $H|_{S^n \times \Set{1}} \colon S^n \to \Set{x_0}$. We have that $H$ factors uniquely through $(S^n \times I)/(S^n \times \Set{1}) \cong D^{n+1}$. With $\sigma$ being the necessary map as below.

    \begin{equation*}
        \begin{tikzcd}
            &S^n \times I       \arrow[d, "q"]      \arrow[r, "H"]                                &X \\
            &(S^n \times I)/(S^n \times \Set{1})    \arrow[ru, "\exists ! \sigma", swap, dotted]
        \end{tikzcd}
    \end{equation*}

\end{proof}

\begin{theorem}[{\cite[Proposition 1B.9]{hatcher_algebraic_2001}}]
    Let $Y$ be a $K(G,1)$ space and $X$ a finite dimensional CW complex consisting of one 0--cell, the point $x_0$. Any homomorphism $\phi \colon \pi_1(X,x_0) \to \pi_1(Y,y_0)$ is induced by a map $\tilde{\phi} \colon X \to Y$ where $\tilde{\phi}$ is unique up to homotopy fixing $x_0$.
    \label{thm:iso_induces_unique_map_for_K_Pi_1}
\end{theorem}

\begin{proof}
    Clearly we must have $\tilde{\phi}(x_0) = y_0$. The 1--skeleton $X^1$ will be a wedge of circles and there is thus a presentation of $\pi_1(X, x_0)$ with each cell $e_\alpha^1$ corresponding to a generator $ [e_\alpha^1] \in \pi_1(X, x_0)$. We can choose $\tilde{\phi}(e_\alpha)$ to trace out a path corresponding to $\phi([e_\alpha^1]) \in \pi_1(Y,y_0)$ for each $e_\alpha^1 \in X^1$.
    
    Let $\psi_\beta \colon S^1 \to X^1$ be an attaching map for a 2--cell $e_\beta^2 \subseteq X$. Let $i \colon X^1 \hookrightarrow X$ be the inclusion. We have that $i_*$ is the surjetcion from the free group generated by each $e_\alpha^1$ to $\pi_1(X,x_0)$. The attaching of the 2--cell $e_\beta^2$ provides a null homotopy for the path traced by $\psi_\beta$. In the presentation of $\pi_1(X, x_0)$ as above, each relation corresponds to the path of a $\psi_\beta$. Thus, $i_*([\psi_\beta]) = 0$ and so $\tilde{\phi}_*([\psi_\beta]) = \phi \circ i_*([\psi_\beta]) = 0$. Thus, $\tilde{\phi} \circ \psi_\beta$ is null homotopic and so can be extended over all of the closure of $e_\beta^2$ by \cref{lem:null_homotopic_extension}. This is an extension of $\tilde{\phi}$ and repeating this allows us to extend $\tilde{\phi}$ over all of $X^2$.
    
    To extend $\tilde{\phi}$ over $e_\gamma^3$ we use that $S^{2}$ is simply connected (as for any $S^n$ with $n\geq 2$) and so for the attaching map $\psi_\gamma \colon S^2 \to X^2$ we have that $\tilde{\phi} \circ \psi_\gamma$ lifts to the universal cover of $Y$, which is contractible since $Y$ is a $K(G,1)$, so $\tilde{\phi} \circ \psi_\gamma$ is null homotopic. This same argument applies for any $e_\delta^n$ for $n\geq 3$. We can thus extend $\tilde{\phi}$ over the 3--cells and proceeding inductively, over all of $X$.

    Now we turn to the uniqueness of $\tilde{\phi}$ up to homotopy. Let $\phi$ be some homomorphism and $\tilde{\phi}_0$ and $\tilde{\phi}_1$ be any such maps constructed as above. Clearly $\tilde{\phi}_0(x_0) = \tilde{\phi}_1(x_0)$ and $\tilde{\phi}_0|_{X^1} \sim \tilde{\phi}_1|_{X^1}$ by the restrictions of our construction. Let $H$ witness this homotopy. Give $X \times I$ the following CW structure: Let $X\times \Set{0}$ and $X\times \Set{1}$ both have the same cell structure as $X$ with cells notated $d_\alpha^n$ and $e_\alpha^n$ respectively. Connect $d^0$ to $e^0$ with a 1--cell $s^1$, called \emph{the spine}. Connect a 2--cell $s_\alpha^2$ along $d_\alpha^1$, then $s^1$ then $e_\alpha^1$ then back along $s^1$ with opposite orientations on $d_\alpha^1$ and $e_\alpha^1$ such that $d^0 \cup e^0 \cup s^1 \cup d_\alpha^1 \cup e_\alpha^1 \cup s_\alpha^2 \cong S^1 \times I$. The spine now consists of $s_1 \cup s_\alpha^2$. Repeat this for each 1--cell in $X$ and then repeat for each 2--cell and so on, attaching an $s_\beta^n$ along $d_\beta^{n-1}$, $e_\beta^{n-1}$ and $s_\beta^{n-1}$, inductively building up the spine. A picture of this CW complex completed for one $s_\alpha^2$ is below.
    \begin{equation*}
    \begin{tikzpicture}
        \tikzstyle{every label}=[font=\scriptsize]
        \tikzstyle{every node}=[font=\scriptsize]

        \node[FSC] (d_0) at (0,0)               [label={[label distance=-4pt]-60:{$d^0$}}]      {};
        \node[FSC] (e_0) at ($(d_0) + (3,0)$)   [label={[label distance=-2pt]240:{$e_0$}}]      {};

        \node            at ($(d_0) + (-0.5,1.5)$)                                              {$d_\alpha^1$};
        \node            at ($(e_0) + (0.5,1.5)$)                                               {$e_\alpha^1$};
        \node            at ($(d_0)!0.5!(e_0) + (0,1.8)$)                                       {$s_\alpha^2$};
        \draw (d_0) to node[below] {$s^1$} (e_0);

        \draw[rotate = 0, name path=d_11]   (d_0) .. controls ($(d_0) + (-1,2)$) and ($(d_0) + (1,2)$) .. (d_0);
        \draw[rotate = 60, name path=d_12]  (d_0) .. controls ($(d_0) + (-1,2)$) and ($(d_0) + (1,2)$) .. (d_0);
        \draw[rotate = 120, name path=d_12] (d_0) .. controls ($(d_0) + (-1,2)$) and ($(d_0) + (1,2)$) .. (d_0);


        \draw[rotate = 0, name path=e_11]    (e_0) .. controls ($(e_0) + (-1,2)$) and ($(e_0) + (1,2)$) .. (e_0);
        \draw[rotate = -60, name path=e_12]  (e_0) .. controls ($(e_0) + (-1,2)$) and ($(e_0) + (1,2)$) .. (e_0);
        \draw[rotate = -120, name path=e_12] (e_0) .. controls ($(e_0) + (-1,2)$) and ($(e_0) + (1,2)$) .. (e_0);


        \foreach \y in {0.1,0.4,...,1.3}{
            \path[name path=s_1] ($(d_0) + (-2,\y)$)          to ($(e_0) + (2,\y)$);
            \path[name path=s_2] ($(d_0) + (-2,\y)+ (0,0.1)$) to ($(e_0) + (2,\y) + (0,0.1)$);
            \node [coordinate, name intersections = {of = d_11 and s_1}] (s_l_1) at (intersection-1) {};
            \node [coordinate, name intersections = {of = d_11 and s_2}] (s_l_2) at (intersection-2) {};

            \node [coordinate, name intersections = {of = e_11 and s_1}] (s_r_1) at (intersection-1) {};
            \node [coordinate, name intersections = {of = e_11 and s_2}] (s_r_2) at (intersection-2) {};

            \draw[white!80!black, line width=0.1mm] (s_l_1) to (s_r_1);
            \begin{pgfonlayer}{background}
                \draw[white!80!black, line width=0.05mm, dashed] (s_l_2) to (s_r_2);
            \end{pgfonlayer}
        }
    \end{tikzpicture}
    \vspace{-0.6cm}
    \end{equation*}
    We can now extend the domain of $H$ from $X^1 \times I$ to all of $X \times I$ using this cell structure. Note that now we have two 0--cells, but this does not cause any issues. Let $H$ have domain $X^1 \times I \subseteq X \times I$. Now extend $H$ such that $H|_{X\times \Set{0}}$ agrees with $\tilde{\phi}_0$ and $H|_{X\times \Set{1}}$ agrees with $\tilde{\phi}_1$. This is possible because $H$ is a homotopy between restrictions of these maps. Note that now $H$ is defined on the whole 2--skeleton of $X \times I$. We can extend $H$ to all the higher dimensional cells by the exact same argument as before, using the contractability of the universal cover of $Y$. Thus, we have a continuous function $H \colon X \times I \to Y$ witnessing the homotopy $\tilde{\phi}_0 \sim \tilde{\phi}_1$.
\end{proof}

\begin{corollary}
    Let $X$ and $Y$ both be $K(G,1)$ spaces. Any isomorphism $\phi \colon \pi_1(X,x_0) \to \pi_1(Y,y_0)$ induces a homotopy equivalence witnessing $X \simeq Y$.
    \label{cor:iso_K_G_1_induces_hom_equiv}
\end{corollary}
\begin{proof}
    We have maps $\tilde{\phi} \colon X \to Y$ and $\widetilde{(\phi^{-1})} \colon Y \to X$ with $(\tilde{\phi} \circ \widetilde{(\phi^{-1})})_* = \id_{\pi_1(Y,y_0)}$. Thus, since the homotopy class of such maps is determined by the induced action on their fundamental groups $\tilde{\phi}\circ \widetilde{(\phi^{-1})} \sim \id_Y$. Similarly, $\widetilde{(\phi^{-1})} \circ \tilde{\phi} \sim \id_X$.
\end{proof}

\begin{lemma}
    \label{lem:extendability_of_maps}
    Let $T \in \Delta_W \setminus \emptyset$. Let $\phi \colon \bigcup_{Q \subsetneq T} X_{W_Q} \to \bigcup_{Q \subsetneq T} X^\prime_{W_Q}$ be a homotopy equivalence. We can extend $\phi$ to a homotopy equivalence $\psi \colon X_{W_T} \to X^\prime_{W_T}$ such that the following diagram commutes.
    \begin{equation*}
        \begin{tikzcd}
\bigcup_{Q \subsetneq T} X_{W_Q}  \ar[r, "\phi"]             &   \bigcup_{Q \subsetneq T} X^\prime_{W_Q}                                                   \\
X_{W_T} \ar[r, "\psi" ] \ar[from=u, hookrightarrow]                                  &   X^\prime_{W_T}  \ar[from = u, hookrightarrow]      \\
        \end{tikzcd}
    \end{equation*}
\end{lemma}
\begin{proof}
    We prove this by cases. By \cref{rmk:salvetti_X_prime_similarities}, $X_{W_T}$ and $X^\prime_{W_T}$ are classifying spaces.
    \begin{enumerate}[i)]
        \item If $\Abs{T}=1$ then $Q$ is uniquely $\emptyset$. Let $\psi$ be any map witnessing $X_{W_T} \simeq X^\prime_{W_T}$ that fixes the point corresponding to $X_{W_Q}$.
        \item If $\Abs{T} = 2$ then we can extend $\phi$ to $\psi$ such that $\psi_* \colon \pi_1(X_{W_T}, X_{\emptyset}) \to \pi_1(X^\prime_{W_T}, X^\prime_{\emptyset})$ is an isomorphism using the same argument as in the proof of \cref{thm:iso_induces_unique_map_for_K_Pi_1}. This map is a homotopy equivalence by \cref{cor:iso_K_G_1_induces_hom_equiv}.
        \item If $\Abs{T} \geq 3$ then we can extend $\phi$ to some map $\psi$ using the same methods as in \cref{thm:iso_induces_unique_map_for_K_Pi_1}. In this case, $\bigcup_{Q \subsetneq T} X_{W_Q}$ contains the 2--skeleton of $X_T$ and similarly for $\bigcup_{Q \subsetneq T} X^\prime_{W_Q}$ and $X^\prime_T$. So $\pi_1(\bigcup_{Q \subsetneq T} X_{W_Q}, X_{W_\emptyset}) =\pi_1(X_T, X_{W_\emptyset})$ and similarly for $X^\prime_T$.  By assumption $\phi$ is a homotopy equivalence and so $\phi_*$ is an isomorphism. Therefore, $\psi_*$ is an isomorphism \cite[Corollary 4.12]{hatcher_algebraic_2001} and thus $\psi$ a homotopy equivalence by \cref{cor:iso_K_G_1_induces_hom_equiv}. 
    \end{enumerate}
\end{proof}

\begin{definition}[Adjunction Space]
    For two spaces $X$ and $U$, with a continuous map $f \colon A \to U$ for some subspace $A \subseteq X$. The \emph{adjunction space} $ X \sqcup_f U$  is the space formed by gluing $X$ and $U$ via the map $f$.
    \begin{equation*}
        X \sqcup_f U \coloneq (X \sqcup Y)/(a \sim f(a))
    \end{equation*}
\end{definition}

An adjunction space is associated to the commutative diagram
\begin{equation}
    \begin{tikzcd}
            A \ar[r, "f" ] \ar[d, "i"]  &   U  \ar[d, "\bar{i}"]    \\
            X \ar[r, "\bar{f}"]         &   X \sqcup_f U            \\
    \end{tikzcd}
\label{eqn:adjunction_diagram}
\end{equation}
where $i$ is inclusion of $A$ in to $X$ and $\bar{i}$ is inclusion of $U$ in to $X \sqcup_f U $. Suppose we also have the adjunction space $Y \sqcup_g V$ with $g \colon B \to V$ and $B\subseteq Y$. Suppose further that we have maps $ \phi_1 \colon X \to Y$, $\phi_2 \colon A \to B$ and $\phi_3 \colon U \to V$ such that the following diagram commutes.
\begin{equation}
    \begin{tikzcd}[row sep=scriptsize, column sep=scriptsize]
A \ar[dr, "\phi_2"'] \ar[rr, "f"] \ar[dd, "i"']     &                                                                           &   U \ar[dr, "\phi_3"] \ar[dd, "\bar{i}", near start]  &                         \\
                                                &   B \ar[rr, crossing over, "g" near start]                                &                                                       &   V \ar[dd, "\bar{j}"]  \\
X \ar[dr, "\phi_1"'] \ar[rr, "\bar{f}"', near end]  &                                                                           &   X \sqcup_f U \ar[dr, "\phi"]                        &                         \\
                                                &   Y \ar[rr, "\bar{g}"']  \ar[from=uu, crossing over, "j"', near start]    &                                                       &   Y\sqcup_gV            \\
    \end{tikzcd}
\label{eqn:adjunction_gluing_diagram}
\end{equation}
If all $\phi_1$, $\phi_2$ and $\phi_3$ are homotopy equivalences then the following lemma tells us that $\phi$ is also a homotopy equivalence.


\begin{lemma}[{\cite[Theorem 7.5.7]{brown_topology_2006}}]
    Consider a commutative diagram as in \eqref{eqn:adjunction_gluing_diagram} where the front and back faces define an adjunction space as in \eqref{eqn:adjunction_diagram}. If $i$ and $j$ are closed cofibrations and $\phi_1$, $\phi_2$ and $\phi_3$ are homotopy equivalences, then the $\phi$ as determined by the diagram is also a homotopy equivalence.
    \label{lem:adjunction_gluing}
\end{lemma}

The restriction of $i$ and $j$ being closed cofibrations is quite mild. In the cases important to us, $i$ and $j$ will be cellular inclusions in to finite CW complexes, and thus closed cofibrations. See \cite{brown_topology_2006} for more details on pushout squares and adjunction spaces.

To use \cref{lem:adjunction_gluing} we must be able to construct $X_W$ and $X_W^\prime$ as a sequence of adjunction spaces. Consider $X_W$ in the following example.

\begin{example}
    \label{eg:inductive_construction}
    Let $\Delta_W = \Set{\emptyset, \Set{s}, \Set{t}, \Set{u}, \Set{s,t}, \Set{s,u}, \Set{t,u}}$ and let $\Delta_W^n \coloneq \Set{T \in \Delta_W \given \Abs{T} = n }$. Clearly we have that $X_W = \bigcup_{T \in \Delta_W^2} X_{W_T}$ with the appropriate gluing. Suppose we had the 1--skeleton $X^1_W \subsetneq X_W$, and some ordering on $\Delta_W^2 = (\Set{s,t}, \Set{s,u}, \Set{t,u})$. To construct $X_W$, we would first glue $X_{\Set{s,t}}$ to $X_W^1$ as an adjunction space in the following way.
    \begin{equation*}
        \begin{tikzcd}
                X_{\Set{s}} \cup X_{\Set{t}} \ar[r, "f" ] \ar[d, "i_1"]   &   X_{\Set{s,t}}  \ar[d, "\overline{i_1}"]    \\
                X_W^1 \ar[r, "\bar{f}"]                                 &   X^1_W \sqcup_f X_{\Set{s,t}}            \\
        \end{tikzcd}
    \end{equation*}
    Where $f$ is inclusion of those 1--cells in to $X_{\Set{s,t}}$, which in this case makes $X^1_W \sqcup_f X_{\Set{s,t}} \cong X_{\Set{s,t}}$. Note that $ X_{\Set{s}} \cup X_{\Set{t}}$ is really shorthand for another adjunction space, which we assume to have been already constructed.
    We can then add $X_{\Set{t,u}}$ to the preceding adjunction space in the following way.
    \begin{equation}
        \begin{tikzcd}
                X_{\Set{t}} \cup X_{\Set{u}} \ar[r, "g" ] \ar[d, "i_2"]   &   X_{\Set{t,u}}  \ar[d, "\overline{i_2}"]    \\
                X^1_W \sqcup_f X_{\Set{s,t}} \ar[r, "\bar{g}"]            &   (X^1_W \sqcup_f X_{\Set{s,t}}) \sqcup_g X_{\Set{t,u}}            \\
        \end{tikzcd}
        \label{eqn:inductive_gluing_step_2}
    \end{equation}
    After which we would continue with $\Set{u,v}$ in the same manner. In the final space, $X_{\Set{s,t}}$ is glued to $X_{\Set{t,u}}$ along $X_{\Set{t}}$. In general, for $T_1,T_2 \in \Delta_W^n$, we have that $X_{T_1}$ and $X_{T_2}$ are glued along $X_{T_1 \cap T_2} \subseteq X^{n-1}_W$ where $T_1 \cap T_2 \in \Delta_W^{n-1}$. We can always construct the $n$--skeleton from the $(n-1)$--skeleton in exactly this way. 
\end{example}

The exact same construction works for $X^\prime_W$. This construction may seem too abstracted, in that much of the structure is hidden away in the maps $f$ and $g$. However, as it turns out, we can use this adjunction structure without considering the details of these maps.

\begin{theorem}[{\cite[Theorem 5.5]{paolini_salvetti_kpi1_2021}}]
    \label{thm:salvetti_cx_equiv_X_prime}
    For a Coxeter group $W$, the space $X^\prime_W$ as in \cref{def:subcomplex_X_prime} is homotopy equivalent to the Salvetti complex $X_W$. 
\end{theorem}
\begin{proof}
    We achieve this inductively. To tidy our notation, in this proof we drop $W$ so that $X$, $X^\prime$, $X_T$ and $X^\prime_T$ correspond to $X_W$, $X^\prime_W$, $X_{W_T}$ and $X^\prime_{W_T}$ respectively. Let $\Delta_W^n$ be as in \cref{eg:inductive_construction}.
    
    Suppose we have the $(n-1)$--skeletons $X^{n-1}$ and $(X^\prime)^{n-1}$ and a homotopy equivalence $\phi \colon X^{n-1} \to (X^\prime)^{n-1}$.
    We wish to show that we can extend $\phi$ to a homotopy equivalence for the respective $n$--skeletons. We do so by constructing the $n$--skeletons as adjunction spaces of the $(n-1)$--skeletons as in \cref{eg:inductive_construction} and use \cref{lem:adjunction_gluing}.

    Suppose we are gluing on the cells corresponding to some $T \in \Delta_W^n$. So any $Q \subsetneq T$ will correspond to a cell in $X^{n-1}$. Let $Y$ and $Y^\prime$ be some intermediate steps in the adjunction gluing, such as the bottom--left term of \eqref{eqn:inductive_gluing_step_2}. Suppose we have a homotopy equivalence $\phi_1 \colon Y \to Y^\prime$ such that ${\phi_1}|_{X^{n-1}} = \phi$ so the following commutes.
    \begin{equation}
        \begin{tikzcd}
\bigcup_{Q \subsetneq T} X_{Q} \ar[r, "\phi" ] \ar[d, hookrightarrow]     &   \bigcup_{Q \subsetneq T} X^\prime_{Q}  \ar[d, hookrightarrow]       \\
Y  \ar[r, "\phi_1"]                                                                 &   Y^\prime                                                  \\
        \end{tikzcd}
        \label{eqn:commutative_Y_inductive_hyp}  
    \end{equation}
We wish to extend $\phi$ to a homotopy equivalence $\psi$ such that the following commutes.
    \begin{equation*}
        \begin{tikzcd}
\bigcup_{Q \subsetneq T} X_{Q}  \ar[r, "\phi"]             &   \bigcup_{Q \subsetneq T} X^\prime_{Q}                                  \\
X_T \ar[r, "\psi" ] \ar[from = u, hookrightarrow]                                  &   X^\prime_T  \ar[from = u, hookrightarrow]      \\
        \end{tikzcd}
    \end{equation*}
    This is possible by \cref{lem:extendability_of_maps}. Now we have the following commutative diagram.

\begin{equation}
    \begin{tikzcd}[row sep=scriptsize, column sep=scriptsize]
\bigcup_{Q \subsetneq T} X_{Q} \ar[dr, "\phi"'] \ar[rr, "f"] \ar[dd, hookrightarrow]     &  &   X_T \ar[dr, "\psi"] \ar[dd, hookrightarrow]  &  \\
 &   \bigcup_{Q \subsetneq T} X^\prime_{Q} \ar[rr, crossing over, "g" near start] & & X^\prime_T \ar[dd, hookrightarrow]                                    \\
Y \ar[dr, "\phi_1"'] \ar[rr, "\bar{f}"', near end]  &                                                                           &   Y \sqcup_f X_T \ar[dr, "\sigma"]                        &                                   \\
                                                &   Y^\prime \ar[rr, "\bar{g}"']  \ar[from=uu, crossing over, hookrightarrow]    &                                                       &   Y^\prime\sqcup_g X^\prime_T        \\
    \end{tikzcd}
\label{eqn:gluing_commutative_diagram_proof}
\end{equation}

    Where the induced map $\sigma$ is a homotopy equivalence by \cref{lem:adjunction_gluing}. At the next inductive step, $Y \sqcup_f X_T$ and $Y^\prime\sqcup_g X^\prime_T$ replace $Y$ and $Y^\prime$ respectively. Accordingly, $\sigma$ replaces $\phi_1$. Suppose we are next going to glue the cells corresponding to $\tilde{T} \in \Delta_W$. To proceed inductively, there are two possible outcomes:
    \begin{enumerate}
        \item We are still constructing $X^n \simeq (X^\prime)^n$ and $\tilde{T} \in \Delta_W^n$.
        \item We completely constructed $X^n \simeq (X^\prime)^n$ in the previous step and $\tilde{T} \in \Delta_W^{n+1}$.
    \end{enumerate}
    
    In Case 1, we have that any $Q \subsetneq \tilde{T}$ corresponds to cells in $X^{n-1}$. By the inductive hypothesis, we can restrict $\phi_1 \colon Y \to Y^\prime$ to $X^{n-1}$, and thus we can do the same for $\sigma$ and so the restriction $\sigma|_{X^{n-1}}$ is well--defined, we can get \eqref{eqn:commutative_Y_inductive_hyp} with the appropriate replacements and proceed inductively. 

    In Case 2, $Y \sqcup_f X_T$ and $Y^\prime\sqcup_g X^\prime_T$ are $X^n$ and $(X^\prime)^n$ respectively. Some $Q\subsetneq \tilde{T}$ will correspond to cells in $X^n$, but $\sigma$ is exactly the restriction $\sigma|_{X^n}$ so, the restriction is well--defined. We get \eqref{eqn:commutative_Y_inductive_hyp} with the appropriate replacements and proceed inductively.

    The base case is $X_\emptyset \simeq X^\prime_\emptyset \simeq \Set{\bullet}$.
\end{proof}

\end{document}