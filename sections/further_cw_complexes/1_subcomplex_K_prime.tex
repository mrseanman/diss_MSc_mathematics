% !TeX root = ../../main.tex
\documentclass[class=article, crop=false]{standalone}
\begin{document}
\subsection{The subcomplex $X_{W}^\prime$}
The definition of $K_{W_w}$ depends on the data of some Coxeter element $w$ and thus implicitly some generating set $S$. For some $w = s_1s_2\cdots s_n$ and some $T\subseteq S$, define $w_T\in W$ to correspond subword of $w$ consisting of elements that are in $T$, respecting the original order in $w$. 
\begin{definition}[The subcomplex]
    Define $X_{W}^\prime$ to be the finite subcomplex of $K_{W_w}$ consisting only of simplices $[x_1 | \cdots | x_n]$ such that $x_1x_2\cdots x_n \in [1,w_T]^W$ for some $T \in \Delta_W$. Recalling $\Delta_W$ from \cref{def:delta_W_finite}. 
\end{definition}
Note that we have dropped the $w$ index. That will be justified in this section as we will show that, up to homotopy equivalence, $X_W^\prime$ has no $w$ dependence.

\begin{lemma}
    A null homotopic map $\rho \colon S^n \to X$ can be extended to a map $\sigma \colon D^{n+1} \to X$.
    \label{lem:null_homotopic_extension}
\end{lemma}
\begin{proof}
    Let $H \colon S^n \times I \to X$ witness the null homotopy with $H|_{S^n \times \Set{1}} \colon S^n \to \Set{x_0}$. We have that $H$ factors uniquely through $(S^n \times I)/(S^n \times \Set{1}) \cong D^{n+1}$. With $\sigma$ being the necessary map as below.

    \begin{equation*}
        \begin{tikzcd}
            &S^n \times I       \arrow[d, "q"]      \arrow[r, "H"]                                &X \\
            &(S^n \times I)/(S^n \times \Set{1})    \arrow[ru, "\exists ! \sigma", swap, dotted]
        \end{tikzcd}
    \end{equation*}

\end{proof}

\begin{theorem}[{\cite[Proposition 1B.9]{hatcher_algebraic_nodate}}]
    Let $Y$ be a $K(G,1)$ space and $X$ a finite dimensional CW complex consisting of one 0--cell, the point $x_0$. Any homomorphism $\phi \colon \pi_1(X,x_0) \to \pi_1(Y,y_0)$ is induced by a map $\tilde{\phi} \colon X \to Y$ where $\tilde{\phi}$ is unique up to homotopy fixing $x_0$.
\end{theorem}

\begin{proof}
    Clearly we must have $\tilde{\phi}(x_0) = y_0$. The 1--skeleton $X^1$ will be a wedge of circles and there is thus a presentation of $\pi_1(X, x_0)$ with each cell $e_\alpha^1$ corresponding to a generator $ [e_\alpha^1] \in \pi_1(X, x_0)$. We can choose $\tilde{\phi}(e_\alpha)$ to trace to a path corresponding to $\phi([e_\alpha^1]) \in \pi_1(Y,y_0)$ for each $e_\alpha^1 \in X^1$.
    
    Let $\psi_\beta \colon S^1 \to X^1$ be an attaching map for a 2--cell $e_\beta^2 \in X$. Let $i \colon X^1 \hookrightarrow X$ be the inclusion. We have that $i_*$ is the surjetcion from the free group generated by each $e_\alpha^1$ to $\pi_1(X,x_0)$. The attaching of the 2--cell $e_\beta^2$ provides a null homotopy for the path traced by $\psi_\beta$. In the presentation of $\pi_1(X, x_0)$ as above, each relation corresponds to the path of a $\psi_\beta$. Thus, $i_*([\psi_\beta]) = 0$ and so $\tilde{\phi}_*([\psi_\beta]) = \phi \circ i_*([\psi_\beta]) = 0$. Thus, $\tilde{\phi} \circ \psi_\beta$ is null homotopic and so can be extended over all of the closure of $e_\beta^2$ by \cref{lem:null_homotopic_extension}. This is an extension of $\tilde{\phi}$ and repeating this allows us to extend $\tilde{\phi}$ over all of $X^2$.
    
    To extend $\tilde{\phi}$ over $e_\gamma^3$ we use that $S^{2}$ is simply connected (as for any $S^n$ with $n\geq 2$) and so for the attaching map $\psi_\gamma \colon S^2 \to X^2$ we have that $\tilde{\phi} \circ \psi_\gamma$ lifts to the universal cover of $Y$, which is contractible since $Y$ is a $K(G,1)$, so $\tilde{\phi} \circ \psi_\gamma$ is null homotopic. This same argument applies for any $e_\delta^n$ for $n\geq 3$. We can thus extend $\tilde{\phi}$ over the 3--cells and proceeding inductively, over all of $X$.

    Now we turn to the uniqueness of $\tilde{\phi}$ up to homotopy. Let $\tilde{\phi}_1$ and $\tilde{\phi}_2$ be any such maps constructed as above. Clearly $\tilde{\phi}_1(x_0) = \tilde{\phi}_2(x_0)$ and $\tilde{\phi}_1|_{X^1} \sim \tilde{\phi}_2|_{X^1}$ by the restrictions of our construction. Let $H$ witness this homotopy. Give $X \times I$ a CW structure where each $e_\alpha^n$ becomes an $e_\alpha^{n+1}$ and the attaching maps are dealt with appropriately. This is achieved with a single 0--cell. Now define $J \colon (X^1 \times I) \cup (X \times \Set {0,1}) \to Y$ such that $J|_{X^1 \times I} = H$ and that $J|_{X \times \Set{0}} = \tilde{\phi}_1$ and $J|_{X \times \Set{1}} = \tilde{\phi}_2$.
\end{proof}

\end{document}