% !TeX root = ../../main.tex
\documentclass[class=article, crop=false]{standalone}
\begin{document}

\section{Poset complex}
\label{sec:poset_cx}

For some edge labelled poset $P$, we can construct a cell complex $K(P)$ from $P$ such that $\pi_1(K(P))$ is $G(P)$. First we make some definitions. An \emph{abstract simplicial complex} is a family of sets that is closed under taking arbitrary subsets.

\begin{definition}
	Given a finite abstract simplicial complex $X$, the \emph{geometric realisation} of that simplicial complex is defined as follows: For each single element set in $X$ assign a point. For each two element set assign an open edge between the two vertices it contains. For each three element set assign an open triangle, the interior of the three edges of its three subsets of size two. In this way, continue constructing simplices of dimension $n$ for each $n+1$ size set in $X$.
	\label{def:geometric_simplicial_cx}
\end{definition}

The set of all chains in a poset $P$ is an abstract simplicial complex. We define $\Delta(P)$ to be the geometric simplicial complex corresponding to the set of all chains in $P$ where each $n$--simplex is an $n$--chain of $P$. Note that as in \cite[Definition 1.7]{mccammond_sulway_artin_2017}, we define an $n$--chain to have $n-1$ elements, e.g.~$(\{1\} \subseteq \{1,2\})$ is a 1--chain.

For example, in \cref{fig:example_edge_labelled_poset}, $\Delta(P)$ would be three 3--simplices all sharing an edge (a 1--simplex) corresponding to the 1--chain $(\emptyset \subseteq \{1,2,3,4\})$. Two of the 3--simplices would share a face corresponding to the 2--chain  $(\emptyset \subseteq \Set{1} \subseteq \Set{1,2,3,4})$. We also assign an orientation on edges in $\Delta(P)$ such that the edge corresponding to the 1--chain $(x \leq y)$ points from $x$ to $y$. For a two--dimensional example, consider the following poset $P$ and corresponding $\Delta(P)$. Here we forget about edge labelling in $P$ for a moment.

\begin{figure}[ht]
\centering
\begin{tikzpicture}[scale=1.5]
	\tikzstyle{every label}=[font=\footnotesize]
	\node[FSC] (base left)    	at (0,0)   				[label=below:{$\{2\}$}]				{};
	\node[FSC] (base right)		at ($(base left) + (1,0)$)  		[label=below:{$\{1\}$}]   			{};							
	\node[FSC] (middle left)	at ($(base left) + (-0.5, 0.866)$)	[label=left:{$\{2,3\}$}]   			{};
	\node[FSC] (middle middle)	at ($(middle left) + (1,0)$)		[label={[label distance=-4]40:{$\{1,2\}$}}]	{};
	\node[FSC] (middle right)	at ($(middle middle) + (1,0)$)		[label=right:{$\{1,4\}$}]			{};
	\node[FSC] (top)		at ($(middle middle) + (0, 0.866)$)	[label=above:{$\{1,2,3,4\}$}]			{};
												
	\draw	(base left)		to		(middle left);
	\draw	(base left)		to 		(middle middle);
	\draw	(base right)		to 		(middle middle);
	\draw 	(base right)		to 		(middle right);
	\draw	(middle left)		to 		(top);
	\draw 	(middle middle)		to 		(top);
	\draw	(middle right)		to 		(top);
\end{tikzpicture}
\hspace{1cm}
\tikzstyle{green polyfill}=[fill=green!20, draw=green!50!black, thick]
\begin{tikzpicture}[scale = 1.5]
	\tikzstyle{every label}=[font=\footnotesize]
	\node[FSC] (base left)    	at (0,0)						[label=below:{$\{2\}$}]				{};
	\node[FSC] (base right)		at ($(base left) + (2,0)$)				[label=below:{$\{1\}$}]   			{};							
	\node[FSC] (middle left)	at ($(base left) + (-0.5, 0.866)$)			[label=left:{$\{2,3\}$}]   			{};
	\node[FSC] (middle middle)	at ($0.5*(base left) + 0.5*(base right) + (0,0.5)$)	[label={[label distance=4pt]270:{$\{1,2\}$}}]	{};
	\node[FSC] (middle right)	at ($(base right) + (0.5,0.866)$)			[label=right:{$\{1,4\}$}]			{};
	\node[FSC] (top)		at ($(middle middle) + (0, 0.866)$)		[label=above:{$\{1,2,3,4\}$}]			{};
	
	\begin{pgfonlayer}{background}
		\filldraw[green polyfill] (base left.center) -- (middle left.center) -- (top.center) -- cycle;
		\filldraw[green polyfill] (base left.center) -- (middle middle.center) -- (top.center) -- cycle;
		\filldraw[green polyfill] (base right.center) -- (middle middle.center) -- (top.center) -- cycle;
		\filldraw[green polyfill] (base right.center) -- (middle right.center) -- (top.center) -- cycle;		
	\end{pgfonlayer}	
\end{tikzpicture}
\caption{An example poset $P$ (left) and corresponding $\Delta(P)$ (right).}
\label{fig:example_poset_with_simplicial_cx}
\end{figure}

We continue, now using an edge labelling on $P$ (\cref{fig:example_edge_labelled_poset_with_KP} (left)).

\begin{figure}[ht]
\centering
\begin{tikzpicture}[scale=1.5]
	\tikzstyle{every label}=[font=\footnotesize]
	\node[FSC] (base left)    	at (0,0)   							[label=below:{$\{2\}$}]   	{};
	\node[FSC] (base right)		at ($(base left) + (1,0)$)  		[label=below:{$\{1\}$}]   	{};							
	\node[FSC] (middle left)	at ($(base left) + (-0.5, 0.866)$)  [label=left:{$\{2,3\}$}]   	{};
	\node[FSC] (middle middle)	at ($(middle left) + (1,0)$)		[label={[label distance=-4]40:{$\{1,2\}$}}]	{};
	\node[FSC] (middle right)	at ($(middle middle) + (1,0)$)		[label=right:{$\{1,4\}$}]	{};
	\node[FSC] (top)			at ($(middle middle) + (0, 0.866)$)	[label=above:{$\{1,2,3,4\}$}]	{};
	
	\tikzstyle{every node}=[font=\footnotesize]
	\draw[a]	(base left) 	to 	node[left] 			{a} 	(middle left);
	\draw[b]	(base left) 	to 	node[right] 		{b}		(middle middle);
	\draw[b]	(base right) 	to 	node[right] 		{b}		(middle middle);
	\draw[a] 	(base right) 	to 	node[right] 		{a}		(middle right);
	\draw[b]	(middle left)	to 	node[left] 			{b} 	(top);
	\draw[a] 	(middle middle)	to 	node[right] 		{a}		(top);
	\draw[b]	(middle right)	to 	node[right]			{b}		(top);
\end{tikzpicture}
\hspace{1.3cm}
\tikzstyle{green polyfill}=[fill=green!20, draw=green!50!black, thick]
\begin{tikzpicture}[scale = 1.5, baseline=-20pt]
	\tikzstyle{every label}=[font=\footnotesize]
	\node[FSC] (base left)    	at (0,0)   							   	{};
	\node[FSC] (base right)		at ($(base left) + (2,0)$)  		   	{};							
	\node[FSC] (middle left)	at ($(base left) + (-0.5, 0.866)$)     	{};
	\node[FSC] (middle middle)	at ($0.5*(base left) + 0.5*(base right) + (0,0.4)$)		{};
	\node[FSC] (middle right)	at ($(base right) + (0.5,0.866)$)			{};
	\node[FSC] (top)			at ($(middle middle) + (0, 1)$)		{};

	
	\begin{scope}[thick]
		\draw[a, arrow_me=stealth]	(base left) 	to	(middle left);
		\draw[b, arrow_me=>>s]		(base left) 	to	(middle middle);
		\draw[b, arrow_me=>>s]		(base right) 	to	(middle middle);
		\draw[a, arrow_me=stealth] 	(base right) 	to	(middle right);
		\draw[b, arrow_me=>>s]		(middle left)	to	(top);
		\draw[a, arrow_me=stealth] 	(middle middle)	to	(top);
		\draw[b, arrow_me=>>s]		(middle right)	to	(top);
		
		\draw[arrow_me=>>>s] (base left) to (top);
		\draw[arrow_me=>>>s] (base right) to (top); 
	\end{scope}	
	
	\node at ($1/3*(base left) + 1/3*(middle left) + 1/3*(top)$) {$\circlearrowright$};
	\node at ($1/3*(base left) + 1/3*(middle middle) + 1/3*(top) + (0.1,0)$) {$\circlearrowright$};
	\node at ($1/3*(base right) + 1/3*(middle middle) + 1/3*(top) + (-0.1,0)$) {$\circlearrowleft$};
	\node at ($1/3*(base right) + 1/3*(middle right) + 1/3*(top)$) {$\circlearrowleft$};
	
	\begin{pgfonlayer}{background}
		\fill[fill=purple!20] (base left.center) -- (middle left.center) -- (top.center) -- cycle;
		\fill[fill=orange!20] (base left.center) -- (middle middle.center) -- (top.center) -- cycle;
		\fill[fill=orange!20] (base right.center) -- (middle middle.center) -- (top.center) -- cycle;
		\fill[fill=purple!20] (base right.center) -- (middle right.center) -- (top.center) -- cycle;
	\end{pgfonlayer}	
\end{tikzpicture}
\caption{The poset in \cref{fig:example_poset_with_simplicial_cx} with edge labelling (left) and the corresponding space $K(P)$ (right).}
\label{fig:example_edge_labelled_poset_with_KP}
\end{figure}

To construct $K(P)$, first we define a labelling on 1--chains in $P$ which extends the edge labelling in $P$. Let $A^*$ denote the set of all words corresponding to the alphabet $A$.
\what[In the last draft I got this completely wrong. The extended labelling only extends to 1--chains. Not $n$--chains. This is important for how we identify $n$--simplices in $K(P)$, which has also changed since the last draft.]

\begin{definition}
	 Given some edge--labelled poset $(P,\leq,l \colon \mathcal{E}(P) \to A)$ and some 1--chain $(x \leq y)$ in $P$, the \emph{extended label} $\mathcal{L}(\sigma) \subseteq A^*$ is the language of all words corresponding to all saturated chains that start at $x$ and end at $y$.
\end{definition}

For example, consider the chain $(\Set{2} \subseteq \Set{1,2,3,4})$ in the context of \cref{fig:example_edge_labelled_poset_with_KP} (left). There are two corresponding saturated chains, $(\Set{2} \subseteq \Set{1,2} \subseteq \Set{1,2,3,4})$ and $(\Set{2} \subseteq \Set{2,3} \subseteq \Set{1,2,3,4})$, which respectively correspond to the words $ba$ and $ab$. So $\mathcal{L}(\Set{2} \subseteq \Set{1,2,3,4}) = \Set{ba, ab}$. Here are some illustrative examples:

\begin{itemize}
	\item $\mathcal{L}(\{1\} \subseteq \{1,2\}) = \mathcal{L}(\{2\} \subseteq \{1,2\}) = \{b\}$.
	\item $\mathcal{L}(\{1\} \subseteq \{1,2,3,4\}) = \mathcal{L}(\{2\} \subseteq \{1,2,3,4\}) = \{ba, ab\}$.
	\item $\mathcal{L}(\{1\}) = \mathcal{L}(\{2\}) = \mathcal{L}(\{1,2\}) = \cdots = \emptyset$.
\end{itemize}

We use this extended labelling to form a quotient space of $\Delta(P)$. Recall the orientation of the edge $(x \leq y)$ is from $x$ to $y$. Consider two closed $n$--simplices $e^n_\alpha$ and $e^n_\beta$ in $\Delta(P)$. We form a relation $\sim$ on $n$--simplices such that $e^n_\alpha \sim e^n_\beta$ iff there exists a continuous edge--orientation and extended label preserving map $f \colon e^n_\alpha \to e^n_\beta$.


\begin{definition}[{Poset Complex \cite[Definition 1.6]{mccammond_introduction_2005}}]
	Given some finite height edge--labelled poset $P$, the poset complex $K(P)$ is the quotient space $\Delta(P)/\approx$ where $\approx$ identifies all $n$--simplices (for any $n$) using the map $f \colon e^n_\alpha \to e^n_\beta$ corresponding to $\sim$ defined above.
	\label{def:poset_complex}
\end{definition}

In the example in \cref{fig:example_edge_labelled_poset_with_KP}, three red edges are identified, four blue edges are identified, two black edges are identified, two orange triangles are identified and two purple triangles are identified. The orientation of the triangles is indicated by a $\acts$ symbol.

We see that this space is homeomorphic to a torus, which has fundamental group $\mathbb{Z}^2 \cong \left\langle a,b \mid ab = ba \right\rangle$, which is also the $G(P)$ for this edge--labelled poset. This fact holds in general, which we now begin to prove.

\begin{lemma}
	Given an edge labelled poset $(P, \, \leq, \, l \colon \mathcal{E}(P) \to A)$, the group $\pi_1(K(P))$ is generated by a set of loops in bijection with $\image(l)$.
	\label{lem:fund_group_poset_complex_gen_labels}
\end{lemma}
\begin{proof}
	Each vertex $p \in \Delta(P)$ has extended label $\emptyset$ and so there is only one vertex in $K(P)$, denote this point $p_0$. The 1--skeleton of $K(P)$ will be a wedge of circles, one for each extended label in $\Set{\mathcal{L}(C) \given C \text{ is a 1--chain}}$. The following is the setup of our proof.

	\begin{enumerate}
		\item Let $\Sigma$ denote 1--chains in $P$.
		\item Let $\Omega$ denote paths between vertices in $\Delta(P)$ along 1--simplices in $\Delta(P)$ such that the direction along the edge corresponding to $(x \leq y)$ is from $x$ to $y$.
		\item Let $\Lambda$ denote loops along 1--simplices in $K(P)$ that start and end at $p_0$. If $\lambda \in \Lambda$, then $[\lambda] \in \pi_1(K(P),p_0)$.  
	\end{enumerate}
	
	Let $\alpha$, be the map from 1--chains to corresponding paths in $\Delta(P)$ and $\beta \colon \Delta(P) \to K(P)$ be the quotient map as in the definition of $K(P)$. We have the following diagram.

	\begin{equation*}
		\begin{tikzcd}
			&\Sigma \ar[r, "\alpha"] &\Omega \ar[r, "\beta"] &\Lambda
		\end{tikzcd}
	\end{equation*}
	
	By our remarks on the 1--skeleton of $K(P)$, we see that $\pi_1(K(P), p_0)$ is generated by the homotopy classes of $\image(\beta \circ \alpha)$. Let $\sigma = (x < y) \in \Sigma$ be a 1--chain and let $\lambda$ be such that $\beta ( \alpha (\sigma)) = \lambda$. Let concatenation of loops $\lambda$ and $\lambda^\prime$ be denoted $\lambda\lambda^\prime$ such that $[\lambda][\lambda^\prime] = [\lambda\lambda^\prime]$. We will show that $\lambda$ has a factorisation where one of the factors is $\beta(\alpha(x \lessdot z))$ for some $z \in P$.


	There exists $z \in P$ such that $x \lessdot z \leq y$. If $z \neq y$, there is a 2--simplex in $\Delta(P)$ corresponding to the 2--chain $(x \lessdot z < y)$. One of the edges of this 2--simplex corresponds to the 1--chain $(x\lessdot z)$.
	The path $\alpha(x \leq y)$ is homotopic (through the 2--simplex $(x \lessdot z < y)$) to the path $\alpha(x \lessdot z)\alpha(z < y)$. Let $H$ witness this homotopy. We have that $\beta \circ H$ is well--defined (and continuous), so the loop $\lambda$ is homotopic to the loop $\beta(\alpha(x \lessdot z)\alpha(z < y)) = \beta(\alpha(x \lessdot z))\beta(\alpha(z < y))$. We repeat the process replacing $\sigma$ with $(z<y)$. Eventually this must stop since our poset is of finite height. After this, we achieve a factorisation of $\lambda$, entirely in factors of the form $\beta(\alpha(r\lessdot s))$.
	
	Consider $\mathcal{E}(P)$ as a subset of all 1--chains in $P$. We see that $\pi_1(K(P),p_0)$ is generated by the homotopy classes of loops in $\image(\beta \circ \alpha|_{\mathcal{E}(P)})$. Each covering relation $(r\lessdot s)$ has extended label $\Set{l(r\lessdot s)}$, therefore $\image(\beta \circ \alpha |_{\mathcal{E}(P)})$ is in bijection with $\image(l)$.
\end{proof}

\begin{lemma}
	For an edge-labelled poset $P$, there exists a surjective homomorphism\newline$\phi \colon G(P) \to \pi_1(K(P),p_0)$ where $p_0$ is as in the previous lemma.
	\label{lem:exists_surj_hom_onto_fund_group}
\end{lemma}
\begin{proof}
	Let us follow from the notation in the proof of \cref{lem:fund_group_poset_complex_gen_labels} and let $\theta \colon \image(l) \to \pi_1(K(P), p_0)$ be the bijection at the end of that proof such that we have $\GroupPres{\,\image(\theta \circ l)\,} = \pi_1(K(P),p_0)$.

	Let $\sigma = (x_1 \lessdot \cdots \lessdot x_i)$ and $\sigma^\prime = (x^\prime_1 \lessdot \cdots \lessdot x^\prime_j)$ be two saturated chains such that $x_1 = x^\prime_1$ and $x_i = x^\prime_j$ with corresponding words $w = w_1\cdots w_{i-1}$, and $w^\prime = w^\prime_1\cdots w^\prime_{j-1}$ in $A^*$ such that $l(x_k \lessdot x_{k+1}) = w_k$ and similarly for $\sigma^\prime$. Recall that $w$ and $w^\prime$ are words that are identified by the relations in the defining presentation for $G(P)$. We want to show there exists a homotopy between the loop $\theta(w_1)\cdots\theta(w_i)$ and the loop $\theta(w^\prime_1)\cdots\theta(w^\prime_j)$.
	
	By doing the process in the proof of \cref{lem:fund_group_poset_complex_gen_labels} in reverse, we get a homotopy between  $\theta(w_1)\cdots\theta(w_i)$ and $\beta(\alpha(x_1 \leq x_i))$ through the two skeleton of $K(P)$. By the same argument, we get a homotopy between $\theta(w^\prime_1)\cdots\theta(w^\prime_j)$ and $\beta(\alpha(x^\prime_1 \leq x^\prime_j))$. Since $x_1 = x^\prime_1$ and $x_i = x^\prime_j$, we have $\theta(w_1)\cdots\theta(w_i) \sim \theta(w^\prime_1)\cdots\theta(w^\prime_j)$.
	
	We have shown that $\pi_1(K(P), p_0)$ has all necessary relations to extend $\theta$ to a surjective homomorphism $\phi \colon G(P) \to \pi_1(K(P), p_0)$. 
\end{proof}

To finish our proof, it is necessary to provide an inverse to $\phi$, which we now complete

\begin{theorem}
	Given an edge labelled poset $(P, \, \leq, \, l \colon \mathcal{E}(P) \to A)$, we have $\pi_1(K(P)) \cong G(P)$.
	\label{thm:fund_group_poset_complex_poset_group}
\end{theorem}
\begin{proof}
	We again follow the notation from \cref{lem:fund_group_poset_complex_gen_labels}.
	In the proof of \cref{lem:fund_group_poset_complex_gen_labels} we remarked that $\pi_1(K(P),p_0)$ is generated by a set of loops in bijection with 
	\begin{equation*}
		\mathcal{L}(\Sigma) \coloneq \Set{\mathcal{L}(C) \given C \text{ is a 1--chain}}.
	\end{equation*}
Let $\chi \colon \mathcal{L}(\Sigma) \to \pi_1(K(P), p_0)$ denote this bijection. We now think of this presentation abstractly, and work to give a set of relations $R$ such that we obtain a group $\GroupPres{\mathcal{L}(\Sigma) \relations R\,} \cong \pi_1(K(P),p_0)$ where the isomorphism is an extension of $\chi$.

By the structure of the 1--skeleton of $K(P)$ remarked in the proof of \cref{lem:fund_group_poset_complex_gen_labels}, there is a set of relations $R$ in bijection with the set of 2--simplices in $K(P)$. Let $e^2$ be a 2--simplex with edges $e^1_i$ for $i \in \Set{1,2,3}$. We have that the edges $e^1_i$ are oriented acyclically as in \cref{fig:example_simplex_in_KP}.

\begin{figure}[h]
	\centering
	\begin{tikzpicture}[scale=1.5]
		\node[FSC] (base) {};
		\node[FSC] (mid) at ($(base) + (0,1)$) {};
		\node[FSC] (top) at ($(base) + (0.8,1.5)$) {};

		\draw[arrow_me=stealth] (base) to node[auto] {$e^1_1$} (mid);
		\draw[arrow_me=stealth] (mid) to node[auto] {$e^1_2$} (top);
		\draw[arrow_me=stealth] (base) to node[auto, swap] {$e^1_3$} (top);
	\end{tikzpicture}
	\caption{The orientation of all simplices in $\Delta(P)$ and $K(P)$.}
	\label{fig:example_simplex_in_KP}
\end{figure}

All simplices are oriented this way because if $x \leq y$ and $y \leq z$ then $x \leq z$.
Let the attaching map for each $e^1_i$ trace the loop $\lambda_i \in \Lambda$ and choose extended labels $\ell_i$ such that $\chi(\ell_i) = \lambda_i$. We see that the 2--simplex $e^2$ corresponds to the relation $\ell_1\ell_2\ell_3^{-1} = 0$. Here multiplication of extended labels is formal. As are inverses, denoted $\ell^{-1}$. Define $R$ to be the following set of relations for $\GroupPres{\, \mathcal{L}(\Sigma)\,}$.

\begin{equation*}
	R \coloneq \Set*{\ell_1\ell_2\ell_3^{-1} = 0 \given \exists \text{ 2--chain } x< y < z \text{ where} \quad
	\begin{aligned}
		&\mathcal{L}(x < y) = \ell_1 \\
		&\mathcal{L}(y < z) = \ell_2 \\
		&\mathcal{L}(x < z) = \ell_3
	\end{aligned}}
\end{equation*}

Following from the definition of $K(P)$, these are exactly the relations we need to extend $\chi$ to an isomorphism witnessing $\GroupPres{ \mathcal{L}(\Sigma) \relations R \,} \cong \pi_1(K(P),p_0)$.

We now make some geometric observations about $R$. To each relation $r \in R$, there is a corresponding 2--simplex $e^2(r) \subseteq \Delta(P)$. Let $\omega_i \in \Omega$ be the paths going along the edges $e^1_i$ in either cyclic direction around $e^2(r)$ and choose $\ell_i$ such that $\chi(\ell_i) = \beta \circ \omega_i \in \pi_1(K(P),p_0)$. The two sides of any equation resulting from the relation $r$ correspond to two paths along the edges of $e^2(r)$ that start and end at the same point e.g.~$\ell_1\ell_2 = \ell_3, \, \ell_1 = \ell_3\ell_2^{-1}, \, \ell_1\ell_2\ell_3^{-1} = 0$, where 0 corresponds to the trivial path.

This also holds for repeated application of any number of relations in $R$. Suppose we are given some $x_i,y_i \in \mathcal{L}(\Sigma)$ such that we have the equation $x_1x_2 \cdots x_m  = y_1y_2 \cdots y_n$ in $\GroupPres{ \mathcal{L}(\Sigma) \relations R \,}$. We see there must be two edge paths $s = s_1,s_2,\ldots,s_{m-1}$ and $t = t_1,t_2,\ldots,t_{n-1}$ where $s_i,t_i$ are vertices in $\Delta(P)$ such that $\mathcal{L}(s_i\leq s_{i+1}) = x_i$ and $\mathcal{L}(t_i\leq t_{i+1}) = y_i$ with $s_1 = t_1$ and $s_m = t_n$.

Recall that $\pi_1(K(P),p_0)$ is generated by loops corresponding to covering relations, so the subgroup $\GroupPres{ \mathcal{L}(\mathcal{E}(P))}$ is all of $\GroupPres{\, \mathcal{L}(\Sigma) \relations R \,}$. Now consider an equation as above where all the $x_i$ and $y_i$ are single set labels $\Set{a} \in \mathcal{L}(\mathcal{E}(P))$ for some $a \in A$. The paths $s$ and $t$ must consist of covering relations, i.e. $s_i \lessdot s_{i+1}$ and $t_i \lessdot t_{i+1}$ for all $i$. Edges in $\Delta(P)$ of the form $(r \lessdot s)$ correspond to edges in the Hasse diagram of $P$. So such an edge path in $\Delta(P)$ where all edges are covering relations corresponds to a path in the Hasse diagram of $P$. Therefore, any equation relating formal words in $\mathcal{L}(\mathcal{E}(P))$ corresponds to two paths in the Hasse diagram of $P$ that start and end in the same point.

Let $\psi \colon \mathcal{L}(\mathcal{E}(P)) \to G(P)$ act as $\psi(\Set{a}) = a$. By the previous remarks, we can extend $\psi$ to a homomorphism $\psi \colon \GroupPres{\,\mathcal{L}(\mathcal{E}(P))\,} \to K(P)$ and $\psi \circ \chi^{-1}$ is inverse to $\phi$ from \cref{lem:exists_surj_hom_onto_fund_group}.
\end{proof}

Note that our proof of the above was ambivalent to the identification of $n$--simplices for $n\geq2$. Indeed, these identifications do not affect the fundamental group, but they do ensure that higher homotopy groups are trivial. We can see if we did not identify the 2--simplices in \cref{fig:example_edge_labelled_poset_with_KP}, $\pi_2(K(P))$ would be non-trivial.

\end{document}
