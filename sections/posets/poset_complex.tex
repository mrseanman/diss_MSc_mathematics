% !TeX root = ../../main.tex
\documentclass[class=article, crop=false]{standalone}
\begin{document}

\section{Poset complex}
\label{sec:poset_cx}

Given some edge labelled poset $P$, we can construct a cell complex $K(P)$ from $P$ such that $\pi_1(K(P))$ is $G(P)$. First we make some definitions. An \emph{abstract simplicial complex} is a family of sets that is closed under taking arbitrary subsets.

\begin{definition}
	Given a finite abstract simplicial complex $X$, the \emph{geometric realisation} of that simplicial complex is defined as follows. For each single element set in $X$ assign a point. For each two element set assign an open 1--simplex between the two vertices it contains. For each three element set assign an open 2--simplex, which is the interior of the three 1--simplices corresponding to its three subsets of size two. In this way, continue constructing simplices of dimension $n$ for each $n+1$ size set in $X$.
	\label{def:geometric_simplicial_cx}
\end{definition}

The set of all chains in a poset $P$ is an abstract simplicial complex. We define $\Delta(P)$ to be the geometric simplicial complex corresponding to the set of all chains in $P$ where each $n$--simplex is an $n$--chain of $P$. Note that as in \cite[Definition 1.7]{mccammond_sulway_artin_2017}, we define an $n$--chain to have $n-1$ elements, e.g.~$(\{1\} \subseteq \{1,2\})$ is a 1--chain.

For example, in \cref{fig:example_edge_labelled_poset}, $\Delta(P)$ would be three 3--simplices all sharing an edge (a 1--simplex) corresponding to the 1--chain $(\emptyset \subseteq \{1,2,3,4\})$. Two of the 3--simplices would share a face corresponding to the 2--chain  $(\emptyset \subseteq \Set{1} \subseteq \Set{1,2,3,4})$.

We also assign an orientation on edges in $\Delta(P)$ such that the edge corresponding to the 1--chain $(x \leq y)$ points from $x$ to $y$. For a two--dimensional example, consider the following poset $P$ and corresponding $\Delta(P)$. Here we forget about edge labelling in $P$ for a moment.

\begin{figure}[ht]
\centering
\begin{tikzpicture}[scale=1.5]
	\tikzstyle{every label}=[font=\footnotesize]
	\node[FSC] (base left)    	at (0,0)   				[label=below:{$\{2\}$}]				{};
	\node[FSC] (base right)		at ($(base left) + (1,0)$)  		[label=below:{$\{1\}$}]   			{};							
	\node[FSC] (middle left)	at ($(base left) + (-0.5, 0.866)$)	[label=left:{$\{2,3\}$}]   			{};
	\node[FSC] (middle middle)	at ($(middle left) + (1,0)$)		[label={[label distance=-4]40:{$\{1,2\}$}}]	{};
	\node[FSC] (middle right)	at ($(middle middle) + (1,0)$)		[label=right:{$\{1,4\}$}]			{};
	\node[FSC] (top)		at ($(middle middle) + (0, 0.866)$)	[label=above:{$\{1,2,3,4\}$}]			{};
												
	\draw	(base left)		to		(middle left);
	\draw	(base left)		to 		(middle middle);
	\draw	(base right)		to 		(middle middle);
	\draw 	(base right)		to 		(middle right);
	\draw	(middle left)		to 		(top);
	\draw 	(middle middle)		to 		(top);
	\draw	(middle right)		to 		(top);
\end{tikzpicture}
\hspace{1cm}
\tikzstyle{green polyfill}=[fill=green!20, draw=green!50!black, thick]
\begin{tikzpicture}[scale = 1.5]
	\tikzstyle{every label}=[font=\footnotesize]
	\node[FSC] (base left)    	at (0,0)						[label=below:{$\{2\}$}]				{};
	\node[FSC] (base right)		at ($(base left) + (2,0)$)				[label=below:{$\{1\}$}]   			{};							
	\node[FSC] (middle left)	at ($(base left) + (-0.5, 0.866)$)			[label=left:{$\{2,3\}$}]   			{};
	\node[FSC] (middle middle)	at ($0.5*(base left) + 0.5*(base right) + (0,0.5)$)	[label={[label distance=4pt]270:{$\{1,2\}$}}]	{};
	\node[FSC] (middle right)	at ($(base right) + (0.5,0.866)$)			[label=right:{$\{1,4\}$}]			{};
	\node[FSC] (top)		at ($(middle middle) + (0, 0.866)$)		[label=above:{$\{1,2,3,4\}$}]			{};
	
	\begin{pgfonlayer}{background}
		\filldraw[green polyfill] (base left.center) -- (middle left.center) -- (top.center) -- cycle;
		\filldraw[green polyfill] (base left.center) -- (middle middle.center) -- (top.center) -- cycle;
		\filldraw[green polyfill] (base right.center) -- (middle middle.center) -- (top.center) -- cycle;
		\filldraw[green polyfill] (base right.center) -- (middle right.center) -- (top.center) -- cycle;		
	\end{pgfonlayer}	
\end{tikzpicture}
\caption{An example poset $P$ (left) and corresponding $\Delta(P)$ (right).}
\label{fig:example_poset_with_simplicial_cx}
\end{figure}

We continue, now using an edge labelling on $P$ (\cref{fig:example_edge_labelled_poset_with_KP} (left)).

\begin{figure}[ht]
\centering
\begin{tikzpicture}[scale=1.5]
	\tikzstyle{every label}=[font=\footnotesize]
	\node[FSC] (base left)    	at (0,0)   							[label=below:{$\{2\}$}]   	{};
	\node[FSC] (base right)		at ($(base left) + (1,0)$)  		[label=below:{$\{1\}$}]   	{};							
	\node[FSC] (middle left)	at ($(base left) + (-0.5, 0.866)$)  [label=left:{$\{2,3\}$}]   	{};
	\node[FSC] (middle middle)	at ($(middle left) + (1,0)$)		[label={[label distance=-4]40:{$\{1,2\}$}}]	{};
	\node[FSC] (middle right)	at ($(middle middle) + (1,0)$)		[label=right:{$\{1,4\}$}]	{};
	\node[FSC] (top)			at ($(middle middle) + (0, 0.866)$)	[label=above:{$\{1,2,3,4\}$}]	{};
	
	\tikzstyle{every node}=[font=\footnotesize]
	\draw[a]	(base left) 	to 	node[left] 			{a} 	(middle left);
	\draw[b]	(base left) 	to 	node[right] 		{b}		(middle middle);
	\draw[b]	(base right) 	to 	node[right] 		{b}		(middle middle);
	\draw[a] 	(base right) 	to 	node[right] 		{a}		(middle right);
	\draw[b]	(middle left)	to 	node[left] 			{b} 	(top);
	\draw[a] 	(middle middle)	to 	node[right] 		{a}		(top);
	\draw[b]	(middle right)	to 	node[right]			{b}		(top);
\end{tikzpicture}
\hspace{1.3cm}
\tikzstyle{green polyfill}=[fill=green!20, draw=green!50!black, thick]
\begin{tikzpicture}[scale = 1.5, baseline=-20pt]
	\tikzstyle{every label}=[font=\footnotesize]
	\node[FSC] (base left)    	at (0,0)   							   	{};
	\node[FSC] (base right)		at ($(base left) + (2,0)$)  		   	{};							
	\node[FSC] (middle left)	at ($(base left) + (-0.5, 0.866)$)     	{};
	\node[FSC] (middle middle)	at ($0.5*(base left) + 0.5*(base right) + (0,0.4)$)		{};
	\node[FSC] (middle right)	at ($(base right) + (0.5,0.866)$)			{};
	\node[FSC] (top)			at ($(middle middle) + (0, 1)$)		{};

	
	\begin{scope}[thick]
		\draw[a, arrow_me=stealth]	(base left) 	to	(middle left);
		\draw[b, arrow_me=>>s]		(base left) 	to	(middle middle);
		\draw[b, arrow_me=>>s]		(base right) 	to	(middle middle);
		\draw[a, arrow_me=stealth] 	(base right) 	to	(middle right);
		\draw[b, arrow_me=>>s]		(middle left)	to	(top);
		\draw[a, arrow_me=stealth] 	(middle middle)	to	(top);
		\draw[b, arrow_me=>>s]		(middle right)	to	(top);
		
		\draw[arrow_me=>>>s] (base left) to (top);
		\draw[arrow_me=>>>s] (base right) to (top); 
	\end{scope}	
	
	\node at ($1/3*(base left) + 1/3*(middle left) + 1/3*(top)$) {$\circlearrowright$};
	\node at ($1/3*(base left) + 1/3*(middle middle) + 1/3*(top) + (0.1,0)$) {$\circlearrowright$};
	\node at ($1/3*(base right) + 1/3*(middle middle) + 1/3*(top) + (-0.1,0)$) {$\circlearrowleft$};
	\node at ($1/3*(base right) + 1/3*(middle right) + 1/3*(top)$) {$\circlearrowleft$};
	
	\begin{pgfonlayer}{background}
		\fill[fill=purple!20] (base left.center) -- (middle left.center) -- (top.center) -- cycle;
		\fill[fill=orange!20] (base left.center) -- (middle middle.center) -- (top.center) -- cycle;
		\fill[fill=orange!20] (base right.center) -- (middle middle.center) -- (top.center) -- cycle;
		\fill[fill=purple!20] (base right.center) -- (middle right.center) -- (top.center) -- cycle;
	\end{pgfonlayer}	
\end{tikzpicture}
\caption{The poset in \cref{fig:example_poset_with_simplicial_cx} with edge labelling (left) and the corresponding space $K(P)$ (right).}
\label{fig:example_edge_labelled_poset_with_KP}
\end{figure}

To construct $K(P)$, first we define a labelling on 1--chains in $P$ (which are edges in $\Delta(P)$) which extends the edge labelling $l$.
% \what[In the last draft I got this completely wrong. The extended labelling only extends to 1--chains. Not $n$--chains. This is important for how we identify $n$--simplices in $K(P)$, which has also changed since the last draft.]

\begin{definition}
	 Given some edge--labelled poset $(P,\leq,l \colon \mathcal{E}(P) \to A)$ and some 1--chain $\sigma = (x \leq y)$ in $P$, the \emph{extended label} $\mathcal{L}(\sigma) \subseteq A^*$ is the language of all words corresponding to all saturated chains that start at $x$ and end at $y$. For any element $p \in P$ (a 0--chain), define $\mathcal{L}(p)$ to be $\emptyset$. 
\end{definition}

For example, consider the chain $(\Set{2} \subseteq \Set{1,2,3,4})$ in the context of \cref{fig:example_edge_labelled_poset_with_KP} (left). There are two corresponding saturated chains, $(\Set{2} \subseteq \Set{1,2} \subseteq \Set{1,2,3,4})$ and $(\Set{2} \subseteq \Set{2,3} \subseteq \Set{1,2,3,4})$, which respectively correspond to the words $ba$ and $ab$. So $\mathcal{L}(\Set{2} \subseteq \Set{1,2,3,4}) = \Set{ba, ab}$. Here are some illustrative examples:

\begin{itemize}
	\item $\mathcal{L}(\{1\} \subseteq \{1,2\}) = \mathcal{L}(\{2\} \subseteq \{1,2\}) = \{b\}$.
	\item $\mathcal{L}(\{1\} \subseteq \{1,2,3,4\}) = \mathcal{L}(\{2\} \subseteq \{1,2,3,4\}) = \{ba, ab\}$.
	\item $\mathcal{L}(\{1\}) = \mathcal{L}(\{2\}) = \mathcal{L}(\{1,2\}) = \cdots = \emptyset$.
\end{itemize}

We use this extended labelling to form a quotient space of $\Delta(P)$. Recall the orientation of the edge $(x \leq y)$ is from $x$ to $y$.

\begin{definition}[{Poset Complex \cite[Definition 1.6]{mccammond_introduction_2005}}]
	Define the following relation $\sim$ on $\Delta(P)$. For any $n$, consider two $n$--simplices $e^n_\alpha$ and $e^n_\beta$. Denote the closure of these simplices $a$ and $b$ respectively, i.e.~$a$ and $b$ are the open simplices along with all edges (with extended labels) and faces. If there exists an edge--orientation and extended label preserving homeomorphism $f \colon a \to b$, then we identify the points of $e^n_\alpha \subseteq a$ and $e^n_\beta \subseteq b$ under $\sim$ using the map $f$, i.e.~$a\sim f(a)$. Increasing $n$ from 0, we list every pair of $n$--simplices and identify pairs of points under $\sim$ as in the above construction, if such an $f$ exists. Finally, we take the transitive closure of $\sim$ once this construction is complete. Define the \emph{poset complex} $K(P)$ to be the quotient space $\Delta(P)/\sim$.
	\label{def:poset_complex}
\end{definition}

In the example in \cref{fig:example_edge_labelled_poset_with_KP}, three red edges are identified, four blue edges are identified, two black edges are identified, two orange triangles are identified and two purple triangles are identified. The orientation of the triangles is indicated by a $\acts$ symbol.

We see that this space is homeomorphic to a torus, which has fundamental group $\mathbb{Z}^2 \cong \left\langle a,b \mid ab = ba \right\rangle$, which is also $G(P)$ for this edge--labelled poset. This fact holds in general, which we now begin to prove.

Here, and in several other places in this work, we will compute the fundamental group of a CW--complex with only one 0--cell. Given such a CW--complex $X$, denote its set of 1--cells $\Set{e^1_\alpha}_{\alpha \in A}$ and set of 2--cells $\Set{e^2_\beta}_{\beta \in B}$. For each $\beta \in B$, let $\psi_b \colon S^1 \to X^1$ denote the attaching map for the 2--cell $e^2_\beta$. For each $\alpha \in A$, let $\lambda_\alpha$ denote a loop (in any orientation) going around the 1--cell $e^1_\alpha$. The fundamental group of $X$ will have the following useful presentation using only this data \cite[Proposition 1.26]{hatcher_algebraic_2001}.
\begin{equation}
	\pi_1(X) = \GroupPres{\Set{[\lambda_\alpha]}_{\alpha \in A} \relations \Set{[\psi_\beta]}_{\beta \in B}}
	\label{eqn:CW_fund_group_presentation}
\end{equation}
Each relation $[\psi_\beta]$ should be thought of as the equation $[\psi_\beta] = 1$ where the left hand side is a factorisation of $[\psi_\beta]$ in the set $\Set{[\lambda_\alpha]}_{\alpha \in A}$, which is always possible. Accepting that the fundamental group of $X^1$ is the free group generated by $\Set{[\lambda_\alpha]}_{\alpha \in A}$, we get this result by repeated use of Van--Kampen's theorem, inductively attaching each 2--cell in $\Set{e^2_\beta}_{\beta \in B}$. An immediate corollary of this is that the fundamental group of such a CW--complex depends only on its 2--skeleton. This generalises to all CW--complexes and for higher homotopy groups \cite[Corollary 4.12]{hatcher_algebraic_2001}.

The following is a basic group theoretic fact, but it will be useful to explicitly state and prove it. For some set $S$, let $F_S$ be the free group generated by $S$.
\begin{lemma}
	Suppose we have the groups $G_1$ and $G_2$ such that $G_1$ has a presentation $G_1 \cong \GroupPres{S \relations R}$ where $R$ is a set of words in $S$ that are the identity in $G_1$. Further suppose we are given some map $f \colon S \to G_2$. If for each word $s_1s_2 \cdots s_n \in R$ we have that $f(s_1)f(s_1)\cdots f(s_n) = 1$ in $G_2$ then we can extend $f$ to a homomorphism  $h \colon G_1 \to G_2$ with $h|_{S} = f$.
	\label{lem:extend_map_to_homomorphism}
\end{lemma}
\begin{proof}
	We can extend $f$ to a homomorphism $f^\prime \colon F_{S_1} \to G_2$ recursively by setting $f^\prime(1) =1$ and then setting $f^\prime(s_1s_2 \cdots s_n) = f(s_1)f^\prime(s_2\cdots s_n)$ for all words $s_1s_2\cdots s_n$. Let $N_1$ be the minimal normal subgroup in $F_{S_1}$ which contains the elements of $R_1$ such that $G_1 \cong F_{S_1}/N_1$. By assumption, for each $r \in R_1$ we have $f^\prime(r) = 1$. So $R_1 \subseteq \ker(f^\prime)$. So $N_1 \subseteq \ker(f^\prime)$. By the universal property of the quotient we have the following commutative diagram where the unique map $h$ is the required homomorphism.
	\begin{equation*}
		\begin{tikzcd}
			F_{S_1} \ar[r, "f^\prime"] \ar[d] &G_2 \\
			G_1 \cong F_{S_1}/N_1 \ar[ru, "\exists ! h"']
		\end{tikzcd}
	\end{equation*}
\end{proof}
 Note that if $f$ is a surjection then so is $h$.
\begin{lemma}
	Given an edge labelled poset $(P, \, \leq, \, l \colon \mathcal{E}(P) \to A)$, the group $\pi_1(K(P))$ is generated by a set of loops in bijection with $A$.
	\label{lem:fund_group_poset_complex_gen_labels}
\end{lemma}
\begin{proof}
	Each 0--simplex $p \in \Delta(P)$ has extended label $\emptyset$ and so there is only one 0--simplex in $K(P)$, denote this point $p_0$. The 1--skeleton of $K(P)$ will be a wedge of circles, one for each extended label in $\Set{\mathcal{L}(C) \given C \text{ is a 1--chain}}$. The fundamental group will have a presentation as in \eqref{eqn:CW_fund_group_presentation}. The following is the setup of our proof.

	\begin{enumerate}
		\item Let $\Sigma$ denote the set of all 1--chains in $P$.
		\item Let $\Omega$ denote the set of paths between 0--simplices of $\Delta(P)$ along a single 1--simplex of $\Delta(P)$ such that the direction along the 1--simplex corresponding to $(x \leq y)$ is from $x$ to $y$.
		\item Let $\Lambda$ denote the set of loops along 1--simplices in $K(P)$ that start and end at $p_0$. If $\lambda \in \Lambda$, then $[\lambda] \in \pi_1(K(P),p_0)$.  
	\end{enumerate}
	
	Let $\alpha$, be the map from 1--chains to corresponding paths in $\Delta(P)$ and $\beta \colon \Delta(P) \to K(P)$ be the quotient map as in the definition of $K(P)$. Furthermore, let $\beta_*$ be the push forward map on paths. We have the following diagram.

	\begin{equation*}
		\begin{tikzcd}
			&\Sigma \ar[r, "\alpha"] &\Omega \ar[r, "\beta_*"] &\Lambda
		\end{tikzcd}
	\end{equation*}
	
	By our remarks on the 1--skeleton of $K(P)$, we see that $\pi_1(K(P), p_0)$ is generated by the homotopy classes of $\image(\beta_* \circ \alpha)$. Let $\sigma = (x < y) \in \Sigma$ be a 1--chain and let $\lambda$ be such that $\beta_* \circ \alpha (\sigma) = \lambda$. Let concatenation of loops $\lambda_1$ and $\lambda_2$ be denoted $\lambda_1\lambda_2$ such that $[\lambda_1][\lambda_2] = [\lambda_1\lambda_2]$. We similarly define concatenation of paths in $\Omega$. We will show that $[\lambda]$ has a factorisation where one of the factors is $[\beta_* \circ \alpha(x \lessdot z)]$ for some $z \in P$.

	There exists $z \in P$ such that $x \lessdot z \leq y$. If $z \neq y$, there is a 2--simplex in $\Delta(P)$ corresponding to the 2--chain $(x \lessdot z < y)$. One of the edges of this 2--simplex corresponds to the 1--chain $(x\lessdot z)$.
	The path $\alpha(x \leq y)$ is homotopic (through the 2--simplex $(x \lessdot z < y)$) to the path $\alpha(x \lessdot z)\alpha(z < y)$. Let $H$ witness this homotopy such that $\beta \circ H$ factors through $I/(0 \sim 1) \times I \cong S^1 \times I$ as in the following diagram.
	\begin{equation*}
		\begin{tikzcd}
			I \times I \ar[r, "H"] &\Delta(P) \ar[r, "\beta"] &K(P) \\
			S^1 \times I \ar[from = u] \ar[rru, "\overline{H}"']
		\end{tikzcd}
	\end{equation*}
	The map $\overline{H}$ witnesses the following homotopy of loops.
	\begin{equation*}
		\lambda = \beta_*(\alpha(\sigma)) \sim \beta_* (\alpha(x \lessdot z)\alpha(z < y)) = (\beta_* \circ \alpha(x \lessdot z))(\beta_* \circ \alpha(z<y))
	\end{equation*}
	So we have $[\lambda] = [\beta_* \circ \alpha(x \lessdot z)][\beta_* \circ \alpha(z < y)]$.
	We then repeat the process, replacing $\sigma$ with $(z<y)$. Eventually this must stop since our poset is of finite height. After this, we achieve a factorisation of $[\lambda]$, entirely in factors of the form $[\beta_* \circ \alpha(r\lessdot s)]$.
	
	Consider $\mathcal{E}(P)$ as a subset of $\Sigma$. We see that $\pi_1(K(P),p_0)$ is generated by the homotopy classes of loops in $\image(\beta_* \circ \alpha|_{\mathcal{E}(P)})$. Each covering relation $(r\lessdot s)$ has extended label $\Set{l(r\lessdot s)}$. For each $a \in A$ there is exactly one edge in $K(P)$ with label $\Set{a}$. For each $a$, let $\sigma_a = (x \lessdot y) \in \mathcal{E}(P)$ be a 1--chain such that $\mathcal{L}(\sigma_a) = \Set{a}$. We have that $\theta \colon A  \to \pi_1(K(P), p_0)$ defined as $\theta(a) = [\beta_* \circ \alpha(\sigma_a)]$ is our required bijection. Note that $\theta$ does not depend on the map $a \mapsto \sigma_a$ (for which there is a choice to be made).
\end{proof}

\begin{lemma}
	For an edge-labelled poset $P$, there exists a surjective homomorphism\newline$\phi \colon G(P) \to \pi_1(K(P),p_0)$ where $p_0$ is as in the previous lemma.
	\label{lem:exists_surj_hom_onto_fund_group}
\end{lemma}
\begin{proof}
	Let us follow from the notation in the proof of \cref{lem:fund_group_poset_complex_gen_labels} and let $\theta \colon A \to \pi_1(K(P), p_0)$ be the bijection at the end of that proof such that we have $\GroupPres{\image(\theta \circ l)} = \pi_1(K(P),p_0)$.

	Let $\sigma = (x_1 \lessdot \cdots \lessdot x_i)$ and $\sigma^\prime = (x^\prime_1 \lessdot \cdots \lessdot x^\prime_j)$ be two saturated chains such that $x_1 = x^\prime_1$ and $x_i = x^\prime_j$ with corresponding words $w = w_1\cdots w_{i-1}$, and $w^\prime = w^\prime_1\cdots w^\prime_{j-1}$ in $A^*$ such that $l(x_k \lessdot x_{k+1}) = w_k$ and similarly for $\sigma^\prime$. Recall that $w$ and $w^\prime$ are words that are identified by the relations in the defining presentation for $G(P)$. We want to show there exists a homotopy between representatives of $\theta(w_1)\cdots\theta(w_i)$ and representatives of $\theta(w^\prime_1)\cdots\theta(w^\prime_j)$.
	
	By doing the process in the proof of \cref{lem:fund_group_poset_complex_gen_labels} in reverse, we get a homotopy between a representative of $\theta(w_1)\cdots\theta(w_i)$ and $\beta_* \circ \alpha(x_1 \leq x_i)$ through the two skeleton of $K(P)$. By the same argument, we get a homotopy between a representative of $\theta(w^\prime_1)\cdots\theta(w^\prime_j)$ and $\beta_* \circ\alpha(x^\prime_1 \leq x^\prime_j)$. Since $x_1 = x^\prime_1$ and $x_i = x^\prime_j$, we have $\theta(w_1)\cdots\theta(w_i) = \theta(w^\prime_1)\cdots\theta(w^\prime_j)$.
	
	We have shown that $\pi_1(K(P), p_0)$ has all necessary relations to extend $\theta$ to a surjective homomorphism $\phi \colon G(P) \to \pi_1(K(P), p_0)$ by \cref{lem:extend_map_to_homomorphism}. 
\end{proof}

\begin{theorem}
	Given an edge labelled poset $(P, \, \leq, \, l \colon \mathcal{E}(P) \to A)$, we have $\pi_1(K(P)) \cong G(P)$.
	\label{thm:fund_group_poset_complex_poset_group}
\end{theorem}
\begin{proof}
	We again follow the notation from \cref{lem:fund_group_poset_complex_gen_labels}.
	At the start of the proof of \cref{lem:fund_group_poset_complex_gen_labels} we remarked that by the structure of the 1--skeleton of $K(P)$ that $\pi_1(K(P),p_0)$ is generated by a set of loops in bijection with 
	\begin{equation*}
		\mathcal{L}(\Sigma) \coloneq \Set{\mathcal{L}(C) \given C \text{ is a 1--chain}} \,.
	\end{equation*}
Of course, \cref{lem:fund_group_poset_complex_gen_labels} tells us that this is not a minimal generating set. Let $\chi \colon \mathcal{L}(\Sigma) \to \pi_1(K(P), p_0)$ denote this bijection. We now think of this presentation abstractly, and work to give a set of relations $R$ such that we obtain a group $\GroupPres{\mathcal{L}(\Sigma) \relations R\,} \cong \pi_1(K(P),p_0)$ where the isomorphism is an extension of $\chi$.

Again, by the structure of the 1--skeleton of $K(P)$ remarked in the proof of \cref{lem:fund_group_poset_complex_gen_labels}, there is such a set of relations $R$ in bijection with the set of 2--simplices in $K(P)$. Let $e^2$ be a 2--simplex with edges $e^1_i$ for $i \in \Set{1,2,3}$. We have that the edges $e^1_i$ are oriented acyclically as in \cref{fig:example_simplex_in_KP}.

\begin{figure}[h]
	\centering
	\begin{tikzpicture}[scale=1.5]
		\node[FSC] (base) {};
		\node[FSC] (mid) at ($(base) + (0,1)$) {};
		\node[FSC] (top) at ($(base) + (0.8,1.5)$) {};

		\draw[arrow_me=stealth] (base) to node[auto] {$e^1_1$} (mid);
		\draw[arrow_me=stealth] (mid) to node[auto] {$e^1_2$} (top);
		\draw[arrow_me=stealth] (base) to node[auto, swap] {$e^1_3$} (top);
	\end{tikzpicture}
	\caption{The orientation of all simplices in $\Delta(P)$ and $K(P)$.}
	\label{fig:example_simplex_in_KP}
\end{figure}

All simplices are oriented this way because if $x \leq y$ and $y \leq z$ then $x \leq z$.
Define loops $\lambda_i \in \Lambda$ such that each $\lambda_i$ follows the oriented edge $e_i$. Choose extended labels $\ell_i$ such that $\chi(\ell_i) = [\lambda_i]$. We see that the 2--simplex $e^2$ corresponds to the relation $\ell_1\ell_2\ell_3^{-1} = 1$. Here multiplication of extended labels is formal, as are inverses, denoted $\ell^{-1}$. Define $R$ to be the following set of relations for $\GroupPres{\mathcal{L}(\Sigma)}$.

\begin{equation}
	R \coloneq \Set*{\ell_1\ell_2\ell_3^{-1} = 1 \given \exists \text{ 2--chain } x< y < z \text{ where} \quad
	\begin{aligned}
		&\mathcal{L}(x < y) = \ell_1 \\
		&\mathcal{L}(y < z) = \ell_2 \\
		&\mathcal{L}(x < z) = \ell_3
	\end{aligned}}
	\label{eqn:relations_in_extended_labels}
\end{equation}
Following from the definition of $K(P)$, and considering its 2--skeleton, these are exactly the relations we need to extend $\chi$ to an isomorphism witnessing $\GroupPres{ \mathcal{L}(\Sigma) \relations R } \cong \pi_1(K(P),p_0)$.

We can choose (using axiom of choice if necessary) an element of $\ell$ for all $\ell \in \mathcal{L}(\Sigma)$. We encode this choice in the function $f\colon \mathcal{L}(\Sigma) \to A^*$ such that $f(\ell) \in \ell$ for all $\ell \in \mathcal{L}(\Sigma)$. Considering $A$ as generators of $G(P)$ with $A^*$ corresponding to elements of $G(P)$, we now work to show that $f$ extends to a homomorphism $h\colon \GroupPres{ \mathcal{L}(\Sigma) \relations R } \to G(P)$.

Recall the definition of $\mathcal{L}$ and the underlying geometry of relations in $R$. Consider the relation $\ell_1\ell_2\ell_3^{-1} = 1$, as in \eqref{eqn:relations_in_extended_labels}. Let $x$, $y$ and $z$ be such that the relevant 2--chain corresponding to the relation $\ell_1\ell_2\ell_3^{-1} = 1$ is $x < y < z$. There exists a saturated chain $C_1 = (x_1 \lessdot \cdots \lessdot x_m)$ such that the word corresponding to $C_1$ is $f(\ell_1)$ with $x_1=x$ and $x_m=y$. There also exists a chain $C_2 = (y_1 \lessdot \cdots \lessdot y_n)$ such that the word corresponding to $C_2$ is $f(\ell_2)$ with $y_1 = y$ and $y_n=z$. Finally, there is a chain $C_3 = (z_1 \lessdot \cdots \lessdot z_k)$ such that the word corresponding to $C_3$ is $f(\ell_3)$ with $z_1 = x$ and $z_k=z$. The chains $(x = x_1 \lessdot \cdots \lessdot x_m=y=y_1 \lessdot \cdots \lessdot y_n = z)$ and $(x = z_1 \lessdot \cdots \lessdot z_k=z)$ correspond to paths in the Hasse diagram of $P$ that start and end at the same place. Thus, they correspond to the $G(P)$ relation $f(\ell_1)f(\ell_2) = f(\ell_3)$. Thus, by \cref{lem:extend_map_to_homomorphism} we can extend $f$ to a homomorphism $h\colon \GroupPres{ \mathcal{L}(\Sigma) \relations R } \to G(P)$.

Recall $\phi$ from \cref{lem:exists_surj_hom_onto_fund_group}.
We have that $h(\Set{a}) = a$, thus $h \circ \chi^{-1 }\circ \phi \colon G(P) \to G(P)$ is the identity on the generating set $A \subseteq G(P)$. Thus, $h \circ \chi^{-1 }\circ \phi$ is the identity and $\phi$ is an isomorphism.
\end{proof}

Note that our proof of the above was ambivalent to the identification of $n$--simplices for $n\geq2$. Indeed, these identifications do not affect the fundamental group, but they do ensure that higher homotopy groups are trivial. We can see if we did not identify the 2--simplices in \cref{fig:example_edge_labelled_poset_with_KP}, $\pi_2(K(P))$ would be non-trivial.

\end{document}
