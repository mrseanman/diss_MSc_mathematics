% !TeX root = ../../main.tex
\documentclass[class=article, crop=false]{standalone}
\begin{document}
	
\section{The \texorpdfstring{$K(\pi,1)$}{K pi 1} conjecture}
Given a group $G$ and natural number $n$, an \emph{Eilenberg-MacLane space} \cite{eilenberg_maclane_relations_1945} is a space $X$ such that $\pi_n(X)=G$ and $\pi_i(X) = 0$ for all $1\leq i \neq n$. We call such an $X$ a \emph{$K(G,n)$ space}. We will also use the terminology \emph{classifying space for $G$} to mean that $X$ is a $K(G,1)$ space.
\begin{conjecture}[$K(\pi,1)$ Conjecture]
	For all Coxeter groups $W$, the space $Y_W$ is a $K(G_W,1)$ space.
\end{conjecture}

Admittedly, the use of $\pi$ in the name of the conjecture is confusing. An equivalent formulation of the conjecture is that the universal cover of $Y_W$ is contractible. These statements are equivalent since a cover $p \colon \widetilde{X} \to X$, induces an isomorphism  $p_* \colon \pi_n(\widetilde{X},\widetilde{x_0}) \to \pi_n(X,x_0)$ for all $n\geq 2$ \cite[Proposition 4.1]{hatcher_algebraic_2001}.

In the previous section we focused on Coxeter groups of type $A_n$. We quoted the result that $\pi_1(Y_W) \cong G_W$ for these groups. We now prove the $K(\pi,1)$ conjecture for this same family of Coxeter groups. To do so, we need to verify that the higher homotopy groups of $Y_W$ are trivial.

\begin{lemma}
	For all $k>1$, $\pi_k(\conf_n(\C))$ is trivial.
	\label{lem:labelled_conf_classiftying_space}
\end{lemma}
\begin{proof}
	We use that $\conf_n(\C)$ is a fibre bundle over $\conf_{n-1}(\C)$ with projection $p$ forgetting a point and fibres homeomorphic to $\C \setminus \Set{\text{n distinct points}}$ \cite[Theorem 3]{fadell_neuwirth_configuration_1962}.

The space $\C \setminus \Set{\text{n distinct points}}$ is homotopy equivalent to $\bigvee_n S^1$, so we have the fibration $\bigvee_n S^1 \hookrightarrow \conf_n(\C) \to \conf_{n-1}(\C)$ and the corresponding short exact sequence
\begin{equation}
	\begin{tikzcd}
		&\pi_k(\bigvee_{n-1} S^1) \arrow[hook]{r} &\pi_k(\conf_n(\C)) \arrow{r}{p} &\pi_k(\conf_{n-1}(\C))
	\end{tikzcd}
	\label{eqn:conf_fibration}
\end{equation}
for all $k$.
We prove that $\pi_k(\conf_n{\C})=0$ for all $k$ by induction on $n$. We know $\pi_k(\bigvee_nS^1)=0$ for all $k>1$ and for all $n$. So the leftmost term in \eqref{eqn:conf_fibration} is always trivial. Our base case is $n=2$. The rightmost term in \eqref{eqn:conf_fibration} is $\conf_1(\C) \cong \C$, which has trivial higher homotopy. The leftmost term is $\bigvee_1 S^1 \cong S^1$ which also has trivial higher homotopy. Thus, $\conf_2(\C)$ has trivial higher homotopy. Assuming now that $\pi_k(\conf_{n-1}(\C)) = 0$ for all $k>1$, the inductive step follows immediately from the short exact sequence.
\end{proof}

Note that so far we have only proved that $Y$ (from \cref{def:config_space}) has trivial higher homotopy, not $Y_W$.

\begin{theorem}
	The $K(\pi,1)$ conjecture holds for Coxeter groups of type $A_n$.
\end{theorem}
\begin{proof}
	The action $A_n \acts Y$ is by permutation of points, therefore $Y_W$ is the space of configurations of $n+1$ \emph{unlabelled} points. Let $q \colon \widetilde{Y} \to Y$ be the universal cover for $Y$. Let $r \colon Y \to Y_W$ be the quotient map induced by the group action. We see that $r$ is a covering map such that $r^{-1}(z)$ is finite for all $z \in Y_W$. Thus, by \cite[Exercise 53.4]{munkres_topology_2000}
	\what[{Unfortunately I couldn't find a better (non-exercise) reference for this. The proof is OK, but not that short and a bit of a detour in my opinion}]
	the composition $r \circ q$ is also a covering map so $\widetilde{Y}$ is also the universal cover of $Y_W$ and so $Y_W$ also has trivial higher homotopy by \cref{lem:labelled_conf_classiftying_space}.
\end{proof}

A vital result due to Deligne expands on this result.
\begin{theorem}[\hspace{1sp}{\cite{deligne_immeubles_1972}}]
	The $K(\pi,1)$ conjecture holds for all finite Coxeter groups $W$.
	\label{thm:k_pi_1_finite}
\end{theorem}

The paper of interest to us, \cite{paolini_salvetti_kpi1_2021}, proves the $K(\pi,1)$ conjecture for affine Coxeter groups. 

\end{document}