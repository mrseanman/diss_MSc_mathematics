% !TeX root = ../../main.tex
\documentclass[class=article, crop=false]{standalone}
\begin{document}
	
\section{Coxeter groups and Artin groups}

In this section we will cover the constructions and some properties of Coxeter groups and Artin groups. Coxeter groups are a generalisation of \emph{reflection groups}, which are subgroups of $GL_n(\R)$ generated by a finite set of reflections. Although the definition of Coxeter groups is tied to an abstract group presentation, we may also think of them as groups acting on a space by compositions of reflections. For example, finite Coxeter groups can be realised as reflection groups on spheres and affine Coxeter groups can be realised as groups generated by affine reflections in $\R^n$ (with plane of reflection not necessarily passing through the origin). Note that the realisation of a Coxeter group as a group generated by reflections is not unique and that some Coxeter groups cannot be realised as a subgroup of $GL_n(\R)$.

\begin{definition}
	Given a finite set $S$, let $m \colon S \times S \to \N \cup \Set{\infty}$ be a symmetric matrix indexed over $S$ such that $m(s,s)=1$ for all $s \in S$ and $m(s,t)$ takes values in $\Set{2,3,\ldots} \cup \Set{\infty}$ for all $s \neq t$. The \emph{Coxeter group} $W$ associated to $m$ is the group with the following presentation.
	\begin{equation*}
		W = \GroupPres{ S \relations (st)^{m(s,t)} = 1 \quad \forall m(s,t) \neq \infty}
		\label{eq:coxeter_presentation}
	\end{equation*}
	\label{def:coxeter_group}
	The data of $m$ and $S$ along with the associated Coxeter group $W$ is denoted $(W,S)$ and called a \emph{Coxeter system}.
\end{definition}

Throughout this work we will use 1 to denote the identity element in a group and $\Set{1}$ to denote the trivial group.
The pairs $(s,t)$ such that $m(s,t)=\infty$ are pairs of generators that have no explicit relations. Since $m(s,s)=1$ for all $s \in S$, all generators have order 2. Note that the data of $m$ is not uniquely determined by the isomorphism class of $W$, hence the need to distinguish a Coxeter system, not just a Coxeter group. The set $R \coloneq \Set{wsw^{-1} \mid w \in W, s \in S}$ is the set of \emph{reflections} in $W$. Sometimes $S$ is referred to as the set of \emph{basic reflections}, or that a choice of $S$ is a choice of basic reflections.

A labelled graph, called the \emph{Coxeter diagram}, is often used to encode the data of the matrix $m$ and its corresponding Coxeter group. In this graph, each element of $S$ is a node and relations between pairs in $S$ correspond to labelled edges. There are two conventions for this labelling. The \emph{classical labelling}, where edges with $m(s,t)=2$ are not drawn, edges with $m(s,t)=3$ are drawn but not labelled and all other edges are drawn with the value of $m(s,t)$ as their label. And the \emph{modern labelling}, where edges with $m(s,t)=\infty$ are not drawn, edges with $m(s,t)=2$ are drawn but not labelled and all other edges are drawn and labelled. An example highlighting these different conventions is given in  \cref{fig:example_coxeter_diagrams}. We will only use the classical labelling here, but awareness of the modern labelling is useful.

\begin{figure}[ht]
	\begin{center}
	\begin{tikzpicture}[scale=1.2]
		\tikzstyle{every label}=[font=\tiny]
		\tikzstyle{every node}=[font=\scriptsize]
		
		\node[FSC] (1)	at (0,0)	   						{};
		\node[FSC] (2)	at ($ (1) + (-0.866, 0.5)$)			{};
		\node[FSC] (3)	at ($ (1) + (0, 1)$)				{};
		\node[FSC] (4) 	at ($ (1) + (0.866,0.5)$)			{};
		\node[FSC] (5) 	at ($ (4) + (1,0)$)					{};		
		
		\draw (1) 	to								(2);
		\draw (1)	to								(3);
		\draw (2)	to								(3);
		\draw (4)	to node[above] {$\infty$}		(5);
	\end{tikzpicture}
	\hspace{2cm}
	\begin{tikzpicture}[scale=1.2]
		\tikzstyle{every label}=[font=\tiny]
		\tikzstyle{every node}=[font=\scriptsize]
		
		\node[FSC] (1)	at (0,0)	   						{};
		\node[FSC] (2)	at ($ (1) + (-0.866, 0.5)$)			{};
		\node[FSC] (3)	at ($ (1) + (0, 1)$)				{};
		\node[FSC] (4) 	at ($ (1) + (0.866,0.5)$)			{};
		\node[FSC] (5) 	at ($ (4) + (1,0)$)					{};
		
		\draw (1) 	to node[auto]	{3}			(2);
		\draw (1)	to node[left]	{3}			(3);
		\draw (2)	to node[auto]	{3}			(3);
		\draw (1)	to							(4);
		\draw (3)	to							(4);
	\end{tikzpicture}
	\end{center}
	\caption{Coxeter diagram for a certain Coxeter group with classical labelling (left) and modern labelling (right).}
	\label{fig:example_coxeter_diagrams}
	\end{figure}

In the classical labelling, if the diagram has multiple connected components then $W$ is a direct product of the Coxeter groups corresponding to those components. Similarly, in the modern labelling, connected components correspond to factors in a free product. Other topological properties of these diagrams can be used, for example, in work of Huang \cite{huang_labeled_2023} which proves the $K(\pi, 1)$ conjecture for certain $W$ with diagrams being trees or containing cycles. The property of Coxeter groups that allows us to make this graph construction is that every relation in a Coxeter group only involves two generators and is encoded by a number. 

To each Coxeter system $(W,S)$ there is an associated Artin group $G_W$ defined as follows.

\begin{definition}
	For group elements $s$ and $t$, let $\Pi(s,t;n)$ be the alternating product of $s$ and $t$ starting with $s$ with total length $n$, e.g.~$\Pi(s,t;3) = sts$. Given a Coxeter system $(W,S)$ with associated matrix $m$, the associated \emph{Artin group} is
	\begin{equation*}
		G_W \coloneq \GroupPres{ S \relations \Pi(s,t;{m(s,t)}) = \Pi(t,s;{m(s,t)}) \,\,\, \forall s \neq t \,\, \text{and} \,\, m(s,t) \neq \infty}.
	\end{equation*}
	\label{def:artin_group}
\end{definition}

Note that $m(s,s)=1$ now carries no meaning in the presentation of $G_W$ and that if we add the relations $s^2=1$ for all $s \in S$ we retrieve the original Coxeter group. The Coxeter diagram for $W$ also encodes the data of $G_W$ and the connected components of the diagram correspond to factors of $G_W$ as a direct product or as a free product as with $W$.

Our notation for Artin groups (as with much of the notation here) is from \cite{paolini_salvetti_kpi1_2021}. Another common notation is $W_\Gamma$ and $A_\Gamma$ for the Coxeter and Artin groups corresponding to the Coxeter diagram $\Gamma$. When classifying Artin groups, it is common to inherit properties from the corresponding Coxeter group such that ``\texttt{property} (type) Artin groups'' describes a family of Artin groups to which their corresponding Coxeter groups are \texttt{property}.

In particular, spherical or finite type Artin groups have associated spherical or finite Coxeter groups. Similarly, affine Artin groups have associated Coxeter groups which are affine.

\end{document}