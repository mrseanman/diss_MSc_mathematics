% !TeX root = ../../main.tex
\documentclass[class=article, crop=false]{standalone}
\begin{document}
	
\subsection{Coxeter Groups and Artin Groups}

In this section we will cover the constructions and linked properties of the two groups of interest to this paper. Coxeter groups are a generalisation of \emph{reflection groups}, which are subgroups of $O(\R,n)$ generated by a finite set of reflections. Although the definition of Coxeter groups is tied to an abstract group presentation, we must also think of them as groups with a natural reflection action on some space. For example, finite Coxeter groups can be realised as reflection groups on spheres and affine Coxeter groups can be realised as groups generated by affine (with plane of reflection not necessarily passing through the origin) reflections in $\R^n$. Note that the realisation of a Coxeter group as a group generated by reflections is not unique and that some Coxeter groups cannot be realised as a subgroup of $O(\R,n)$.

\begin{definition}[Coxeter Group]
	For a finite set $S$, a Coxeter group $W$ generated by $S$ is a group with presentation of the form
	\begin{equation*}
		W = \GroupPres{ S \relations (st)^{m(s,t)} = 1 \quad \forall m(s,t) \neq \infty}
		\label{eq:coxeter_presentation}
	\end{equation*}
	where $m \colon S \times S \to \N$ is a symmetric matrix indexed over $S$ where $m(s,s)=1$ for all $s \in S$ and $m(s,t)$ takes values in $\Set{2,3,\ldots} \cup \Set{\infty}$ for all $s \neq t$.
	\label{def:coxeter_group}
\end{definition}

The infinities correspond to pairs of elements that have no explicit relations. The ones along the diagonal of $m$ ensure that all generators have order 2. The set $R \coloneq \Set{wsw^{-1} \mid w \in W, s \in S}$ is the set of \emph{reflections} in $W$. Sometimes $S$ is referred to as the set of \emph{basic reflections}, or that a choice of $S$ is a choice of basic reflections.

A graph, called the \emph{Coxeter diagram}, is often used to encode the data of the matrix $m$ and its corresponding Coxeter group. In this graph, each element of $S$ is a node and relations between pairs in $S$ correspond to labelled edges. There are two conventions for this labelling: The \emph{classical labelling}, where edges with $m(s,t)=2$ are not drawn, edges with $m(s,t)=3$ are drawn but not labelled and all other edges are drawn with the value of $m(s,t)$ as their label. And the \emph{modern labelling}, edges with $m(s,t)=\infty$ are not drawn, edges with $m(s,t)=2$ are drawn but not labelled and all other edges are drawn and labelled. An example highlighting these differences is given in  \cref{fig:example_coxeter_diagrams}. We will only use the classical labelling here, but awareness of the modern labelling is useful.

In the classical labelling, if the diagram has multiple connected components then $W$ is a direct product of the groups corresponding to those components. Similarly, in the modern labelling connected components are factors in a free product. Other topological properties of these diagrams can be used, for example in \cite{huang_labeled_2023} which proves the $K(\pi, 1)$ conjecture for certain $W$ with diagrams being trees or containing cycles. The property of Coxeter groups that allows us to make this graph construction is that every relation in a Coxeter group only involves two generators, and that each relation is encoded by a number and a pair of generators. 

\begin{figure}[ht]
\begin{center}
\begin{tikzpicture}[scale=1.2]
	\tikzstyle{every label}=[font=\tiny]
	\tikzstyle{every node}=[font=\footnotesize]
	
	\node[FSC] (1)	at (0,0)	   						{};
	\node[FSC] (2)	at ($ (1) + (-0.866, 0.5)$)			{};
	\node[FSC] (3)	at ($ (1) + (0, 1)$)				{};
	\node[FSC] (4) 	at ($ (1) + (0.866,0.5)$)			{};
	\node[FSC] (5) 	at ($ (4) + (1,0)$)					{};		
	
	\draw (1) 	to								(2);
	\draw (1)	to								(3);
	\draw (2)	to								(3);
	\draw (4)	to node[above] {$\infty$}		(5);
\end{tikzpicture}
\hspace{2cm}
\begin{tikzpicture}[scale=1.2]
	\tikzstyle{every label}=[font=\tiny]
	\tikzstyle{every node}=[font=\footnotesize]
	
	\node[FSC] (1)	at (0,0)	   						{};
	\node[FSC] (2)	at ($ (1) + (-0.866, 0.5)$)			{};
	\node[FSC] (3)	at ($ (1) + (0, 1)$)				{};
	\node[FSC] (4) 	at ($ (1) + (0.866,0.5)$)			{};
	\node[FSC] (5) 	at ($ (4) + (1,0)$)					{};
	
	\draw (1) 	to node[auto]	{3}			(2);
	\draw (1)	to node[right]	{3}			(3);
	\draw (2)	to node[auto]	{3}			(3);
	\draw (1)	to							(4);
	\draw (3)	to							(4);
\end{tikzpicture}
\end{center}
\caption{Coxeter diagram for a certain Coxeter group with classical labelling (left) and modern labelling (right).}
\label{fig:example_coxeter_diagrams}
\end{figure}

To each Coxeter group $W$ there is an associated Artin group $G_W$ defined as follows

\newpage

\begin{definition}[Artin Group]
	For a given Coxeter group $W$ generated by $S$ with associated matrix $m$, the associated Artin group is
	\begin{equation*}
		G_W \coloneq \GroupPres{ S \relations \Pi(s,t,{m(s,t)}) = \Pi(t,s,{m(s,t)}) \,\,\, \forall s \neq t \,\, \text{and} \,\, m(s,t) \neq \infty}
	\end{equation*}
	where $\Pi(s,t,n)$ is defined to be an alternating product of $s$ and $t$ starting with $s$ with total length $n$. E.g.~$\Pi(s,t,3) = sts$.
	\label{def:artin_group}
\end{definition}

Note that the ones along the diagonal of $m$ now carry no meaning in the presentation and that if we add the relation $s^2=1$ for all $s \in S$ we retrieve the original Coxeter group. The Coxeter diagram for $W$ also encodes the data of $G_W$ and the topology of the diagram holds similar meaning for $G_W$. Our notation for Artin groups, $G_W$ (shared in much of the literature), seems to imply the data for the Artin group is inherited from its Coxeter group. While each presentation determines the other, in principle there is no precedence, but practically we often start by defining a Coxeter group. Often ``\texttt{property} type Artin groups'' describes a family of Artin groups to which their corresponding Coxeter groups are \texttt{property}.

In particular, \emph{spherical} or \emph{finite} type Artin groups have associated spherical or finite type Coxeter groups. Similarly, affine type Artin groups have affine associated Coxeter groups.


\end{document}