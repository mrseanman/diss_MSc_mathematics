% !TeX root = ../../main.tex
\documentclass[class=article, crop=false]{standalone}
\begin{document}

In this paper we will be concerned with the $K(\pi,1)$ conjecture for Artin groups. This states that the configuration space $Y_W$ for any Coxeter group $W$ is a $K(G_W, 1)$ space, where $G_W$ is the Artin group associated to $W$. This conjecture emerges as a generalisation of the result for Coxeter groups of type $\tilde{A}_n$ and is originally attributed to Arnol'd, Pham and Thom in \cite{lek_homotopy_1983}. See also \cite{charney_k_1995} for a good overview of the history of the conjecture.

The specific focus here will be on \cite{paolini_salvetti_kpi1_2021} in which the authors prove the $K(\pi,1)$ conjecture for Artin groups of affine type. Here, we will review some theorems in that paper and provide background such that someone not familiar with the field will be able to follow along. Much of this will involve proving a chain of homotopy equivalences. A strong theme will be the involvement of posets in these proofs and related structures in these proofs, hence the title of this paper. We will begin by providing a birds eye view of the main results in \cite{paolini_salvetti_kpi1_2021}, after which we will give an introduction to the objects involved.

\end{document}