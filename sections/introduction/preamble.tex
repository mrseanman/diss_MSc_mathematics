% !TeX root = ../../main.tex
\documentclass[class=guthesis, crop=false]{standalone}
\begin{document}

In this paper we will be concerned with the $K(\pi,1)$ conjecture for Artin groups. This states that the configuration space $Y_W$ for any Coxeter group $W$ is a $K(G_W, 1)$ space, where $G_W$ is the Artin group associated to $W$. This conjecture emerges as a generalisation of the result for Coxeter groups of type $A_n$ and is originally attributed to Arnol'd, Pham and Thom in \cite{lek_homotopy_1983}. See also \cite{charney_davis_pi_1995} for a good overview of the history of the conjecture.

The specific focus here will be on the work of Paolini and Salvetti, \emph{Proof of the $K(\pi, 1)$ conjecture for affine Artin groups} \cite{paolini_salvetti_kpi1_2021}. In this work, we will review some theorems in that paper and provide relevant background. Their main Theorem relied on proving a chain of homotopy equivalences, which will be our focus here. A strong theme will be the involvement of posets and related structures, hence the title of this work. We will begin by providing a birds eye view of the main results in \cite{paolini_salvetti_kpi1_2021}.

\end{document}