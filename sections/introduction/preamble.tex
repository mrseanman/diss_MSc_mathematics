% !TeX root = ../../main.tex
\documentclass[class=guthesis, crop=false]{standalone}
\begin{document}

A classifying space for a group $G$ is a space $X$ such that the fundamental group of $X$ is $G$ and all higher homotopy groups of $X$ are trivial.
In this paper we will be concerned with the $K(\pi,1)$ conjecture for Artin groups, which states that the configuration space $Y_W$ for any Coxeter group $W$ is a classifying space for the Artin group $G_W$.  This conjecture emerges as a generalisation of the result for Coxeter groups of type $A_n$ and is originally attributed to Arnol'd, Brieskorn, Pham and Thom in \cite{paris_kpi1_2014}. See also \cite{charney_davis_pi_1995} for a good overview of the history of the conjecture.

We will focus on the work of Paolini and Salvetti, \emph{Proof of the $K(\pi, 1)$ conjecture for affine Artin groups} \cite{paolini_salvetti_kpi1_2021}. We will review some theorems therein and provide relevant background. The theorem which is the namesake of \cite{paolini_salvetti_kpi1_2021} relies on proving a chain of homotopy equivalences \eqref{eqn:proof_overview}, detailing two of these homotopy equivalences is the aim of this work. A strong theme will be the involvement of posets and related structures, hence this work's title. We will begin by providing a birds eye view of the conjecture and the  main results of \cite{paolini_salvetti_kpi1_2021}.

The main results of this work are \cref{thm:fund_group_poset_complex_poset_group,thm:salvetti_cx_equiv_X_prime,thm:subcomplex_K_prime_hom_equiv_K}.
\cref{thm:fund_group_poset_complex_poset_group} is stated by McCammond in \cite{mccammond_introduction_2005}. It is not proven there, but it is alluded to being widely accepted as true. Its proof here and the lemmas leading to that proof are the author's own work.
\cref{thm:salvetti_cx_equiv_X_prime} is proven by Paolini and Salvetti in \cite{paolini_salvetti_kpi1_2021}. The proof here follows that in \cite{paolini_salvetti_kpi1_2021} but fills in significant details.
The proof of \cref{thm:subcomplex_K_prime_hom_equiv_K} largely follows that in \cite{paolini_salvetti_kpi1_2021}. The main original contribution to the proof of \cref{thm:subcomplex_K_prime_hom_equiv_K} here is the proof of \cref{lem:K_prime_valid_subcomplex}, which is omitted in \cite{paolini_salvetti_kpi1_2021}.
Fibre doubling and the concepts explored in \cref{chap:ideas_exploration_conclusion} are the authors own work.


\end{document}