% !TeX root = ../../main.tex
\documentclass[class=guthesis, crop=false]{standalone}
\begin{document}

In this paper we will be concerned with the $K(\pi,1)$ conjecture for Artin groups. This states that the configuration space $Y_W$ for any Coxeter group $W$ is a classifying space for the Artin group $G_W$. A classifying space for a group $G$ is a space $X$ such that the fundamental group of $X$ is $G$ and all higher homotopy groups of $X$ are trivial. This conjecture emerges as a generalisation of the result for Coxeter groups of type $A_n$ and is originally attributed to Arnol'd, Pham and Thom in \cite{lek_homotopy_1983}. See also \cite{charney_davis_pi_1995} for a good overview of the history of the conjecture.

We will focus on the work of Paolini and Salvetti, \emph{Proof of the $K(\pi, 1)$ conjecture for affine Artin groups} \cite{paolini_salvetti_kpi1_2021}. We will review some theorems therein and provide relevant background. Their main Theorem relies on proving a chain of homotopy equivalences \eqref{eqn:proof_overview}, detailing these homotopy equivalences is the aim of this work. A strong theme will be the involvement of posets and related structures, hence this work's title. We will begin by providing a birds eye view of the conjecture and the  main results in \cite{paolini_salvetti_kpi1_2021}.

\end{document}