% !TeX root = ../../main.tex
\documentclass[class=article, crop=false]{standalone}
\begin{document}

\section{The configuration space}
\label{sec:config_space_intro}

This section contains the definition of the configuration space $Y_W$ for a given Coxeter group $W$. We also introduce the $A_n$ family of Coxeter groups for which the space $Y_W$ is the space of configurations of $n+1$ labelled points in $\C$. 

For some finite or affine Coxeter group $W$ acting on $\R^n$, the set of reflections $R \in W$ acts on $\R^n$ by reflection through hyperplanes, one for each $r \in R$. For some $r \in R$, denote its hyperplane by $H_r \subseteq \R^n$. Denote the union of all hyperplanes by $\mathcal{H} \coloneqq \bigcup_{r \in R} H_r$.
Consider the tensor product $\R^n \otimes \C$. This is isomorphic to $\C^n$ under the natural isomorphism $x \otimes \lambda \mapsto x\lambda$. We can extend the action $W\acts \R^n$ to $W\acts (\R^n \otimes \C)$ via $w\cdot(x \otimes \lambda) = (w\cdot x)\otimes \lambda$. We call this act of transporting objects related to $\R^n$ over to $\R^n \times \C$ (which is isomorphic to $\C^n$) via the tensor product \emph{complexification}.

\begin{remark}
	\label{rmk:action_well_defined_on_complement}
	Since $R$ generates $W$, the action $W \acts \R^n$ fixes $\mathcal{H}$ and by the nature of the action $R \acts \R^n$, we have that $w \cdot x \in \mathcal{H} \implies x \in \mathcal{H}$.
\end{remark}
\begin{definition}
	For an affine Coxeter group $W$ and associated hyperplane system $\mathcal{H}$ as above, we define
	\begin{equation*}
		Y \coloneqq (\R^n\otimes \C) \, \setminus \, \left( \mathcal{H} \otimes \C \right)
	\end{equation*}
	and define the \emph{configuration space} $Y_W$ to be the quotient $Y/W$ with the action of $W$ defined above. This action is well--defined by \cref{rmk:action_well_defined_on_complement}.
	\label{def:config_space}
\end{definition}

Note that the importance of $\C$ is that it is 2--dimensional. When one takes the complement of a codimension--1 object, typically the resulting topology is not very interesting. By complexifying the hyperplanes and then taking the complement within $\R^n \otimes \C$, we are effectively taking the complement of a codimension--2 object, and there is much more room for interesting topologies. The same construction can be achieved using $\R^{2n}$ and $\mathcal{H}\times\mathcal{H}$. A more general construction of $Y_W$ for all Coxeter groups using the \emph{Tits cone} can be found in \cite{paris_kpi1_2014}. We will not go in to the details of this construction, but will assume $Y_W$ to be defined for all Coxeter groups $W$, not just affine $W$ as in \cref{def:config_space}.

For a concrete example concerning $Y_W$, we will introduce the $A_n$ family of Coxeter groups and show that the space $Y_W$ for these groups is the space of configurations of $n+1$ points in $\C$, thus explaining the name \emph{configuration space} for general $Y_W$.

The family $A_n$ all have Coxeter diagrams of the form as in \cref{fig:A_n_dynkin_diagrams} and a specific $A_n$ will have presentation.
\begin{equation}
	A_n = 
	\GroupPres*{ \sigma_1, \sigma_2, \ldots ,\sigma_n \relations
	\begin{aligned}
		\sigma_i^2 &= 1 													&&\forall i\\
		(\sigma_i\sigma_j)^2 &= 1 											&&\forall(i+1 < j \leq n) \\
		(\sigma_i\sigma_{i+1})^3 &= 1 										&&\forall(i < n)
	\end{aligned}}
	\label{eq:A_n_presentation}
\end{equation}
 
 This is well known to be a presentation for the symmetric group $S_{n+1}$ with generators being adjacent transpositions \cite[Proposition 1.5.4]{bjorner_brenti_combinatorics_2010}. Accordingly,  we will use the associated cycle notation for symmetric groups to talk about elements of $A_n$. 

\begin{figure}
\begin{center}
\begin{tikzpicture}[baseline=0.7em]
\node[FSC] (1) 			at (0,0) 	{};
\node[FSC] (2) 			at (1,0) 	{};
\node[FSC] (3) 			at (2,0) 	{};
\node (3 and a bit) 	at (2.4,0)	{};

\node (dots) at (3,0) {\large$\cdots$};

\node (4 minus a bit) 	at (3.6, 0)	{};
\node[FSC] (4) 			at (4,0) 	{};
\node[FSC] (5) 			at (5,0) 	{};

\draw (1) to (2) to (3) to (3 and a bit);
\draw (4 minus a bit) to (4) to (5);

\draw [decorate,
			decoration = {calligraphic brace,mirror, raise=4pt, amplitude=8pt}]
	($(1) - (0.2, 0)$) --  ($ (5) + (0.2,0)$)
	node[pos=0.5, below=15pt, black]{$n$ nodes};

\end{tikzpicture}
\end{center}
\caption{The classical Coxeter diagram for the Coxeter group of type $A_n$.}
\label{fig:A_n_dynkin_diagrams}
\end{figure}

The action of $A_n$ as a reflection group is realised on the space $\R^{n+1}$ with basis $\Set{e_i}$, where $A_n$ acts on $\R^{n+1}$ by permuting components with respect to that basis.
The set of reflections $R$ of $A_n$ is all conjugations of the $n$ adjacent generating transpositions $(l, l+1)$. So, $R$ is the set of all transpositions $(l, k)$.
Some $(l, k) \in R$ acts on $\R^{n+1}$ as reflection through the plane $\Set{ (x_1, \ldots, x_{n+1} ) \in \R^{n+1} \mid x_l = x_k}$. Thus, taking the complement of the complexification of all such planes, we have $Y = \Set{(\mu_1, \ldots ,\mu_{n+1}) \in \C^{n+1} \given \forall i,j \,\,\, \mu_i \neq \mu_j}$ (here $Y$ is as in \cref{def:config_space}).
We can think of this as the space of $n+1$ distinct, labelled points in $\C$, denoted $\conf_{n+1}(\C)$. The action of $A_n$ on $\R^n \otimes \C$ is also by permutation of components, so $Y_W = Y/A_n$ is the space of $n+1$ \emph{unlabelled} points in $\C$, denoted $\confu_n(\C)$

Historically, Artin \cite{artin_braids_1947} originally defined the braid group on $n$ strands $B_n$ to be $\pi_1(\confu_n(\C))$. He then proved the well known presentation of the braid group. 

\begin{equation*}
	B_n = \GroupPres*{ \,\, \sigma_1, \sigma_2, \ldots , \sigma_{n-1}
		\relations
	\begin{aligned}
		\sigma_i\sigma_j &= \sigma_j\sigma_i &&\forall(i+1 < j \leq n) \\
		\sigma_i\sigma_{i+1}\sigma_i &= \sigma_{i+1}\sigma_{i}\sigma_{i+1} &&\forall(i < n)
	\end{aligned}
	}
\end{equation*}

In this context, showing the validity of the presentation immediately proves $B_{n+1} \cong G_W$ and thus that $\pi_1(Y_W) \cong G_W$. See the work of Fox and Neuwirth \cite{fox_neuwirth_braid_1962} for an alternative proof of the presentation.

\end{document}
