\documentclass[class=article, crop=false]{standalone}
% For whatever reason these must be in here as well as in my_preamble
\usepackage[subpreambles=true]{standalone}
\usepackage{import}
\usepackage{sty/my_preamble}
\begin{document}

\subsection{Configuration Space}

Here we will give the definition of the configuration space $Y_W$ for a given Coxeter group $W$ and explain the name \emph{configuration space} by going through an example where we will show that $\pi_1(Y_W) \cong G_W$ for $W$ of type $A_n$ following from \cite{fox_braid_1962}.

For some Coxeter group $W$ acting on $\R^n$, the set of reflections $R \in W$ acts on $\R^n$ by reflection through hyperplanes. For some $r \in R$ denote its hyperplane by $H(r) \subseteq \R^n$. Dentote the union of all hyperplanes by $\mathcal{H} \coloneqq \bigcup_{r \in R} H(r)$. We associate $\R^n \otimes \C$ with $\C^n$ under the natural isomorphism. This also extends the action $W\acts \R^n$ to $W\acts \C^n$ via $w\cdot(x \otimes \lambda) = (w\cdot x)\otimes \lambda$. We can then make our definition.

\begin{definition}[Configuration space] 
	For some Coxeter group $W$ and associated hyperplane system $\mathcal{H}$ as above, we define
	\begin{equation*}
		Y \coloneqq \C^n \, \setminus \, \left( \mathcal{H} \otimes \C \right)
	\end{equation*}
	and define the configuration space $Y_W$ to be the quotient $Y/W$.
	\label{eq:config_space_def}
\end{definition}

For a concrete example, we will introduce the $A_n$ family of Coxeter groups and show that the space $Y_W$ for Coxeter groups of type $A_n$ is the space of configurations of $n+1$ points in $\C$, explaining the name.

The family $A_n$ all have Coxeter-Dynkin diagrams of the form as in Figure \ref{fig:A_n_dynkin_diagrams}. This means they have presentation $\langle \sigma_1, \sigma_2, \ldots ,\sigma_n \mid \sigma_i^2 = 1 \,\, \forall i, \,\, (\sigma_i \sigma_{i+1})^3 = 1 \,\, \forall i<n\,\,\rangle$. This well known to be a presentation for the symmetric group $S_{n+1}$ with generators being adjacent transpositions \cite[Proposition 1.5.4]{bjorner_combinatorics_2005}. Accordingly,  we will use the associated cycle notation in $S_{n+1}$ to talk about elements of $A_n$. 

\begin{figure}
\begin{center}
\begin{tikzpicture}

\node[FSC] (1) 			at (0,0) 	{};
\node[FSC] (2) 			at (1,0) 	{};
\node[FSC] (3) 			at (2,0) 	{};
\node (3 and a bit) 	at (2.4,0)	{};

\node (dots) at (3,0) {\large$\cdots$};

\node (4 minus a bit) 	at (3.6, 0)	{};
\node[FSC] (4) 			at (4,0) 	{};
\node[FSC] (5) 			at (5,0) 	{};

\draw (1) to (2) to (3) to (3 and a bit);
\draw (4 minus a bit) to (4) to (5);

\draw [decorate,
			decoration = {calligraphic brace,mirror, raise=4pt, amplitude=8pt}]
	($(1) - (0.2, 0)$) --  ($ (5) + (0.2,0)$)
	node[pos=0.5, below=15pt, black]{$n$};

\end{tikzpicture}
\end{center}
\caption{The clasical Coxeter-Dynkin diagram for the Coxeter group of type $A_n$.}
\label{fig:A_n_dynkin_diagrams}
\end{figure}

The action of $A_n$ as a reflection group is realised on the space $\R^{n+1}$ with basis $\{e_i\}$, where $A_n \acts \R^{n+1}$ by permuting components with respect to that basis.
The set of reflections $R$ of $A_n$ is all conjugations of the $n$ adjacent generating transpositions $(l, l+1)$. Which is to say, $R$ is the set of all transpositions $(l, n)$. Some $(l, n) \in R$ acts on $\R^{n+1}$ as reflection through the plane $P_{(l,n)} = \{ (x_1, \ldots, x_{n+1} ) \in \R^{n+1} \mid x_l = x_n\}$. Thus here we have $Y = \{(\mu_1, \ldots \mu_n) \in \C^n \mid \forall i,j \,\,\, \mu_i \neq \mu_j\}$. We can think of this as the space of $n$ distinct labelled points in $\C$. The action $A_n \acts \C^n$ is also permutation components, so we can think of the configuration space $Y_{W}$ as the set of $n$ distinct \emph{unlabelled} points in $\C$, which we will denote $X^n$.

Historically, Emile Artin \cite{artin_braids_1947} originally defined the braid group $B_n$ to be $\pi_1(X^n)$. He then showed the validity of the well known presentation of the braid group. In this context, showing the validity of that presentation in turn proves $B_n = G_W$. This proof by Artin is often considered dubious and other proofs are available. One good example is \cite{fox_braid_1962}.

\end{document}