% !TeX root = ../../main.tex
\documentclass[class=guthesis, crop=false]{standalone}
\begin{document}

\section{Overview of main results}
Here we compile the main results from \cite{paolini_salvetti_kpi1_2021} in to a few theorems. Many of the objects here are not yet defined. Those mentioned in this section and defined in this paper are;\,\,
$W$:~\labelcref{def:coxeter_group},\,\,
$G_W$:~\labelcref{def:artin_group},\,\,
$Y_W$:~\labelcref{def:config_space},\,\,
$K_W$:~\labelcref{def:interval_complex}\,\,
$X_W$:~\labelcref{def:salvetti_cx}, \,\, 
$X^\prime_W$:~\labelcref{def:subcomplex_X_prime}\,\, and\,\,
$K_W^\prime$:~\labelcref{def:subcomplex_K_prime}.

\begin{theorem}[\hspace{1sp}{\cite{paolini_salvetti_kpi1_2021}}]
	Given an affine Coxeter group $W$, the configuration space $Y_W$ is homotopy equivalent to the order complex $K_W$.
	\label{thm:proof_overview}
\end{theorem}
\begin{proof}
	By \cref{thm:salvetti_hom_equiv_config} the Salvetti complex $X_W$ is homotopy equivalent to the configuration space $Y_W$. Therefore, we need only show $K_W \simeq X_W$. This is done through a composition of homotopy equivalences
	\begin{equation}
		X_W \labelrel\simeq{eqmid:salvetti_salvettiprime}
		X^\prime_W \labelrel\simeq{eqmid:salvettiprime_intervalcomplexprime}
		K^\prime_W \labelrel\simeq{eqmid:intervalcomplexprime_intervalcomplex}
		K_W
	\label{eqn:proof_overview}
	\end{equation}

	Where the results are gathered from the following sources:
	
	\begin{enumerate}[(a)]
		\item \cref{thm:salvetti_cx_equiv_X_prime} \cite[Theorem 5.5]{paolini_salvetti_kpi1_2021}
		\item \cite[Theorem 8.14]{paolini_salvetti_kpi1_2021}
		\item \cref{thm:subcomplex_K_prime_hom_equiv_K} \cite[Theorem 7.9]{paolini_salvetti_kpi1_2021}
	\end{enumerate}
	\vspace{-3em}
\end{proof}
\vspace{1.5em}

Furthermore, in the same paper another main result is shown.

\begin{theorem}[{\cite[Theorem 6.6]{paolini_salvetti_kpi1_2021}}]
	Given an affine Coxeter group $W$, corresponding affine Artin group $G_W$ and Coxeter element $w\in W$, the complex $K_W$ is a classifying space for the dual Artin group $W_w$, i.e.~$K_W \simeq K(W_w, 1)$.
	\label{thm:KW_classifyingspace}
\end{theorem}

It was already known \cite{brieskorn_fundamentalgruppe_1971} that $\pi_1(Y_W) = G_W$. Thus considering $\pi_1(Y_W)$ and combining \cref{thm:proof_overview,thm:KW_classifyingspace} gives
\begin{align*}
	Y_W &\simeq K(G_W,1)\\
	G_W &\cong W_w
	\label{eq:artin_iso_dual}
\end{align*}
for affine $G_W$.

This proves the $K(\pi, 1)$ conjecture for affine Artin groups and provides a new proof than an affine Artin group is naturally isomorphic to its dual, which was already known for finite \cite{bessis_dual_2003} and affine \cite{mccammond_sulway_artin_2017} cases.

The proof of $\pi_1(Y_W) \cong G_W$ for all $W$ in \cite{brieskorn_fundamentalgruppe_1971} is in German and only German or Russian translations are available. This result is fundamental and non-trivial. Alternative proofs for Coxeter groups of type $A_n$ \cite{fox_neuwirth_braid_1962} or affine type \cite{vietdung_fundamental_1983} are available in English.

\end{document}